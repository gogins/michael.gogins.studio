\documentclass[english,11pt,letterpaper,onecolumn]{scrbook}

%\usepackage[utf8]{inputenc}
\usepackage{babel}
\usepackage{mathptmx}
% Extra leading.
\renewcommand{\baselinestretch}{1.125}
\usepackage{tocloft}
\usepackage{fancyhdr}
\usepackage{scrlayer-scrpage}
\usepackage{ifthen}
\usepackage{keyval}
\usepackage{geometry}
\usepackage{url}
\usepackage{calc}
\usepackage{array}
\usepackage{graphicx}
\usepackage{color}
\usepackage{listings}
\usepackage{supertabular}
\usepackage{scrpage2}
\usepackage[pdftex,
            pagebackref=true,
            colorlinks=true,
            linkcolor=blue,
            pdfpagelabels,
            pdfstartpage=3
           ]{hyperref}
\usepackage{poemscol}
\renewcommand{\chapterfont}{\usefont{T1}{fvs}{bc}{n}\large \centering}
\renewcommand{\afterchapterskip}{\bigskip}
\global\verselinenumbersfalse
\usepackage{scrpage2}
\makeindex
\definecolor{LstColor}{cmyk}{0.1,0.1,0,0.025} 
\setcounter{tocdepth}{9}
\begin{document}
\frontmatter

\title{Selected Poems, Stories, Essays}
\author{Michael Gogins \\ \texttt{gogins@pipeline.com}}
\maketitle

\tableofcontents
\mainmatter
\pagestyle{scrheadings}
\refoot{\pagemark}
\cfoot{}
\lefoot{}
\lofoot{\pagemark}
\rofoot{}
\rehead{Michael Gogins}

\lohead{Poems, Stories, Essays}
\part{Poems and Prose Poems}

\chapter{Prose Poems}

\section{I Sleep}

	I sleep with a naked word between me and my love.  We lie not touching but dreaming of mutual death, our glittering wave that overwhelms the Western cities, the tidal wave a mile high and full of sunlight that shines between the floating houses and the dead things.

\attribution{August 15, 1988...June 1, 1989}


\newpage
\section{The Anti-Monogamist Crusade}

	In the anti-monogamist crusade I've been rancher, soldier, prisoner, urban refugee, and finally poet.  For I have learned the most fundamental trope.  It is dislocation, exile.

	Dislocation separates the meanings of one thing, say love, into the redoubled terms of metaphor:  one flesh, two bodies.  Dislocation analyses motives into the unity of opposites:  promises, power.  And so dislocation foreshadows the false denouements of our Empire.  Despite the barricades of the Monogamists, the capitol is decked in white today.  For today we celebrate our beloved Crown Prince's latest, in fact his seventieth, royal wedding.

	My own complete dislocation has been accomplished not through my being dispossessed of my ancestral holdings by the tanks of the Emperor, not through my forced exile from our mountains of black pine, not through the drugs and alcohol of the slums, and not through prolonged meditation or other ascesis, though I have suffered all of these, but through faith alone.  For only faith endures all change of body and state.

	The wheel of the seasons returns me again and again to my task, as if I am being hammered on the anvil of the Sun.  I am seeing all times becoming one time, all cities emptying into this one city wracked by civil war, and all peoples being pounded, despite themselves, by the clash of religions, into one people.

	When he first began to study Yoga, my older brother lived for several months in a mountain canyon with the half-Cherokee woman from the plains he later married.  Slide showing a half-cave in a red-rock cliff, propane stove and aluminum pots set along a slab of sandstone, plastic tarp for a tent, two smiling suntanned lovers.  Later, he joined me in the capital, in a rented house of mud brick in the middle of a block.  He built gliders of paper and balsa and kept a parrot, but his periodic months of Yogic celibacy infuriated her.  I would sit with them through the winter nights, drink tumblers of red wine in the white lamplight, and look at her olive, high-browed face, for it was really her I liked.  And the children in their pink and peach pajamas, cheap toys in hand, would come forward to examine me with solemn black eyes.

	Eventually, my older brother was recruited by a fiercely polygamous sect.  His wife divorced him in outrage, and returned with her children to the still-rebellious plains.  He went back with his two new wives, as one of the enemy, to our conquered mountains.  When winter came on, he was shot by our younger brother, who'd lost his own wife and lands to an Imperial.

	Now the cold wind off the tundra blows snow so high it covers the first doors and reaches the thresholds of the winter doors on the second stories.  In the loft where I used to work, I know that the windows are steaming up from the breath and perspiration of the operators, whose hands of tender flesh must screw together the titanium hands of robot after robot.  At the end of the street, the black brine grinds its floes of ice against the granite piers.  And through tubular bridges of light strung between the office buildings far above, traders and secretaries in pastel silks saunter to and fro, chewing nuts and spice.

	I came here with nothing.  Yet I am determined to rule, even if it is only myself, my inch square field of will.  Sometimes I go out in my old stubs of charred plywood, the yoke I wore as prisoner of war.  For I like to flaunt my rebellion and escape in the chaos of the enemy's very capitol.

	Fear does not rule me, or limit my limitless ambition, though we all rightly fear the armored divisions.  Nevertheless I have no wish to ape the repetitive conquests of our polygamous Prince with some futile victory of blood.  I am revolted by the sophistries shouted in the markets by professional debaters, and by the riots incited by armed Monogamist missionaries.  So I have chosen my own battlegrounds with care:  poetry and faith, the decisive contexts of action.

	This afternoon, after meditating upon the central figure, I sacrificed a pair of prayer wings.  I touched them with a match, and they blazed up into wings of fire, wings of smoke, wings of glowing ash, which crumbled with the breeze into aromatic dust.

	Now Yekaterina climbs the six flights to my roofless room.  She sees how cold and sick I am in my Mexican blanket, how I shiver and speak deliriously of computers to link each mind, of golden scarabs that we could forge to monitor and feed our brains.  She touches my forehead with her hot indifferent hand, and asks if my escape and my sacrifice were worth it.  I remember meeting her at the gallery where she works downtown.  That night they were showing a backwards-leaning throne made of oak trunks lashed together, far too big for anyone to sit in, and its seat was splintered as if by lightning.

	``Worth it?'' I ask.  Her tongue is so pink!  Above, a shell has torn away the roof.  As the night breeze blows ice clouds across a crescent moon, plaster dust drifts down.  I want to sneeze but am too weak.  I think of campfires in the mountains of my birth, and how my father and I used to sit our horses under black pines heavy with snow to look down into the distant glow of this city, coals banked against the dark ocean.  I think of the huts built everywhere here, their little fires flickering, even halfway up the great thick cables of the suspension bridges.  And I wonder how my sister fares, who lives on the other side of the line of rubble dividing the quarters.  But to visit her I'd have to pass those sandbag booths of dark-faced men who demand payment or a certain name, and I have no name but my own.  Indeed, names are too dangerous --- I have no name at all.  I have only these words, which swirl through my consciousness like the glowing sparks of prayer wings one burns as evening falls.

	Deliriously I speak to Yekaterina of restoring the ruined fortunes of my family through canny trading. of ships I will buy, though our nation no longer has ships except for ships of war.  And my family were ranchers and miners, anyway, who never smelled the sea.  In my fever I describe the suit I'll order, with glittering threads of power, bulletproof panels sewn in its lining, buttons of jewel.  ``Worth what?...'' I mutter, as the fiver rises in me again.  Yekaterina brings me another cup of water from the pump in the yard below.  I cough.  But when she leans over and I sense the heat beneath her soft grey sweater I start thinking of marriage, of dynasty even.

	The burning core which drives conquerors now burns in me, for I have tasted the very dregs of dispossession.  There is no glass in the round windows of this beachfront apartment where I squat with my rough friends, where in the cold noon retsina burned our throats, and where playing cards still lie scattered on the black carpet.  And the sea air in my nose reminds how in the summertime, before the fighting entered the city itself, my sister played her flute for coins on the boardwalk by the amusement pier.  Last I heard she was shining shoes and fetching lunches in the stock market, scurrying beneath its glowing screens.

	Dislocation has taught me that all empires, not only the Kingdom of God but even ours, are founded on the immolation of innocence.  It is why one side of a street is brick row houses with plaster saints and olive-skinned kids hosing down a Pontiac under peach trees in October sun, while the other side is all cyclone fence and railroad tracks, where the bitter wind from the north plains blows trash into the fence, and cardboard huts huddle against the burned-out factories.  But my fever confuses me:  Do I want to be the innocent immolated one, or one of the cursed immolating ones removing their curse by immolating the innocent one?

	Night falls, and battle satellites chase each other across the smoke-streaked blackness filled with stars.  To celebrate the seventieth marriage of our beloved Crown Prince, the lasers of the city fire needles of green light up from the tower tops, as if to defend his bright illusions against nightmares in the outer darkness.  But the black book I sleep on tells me that the only innocent one among us, knowing we cursed ones had no one else to immolate to remove our mutual curse, stepped forward and offered himself, himself and not another, to be bound between the oil-drenched wings of sacrifice.

	I enter more deeply into his act as I taste Yekaterina's second cup of cold water, its grains of wet plaster gritting in my teeth.  I suppose I should be glad she even thinks of me among our teeming surplus of unmarried men.  Now she's saying, ``A girl I know has been asking about us, if I would mind, and I said no....''  I think of how I have so often contemplated, and even oftener postponed, my marriage.  I remember mountain weddings of few words said once.

	Anti-aircraft guns are clapping from the edge of the seaport, firing either against the monogamous quarter or in celebration of this latest royal wife.  Yekaterina's face, her gallery girl's face, bent with momentary tenderness over mine, is lit by green flashes on the left side and by orange flashes on the right.  I refuse her wildly, I refuse them all, I feverishly refuse till I should not only be loved but also love, once and only once, in return.

	And I do not know if I am foolish to do so, but it certainly loosens my tongue.  I have learned the most fundamental trope.  It is dislocation, exile.  Dislocation separates the meanings of one thing, say love, into the redoubled terms of metaphor:  two bodies, one flesh.  Dislocation analyses motives into the unity of opposites:  power, promises.  And so dislocation illuminates the false denouements of our Empire.  Despite the barricades of the Monogamists, the capitol is decked in white today.  For today we celebrate our beloved Crown Prince's latest, in fact his seventieth, fallacious royal wedding.

	The whole city is white, white everywhere, except in the one most needful place.  For the Imperial groom and his seventieth teen-age bride, in the tinny box of television, in their heresy, are not wearing the white wings, the paper wings of monogamy.

\attribution{October 15, 1988...June 3, 1989}

\newpage
\section{\emph{Jupiter and Semele}, by Gustave Moreau}

He looks like a brute all right, slouching with a pout, with the red halo of a Satanic Christ, and dark smoldering eyes.  He's almost a beer ad or a faster, purer kind of car.  If only he weren't fourteen feet tall!
Semele stretches herself from his hip, perhaps she is trying to get away, and the clasp of his girdle regards her with a living eye of onyx that holds her in its grasp.  Or perhaps the white bow of Semele is stretched on his hip like his harp's mute box, that hides as much of his shadow as it can inside its square of geometry, the square of knowledge that resounds with shadow.

He muses:

Babe, you're such a waterfall of skin, Babe, your thighs are as white as my dream as I dream it, Babe, your skin is real but my dream unfortunately is not, Babe, so I have your whiteness stretched out over the night, and I am painting without thinking, but only more white, white over white, white in white.  I like to paint at night, and I paint white on white, and my frame is black and crusted, and my shadow is full of music, and I have our skeletons to shake at will inside my box of tin.

And when I stretch my own fabric thin enough for your bed, I'll become painful droplets that give mortal life.

\newpage
\section{In the Head of the Idol}
	My enemy, my son, I have forgotten your year.  I do not believe in reincarnation, but I am lost in the echoes of my father.  I confuse myself with him, and with his father, and sometimes you with me....
	
	At first I hesitated to name our true oppression.  It is shameful to want another's guilt for one's own, and without realizing it.  In the can I found out back there was a layer of your dead brothers and sisters under oil.  We also (naturally, in the course of this game, ``I'' becomes ``we'') --- we also have the fire of books and the smoke of false love, but nothing can hide the splendor of your irremovable lamp.  It makes us stick to the walls like its shadows because we are afraid to face it.
	
	If for one moment I ceased lying to myself I would resume the shape of Adam.  Where in the alleys would you find paradise then, Eve?  Now that the serpent is gone, we are reduced to corrupting each other.  I've hidden myself in this rented room, and I have a job in a garage dipping the new engine in solvent, but I'm afraid, my son, that you will find me and kill me for my key.  You wouldn't recognize your mother among all the others, but you'd know me.

	Since I was born I have forgotten all the angels of pre-existence and their swords except for one image, the shadow of a colossus falling over a battlefield from one end to the other.  Everybody was either dead or sleeping, and smoke arose from the sodden green trenches.

	When I pushed the shop doors open this morning I found the lathes had been working all night by themselves.  They had produced an enormous idol with steel fists for destroying other idols.  I must have programmed the lathes in my dreams.  Now I climb the rungs up his side into his huge crystal head, sit on the chair within, slip my feet into the boots, my hands into the silver gloves.  I don the soft helmet and the goggles of blue television, and pull the canopy down over me.

	Dust boils up around my first gigantic step.  I cannot stop trying to make myself into something more than a human being.  Soon I will give myself gills and wings, put voices and images in my brain for linking with machines.  I will join the ancient conspiracy of the alien nations, the nations that crept out of the black holes remaining from the fallen universes.  We will clement our rebellion by a treaty written in the chemical signatures of our mingled blood.

	Nevertheless, my son, the memory of your feet pursues me.  On the mountain you were pierced at my command and for that reason in whatever fire I dwell I will always taste your last kiss, and my blindness will be no defense.
\attribution{July 1, 1988}

\newpage
\section{The Resting Places of Sisyphus}
	Through its constant falling back and his rolling it forward yet again, a little further reach time, his boulder has become light to him, and Sisyphus has attained the top of his slope.  It is a valley which affords him rest.  The yellow sky is reflected in a stream and its ponds, geese ``vee'' along the jagged horizon.  There is a road, commerce, arches, taverns.

	For several days he rests and then --- again the slope, steeper and higher than before.  And so it goes, seasons revolving into years.  Hills, each higher or lower than the last in no discernible pattern, alternative with valleys, each peopled.  After languages and customs have blended into the blue of his eyes, Sisyphus conceives of watersheds, of an ocean, of a final valley in which he will scent at last the salt breeze that will end the commandment of his journey.  There he will leave his boulder behind, utterly indistinguishable on the rocky shore.  And there, at the mouth of a river, he will find a tile-roofed town that never heard of the tricks a king too clever for his own good once played upon the gods, a town with perhaps a few widows of seamen, and a window with a lamp.

	But the fear that these hills and valleys are only ripples on the waves of yet greater hills, repeating and expanding to infinity, arouses disquiet, becomes worse torture than that first slope Sisyphus so naively had thought unending.  For the intelligent well know that infinity is neither odd nor even, neither slope nor valley.  Does, then, this cordillera of brown and red peaks, this series of roads and tongues, neither end nor not end?

	And the geese, should he understand them as his comforters or as his tormentors?
\attribution{December 13, 1987...June 1, 1989}

\newpage
\section{The Language of Hunger}
	There is no substance to my life.  I have no myth, no story, no goal.  I live in a room in a great city and work in another room.  The light changes outside the windows, hour by hour and day by day, but it is all beyond me.  My friends are faces passing behind a glass screen.  I am aware of the wrongs of the age, but I do nothing to oppose them.  I am passive, waiting, depressed.

	I recall scraps:  white clouds over a lake, thighs opening, lamplight on a wall, conversations at night under the trees while the wind blows.  These scraps imply that I do have a life, but they do not connect.  They only remind me of my lost substance, so that rays of longing run out into the darkness to connect me with myself.  But the rays do not gather to any bright point, any star.  And so I remain in myself, not suicidal, not content, and not forgetting.

	I used to have adventures without purpose:  following a woman around a corner, reading a book pregnant with someone else's idea, boats.  Now all memories lie under a vast sea of ice, dead leaves trapped beneath autumn after autumn of frost.  I await some spring which will melt the ice and release the leaves to their colors, their flood, their end, the tender green their death will feed.

	I am suffering, but not for the wrongs or sins of others.  My pain does nothing to relieve another's pain, though I wish it did.  This longing has about it a tinge of blue, as though the gritty tops of the buildings felt a touch of warmth, as though the sunlight with which spring shines through the green trees along the tracks fell on naked backs, on those lying down, on children.

	I think of the book which will reveal all the other secret sufferers and interpret the tongues of the hidden angels.  I know that my task is to write this book, though of all who write I am the least like an angel.  There is no unselfishness in me, no generosity, only this mute hunger for the bread some fabulous sky might let fall, some port of peace in which empty boats might rock, the antipode of the darkness in the killing basement and under the lonely bed.

	This is a hunger which is blind and dumb, which does not even know the name of its food, which feeds on hollowness and becomes hollower, and yet which tastes the lies in the circling screech of the subway wheels and the untruth in the hum of the fluorescents in the office.  This hunger has searched the open mouths of lovers, the tumultuous light of their sheets, the scattered food on tables after dinner, and it has followed after the poor people hurrying to the fields in which they will labor.

	This hunger is deeper and mightier than I am, and when it forces open my mouth I taste that emptiness so profound its echo is greater than speaking.  I hear the directionless mutter of the homeless, of ghosts, and of characters in stories.  But these voices unite only in the hiss of the shell, the feeble thunder of silence; and the flashes of light which strike me are only sunlight reflecting from the windows as I leave the office.  Silently, I pass many people in the street.  And though I cannot yet speak, I know that the language I am learning will express everything in our walk, the junk in our purses, our briefcases, our shadowed eyes, and our secret wars.

	It is then that I know I am an army, and that I await the sharp tongue of light in the mouth of the sky, the cry of command which will break the silence, and the manes of the white horses.
\attribution{May 24, 1988...June 3, 1989}


\chapter{Verse}

\newpage
\section{Piano Microscope}

\begin{poem}
\begin{stanza}
My inflicted self's\verseline
Blow-blinded eye's no ear\verseline
Yet hears the sparrows on black wires\verseline
In rain as grey as rain\verseline
From fragile ink of sky
\end{stanza}

\begin{stanza}
Let my torn Bible hide you\verseline
In her Martian shadows\verseline
With clock wings of feathery steel\verseline
And feel the mask of quilt\verseline
To sleep with beating hands
\end{stanza}

\begin{stanza}
For the city of music\verseline
Is one white room\verseline
Where a child overinterprets guilt\verseline
And skips through bars of light\verseline
In sudden beats
\end{stanza}

\begin{stanza}
So rise, pilots\verseline
Through circumstance of blank\verseline
And with piano microscope\verseline
Search the stockings of sound up\verseline
To their soft detonation
\end{stanza}

\begin{stanza}
Where the deepest whisper turns\verseline
A brass knob without a legend\verseline
To open the bright mouth of duty\verseline
That shows no rank but \verseline
Obligation
\end{stanza}

\begin{stanza}
Till that tender monster my blue eye\verseline 
Roams in sudden streets of light\verseline
Cutting shadows down to make soft ice\verseline
And turning rays away\verseline
Back into ink
\end{stanza}

\begin{stanza}
For when we die\verseline
There are no \verseline
Audible prayers any more\verseline
Just darkness full of \verseline
Answer
\end{stanza}
\end{poem}

\newpage
\section{Gently}

\begin{poem}
\begin{stanza}
Your face is a burning window in this rainy night,\verseline
My hands are looking blindly for your name,\verseline
My heart has a mouth for it crammed with honey and shame,\verseline
My voice is a broken commandment.
\end{stanza}

\begin{stanza}
Love's the worthless thing everyone dies for,\verseline
The beggar who rules the world,\verseline
This burning dump whose flames dance forever,\verseline
Red slaves always dying chained to their dance.
\end{stanza}

\begin{stanza}
Love, the destitute emptiness,\verseline
Love, the riddle crying for its lost voice,\verseline
Love, the enemy, love, the fool,\verseline
Love, the rebel king, lowering himself to serve everything.
\end{stanza}

\begin{stanza}
The older I am the emptier I become,\verseline
Full of trembling air,\verseline
I, the mere echo of my broken vows,\verseline
Words spilled everywhere.
\end{stanza}

\begin{stanza}
But love's still the song at the end of the tongue,\verseline
Syllables dancing naked and perfumed.\verseline
And love is the war at the end of the world,\verseline
The warrior loving the warriors.
\end{stanza}

\begin{stanza}
The drowned book will swallow up its sea,\verseline
Dead ink reanimate dead mouths.\verseline
The mirror will shatter its diamond,\verseline
And the Sun will part the sky.
\end{stanza}

\begin{stanza}
The kiss that strips away the masks of the face,\verseline
The touch that puts all the useless things down at the end of the world,
\end{stanza}

\begin{stanza}
Gently\verseline
Will touch me.	
\end{stanza}
\attribution{January 14...June 1, 1989}
\end{poem}


\newpage
\section{Black Scum in a Silver Cup}

\begin{poem}
\begin{stanza}
Going to share my life with you.\verseline
Here's my life, black scum in a silver cup, blood that's not half so clean as the pretty blood of Jesus.\verseline
If you want to share my life, you have to drink this cup, just as Jesus drank it down, just as I tell myself I too some day must drink, I really must force myself to share your black life.\verseline
And I will, for Jesus is helping me drink, he is standing beside me and inside me, his hands guide my hand to the blinding cup.\verseline
Be like Jesus to me, drink my cup, help me be like Jesus to you, help me drink your cup, there is only one cup.\verseline
In this cup there is reflected only a whipped face, for the veil was torn when the cup was poured, and God revealed a human face poured out.\verseline
Going to share my life with you, in the human face of God poured out, the bitter wine of the twice-born.\verseline
Then we will be able to sing the song we are already whispering, and how crippled we are will become a sort of dance, a sort of rejoicing.\verseline
There is nothing that will make you more drunk that this cup of black scum.\verseline
You could marry a cloud covering the Sun, a woman with no brain, a man with no legs, a child with a crippled canary.\verseline
Maybe you're thinking you'll never get married, but drink this cup with me anyway.
\end{stanza}
\attribution{December 16, 1988...June 1, 1989}
\end{poem}


\newpage
\section{Evil Wing}

\begin{poem}
\begin{stanza}
The evil wing gives off a terrible negative heat, wills not to be known, pushes away the inquiring will with sick thoughts, a cloud of evil.\verseline
It would kill anyone to remain peaceful within me, larval, growing, dominant, asleep.\verseline
Nobody understands what it hums backwards in music, nobody understands what it paints in shadows, nobody understands its peaceful pain, nobody understands how it has always been rejected, nobody dreams the black sugar of its shame.\verseline
I will kill it by loving it, by bathing it with my tears, by feeding it my blood, by unfolding its bitter red wing in floods of clear light.\verseline
I will kill it by holding it to my chest and climbing into the arms of Christ, the bright lullaby of the nails, the white wings of the ever-folding cross.\verseline
I will kill it by giving it life, unconditional, stark as a spring on the Moon, foaming, dreaming of beer, of salmon swimming, of garden parties behind Lunar houses where nobody understands my Earthly tongue.\verseline
I will kill it before it kills me, and I will give birth, and I will pull myself out of my own side, and my moon will be astonished at the sudden birth of its Sun.
\end{stanza}
\attribution{December 17, 1988...June 1, 1989}
\end{poem}

\newpage
\section{Horn I Sweep}

\begin{poem}
\begin{stanza}
I am free in language as a tart. \verseline
I have a horn, I blow it to blow down your walls. \verseline
I blow, I blow down the walls of your sad closets, \verseline
Your wardrobe of guarded and desired dreams, \verseline
Your negligee of fog and childhood, your rocking horse desire. \verseline
I blow my horn free as a tart and I blow mean, \verseline
I shatter fogs with the hammer of memory and release black steam. \verseline
I blow in sleep with a circular motion, I am the motion, I am the red and infantile commotion. \verseline
I am bold and blue, I drink down music like air, I blow to describe you, to explain you, to shine on you and from inside you. \verseline
I have a hat I put on my horn to play a prank with time. \verseline
I dance back and forth with the enemy, the idiot, the crime. \verseline
I have the radiant closet of the dreaming bed, I find the skin in memory, I find the sin I sleep in, I awaken myself and I dance heavily and proudly on the drum of sleep with my sleeping giant. \verseline
I take up this horn and I sweep out the black tree of dreams, I sweep out stars with its sound, down fall shining cities, memories. \verseline
I get older and older, I wax musical in the shining moonlike bell of my blowing.  I am older with power, and I remember freely, without a hold on you, no price, just the memory of you without a name.  But your name, your precious name, three notes I blow from the touch of your belly. \verseline
I find where you live without knowing you live, and I tenderly play you a fence, a hedge of law, a tower of reason, a burning sword of shame to protect that lawless innocence you forgot, your dreaming, thoughtful, shining spring.
\end{stanza}
\attribution{October 20, 1988...June 3, 1989}
\end{poem}

\newpage
\section{Idiot Perfect}

\begin{poem}
\begin{stanza}
The future is idiot perfect, a woman with red horns. \verseline
She dances with her tail on the glass table of sleep. \verseline
The future, idiot perfect, scorns us... \verseline
All machinery making flour is out of control. \verseline
Drowning in cake, you dream the golden barn, \verseline
Its green doors, its spring breeze \verseline
Touching along your forearm. \verseline
There one speaks the tongue void as a cannon \verseline
Wailing.  You're lost, a ship in time --- \verseline
Black as a sea, your hollow boat \verseline
A sun-hidden farm.
\end{stanza}
\attribution{September 13, 1988...June 1, 1989}
\end{poem}

\newpage
\section{Armor of Memory}

\begin{poem}
\begin{stanza}
Trees are black fans, fly wing lace, whiskers, scratches on the purple armor of memory. \verseline
Cornices and roofs of cardboard bow against the overthundering ship of the white moon sailing. \verseline
Now is the time of yellow peeks glowing, the old wood of hidden balconies, unworkable doors. \verseline
Think of spelling, the ledges of touch dust, the sound of skin writing lonely letters up the smell of linoleum stairs. \verseline
Now your disheveling gold eyes are seas with two small black boats, I will drown in such bottomless boats. \verseline
The water of twilight pours in the windows, \verseline
Invisibly drowning your silent battle of books along the wall, your neglected plants, your clear distant stars. \verseline
The drowning piano marches through the water of dreams, the fate-engine spews out its bright salts as it sinks. \verseline
And my whole hull will split in you.  I have always longed to spill this treasure, \verseline
This golden treasure to light up your black depths.
\end{stanza}
\attribution{November 18, 1988...June 1, 1989}
\end{poem}

\newpage
\section{Judas Come Back}

\begin{poem}
\begin{stanza}
The twelfth hidden one \verseline
Is Judas, remembering.
\end{stanza}

\begin{stanza}
His train of nights is a strange bed, \verseline
The smell of the sea from open windows, bad lights.
\end{stanza}

\begin{stanza}
They flash on and off on his open shirt ---  \verseline
He touches the scar that circles his neck, wondering.
\end{stanza}

\begin{stanza}
Nobody ever lived except in a body, \verseline
Yet its wounds are always deeper than flesh.
\end{stanza}

\begin{stanza}
And he'll have no more children, though he wishes, \verseline
Even hides the book he writes his wars down in.
\end{stanza}

\begin{stanza}
He goes out.  There are those park stairs again, \verseline
The trees tossing their bronze tops above crickets.
\end{stanza}

\begin{stanza}
He thought he'd done everything, that he was everything, \verseline
But there is always more, always.  
\end{stanza}

\begin{stanza}
This is the mill stone \verseline
That no one sees.
\end{stanza}
\attribution{September 10, 15, 17, 1988, February 5, April 26, 1989}
\end{poem}

\newpage
\section{Conjuration}

\begin{poem}
\begin{stanza}
I dance on the dangerous black floor of your eyes\verseline
As you lie dreaming, looking up at your blank white ceiling.\verseline
Clouds of desire and of forgetfulness drift across\verseline
You like arms playing the piano more slowly than time.\verseline
Tango your rolled up stocking showing, tango my black Fedora,\verseline
Tango cloudy scuffed wax with steel tap shoes.
\end{stanza}

\begin{stanza}
In evil magic, against a turquoise wall, with brassy music,\verseline
We perform the dance of conjuration.  (Without a fortune\verseline
And disinherited, I applied myself to learn the keys\verseline
And the names of the objects, the roots, the desires.)
\end{stanza}

\begin{stanza}
That is why I dance, proudly, patched, unshaven, against a\verseline
Turquoise wall, and why you dance with me, for I have your name\verseline
In its crib of white lace, its crib of white lace.
\end{stanza}

\begin{stanza}
I understand so well how little we may have each other\verseline
That we may have each other so much more than the others.
\end{stanza}
\attribution{October 8, 1988...June 3, 1989}
\end{poem}

\newpage
\section{Farmington Bay}

\begin{poem}
\begin{stanza}
I am the least possible woman, I am the simplest man.\verseline
Since I am an atom, therefore I create a field.\verseline
I am an atomic clock, a clock one atom wide, there is only room for one feeling, a feeling the color of light.
\end{stanza}

\begin{stanza}
With brush of music to paint the sky of sound, where ducks fly drums across the green, and the sun goes down the mouth of the black violin.\verseline
In a wooden house you make a wooden house to have a wooden house in case it rains.  It rains memories of a face.  The face is a sweet place to trespass in the underside of things.\verseline
In a wooden house you make a wooden house to have a wooden house for when it burns.  It burns in china organs and it burns in paper reeds.  I lean my shotgun by the lamp and I know how you sound.\verseline
So many birds in your thighs, the rope of your hair to take the sky.
\end{stanza}
\end{poem}

\newpage
\section{Lictor}

\begin{poem}
\begin{stanza}
Wandering in this toothpick town with anvil feet and banjo,\verseline
I'm a cardboard cutout animated by relays and solenoids.\verseline
But my eyes are always shooting for the flesh beyond the fence.
\end{stanza}

\begin{stanza}
With every shot a puff of dust from the dusty old bed\verseline
Where a shell of lace the size and color of a leaf\verseline
Just born tries to hide your ancient face.
\end{stanza}

\begin{stanza}
It's the wrong size\verseline
For the world but I'm adjusting the world to fit you.
\end{stanza}

\begin{stanza}
With every step I split in two in mind as well as body,\verseline
I go to war with armies of forgotten mes and win and lose forever\verseline
But every flag and every torch reminds me of when
\end{stanza}

\begin{stanza}
I was whole, when I could touch you with my eyes,\verseline
And the rain's dream defined a roof of glass\verseline
For the sleeping body of the fog --- just a little smile please ---
\end{stanza}

\begin{stanza}
You and I rolling over like ocean liners burning at sea,\verseline
Flames running out over the black water, up and down the waves.
\end{stanza}

\begin{stanza}
All's fair in love and war:  therefore I make this language,\verseline
Itself a spy, itself a message, itself my general strategy.\verseline
With enough rhythm it will get a body together, and
\end{stanza}

\begin{stanza}
Coalesce right out of ash, colored fog with arms held out in front.\verseline
All over my linen wrappings dance black insect letters:\verseline
Love letters from forever.  Do not be afraid.
\end{stanza}

\begin{stanza}
Forever is the sound of one word laughing,\verseline
Forever is the sound of every lamp going out.
\end{stanza}

\begin{stanza}
So unroll my head to see if grey marmalade\verseline
Wrapped in tissue of bone can think of a stone\verseline
Clear enough to hold the sun of your memory.
\end{stanza}

\begin{stanza}
Orange tinge on cloud bellies,\verseline
On the ceiling of a room,\verseline
On white skin rolling towards me,
\end{stanza}

\begin{stanza}
Your flag \verseline
Of surrender, my vacant field
\end{stanza}

\begin{stanza}
Of what I will try repeatedly to name\verseline
In a way that fireworks boxing in purple clouds\verseline
Only vaguely indicate.
\end{stanza}

\begin{stanza}
Target within arrow,\verseline
Bird bringing cage tiny silver bell in beak,\verseline
My heart on a string, clotted carnelian.
\end{stanza}
\end{poem}

\newpage
\section{Long House}

\begin{poem}
\begin{stanza}
There was a block of houses, or a long apartment house, I think it was on Third South west of West Temple, in Salt Lake City.\verseline
I think Indians and Gypsies lived there.  I saw women sitting in white iron chairs with big white arms and white print dresses but their skin was dark their hair was dark.  I saw old brick crumbling I saw tarpaper roofs I saw chimneys against a dark blue sky I saw yellow windows I saw sprinklers arcing on the lawn I saw bare bulbs inside the yellow windows I saw yellow blinds half pulled down.  I wanted to live there.  I wanted to smell the sprinkler on the cool night lawn of summer and I wanted to go up steep wooden stairs and into a room with a naked bulb and a woman different from any woman I'd ever met.  Or no woman.  Crickets outside.  Railroads.\verseline
I wanted to be able to say I had lived four thousand years.  I wanted to be able to say wooden stairway.  I wanted to be able to say naked bulb.  I wanted to be able to say my mother's name.\verseline
Smell of sprinkled fresh cut lawns in summertime.\verseline
Sound of radio from an open car door.\verseline
Crickets singing in the cottonwoods.\verseline
Railroads.\verseline
Students from China men from Basque young men from Cedar City.\verseline
Sunrise Cafe.\verseline
Acid drifter don't run away with my sister.\verseline
Oklahoma Prison return address love letter blue paper God save us God bless.
\end{stanza}

\begin{stanza}
August night sky blue pit seething cottonwoods and elms.\verseline
Smell of sprinklers hot asphalt sound of television and sticky rubber.\verseline
Across the hot and dusty tracks,\verseline
On the drunk side of town
\end{stanza}

\begin{stanza}
I do not care if you will kiss me.\verseline
I do not care if you will suck my cock.\verseline
I do not care if you will remember me.\verseline
I will talk, I will strip in your bed, I will go away, I will do none of these things.
\end{stanza}

\begin{stanza}
Old cotton undershirt faded not quite white\verseline
Smell of linoleum cricket song\verseline
Skin washed but still a little sweaty\verseline
Lift your arms above your head
\end{stanza}

\begin{stanza}
Every history every morality\verseline
Cup of tea on kitchen table\verseline
Bulb light on the painted white\verseline
Eyes in eyes.
\end{stanza}

\begin{stanza}
Smell of sprinklers on dusty grass\verseline
Smell of cottonwoods exhaling into evening\verseline
Smell of gas burning across the junkyard\verseline
Smell of kerosene from the jets above
\end{stanza}

\begin{stanza}
Row house under diseased elms\verseline
Saturday night August sprinklers going\verseline
Windows without curtains lamps without shades\verseline
White iron chair on the porch without a roof\verseline
Brick without color brick skin brick soul
\end{stanza}

\begin{stanza}
Bed without a blanket\verseline
Dresser without a mirror\verseline
Lift your arms above your shoulders\verseline
Show me
\end{stanza}

\begin{stanza}
I do not drink but I am drunken\verseline
I speak without language\verseline
I am so other\verseline
I am in you
\end{stanza}

\begin{stanza}
All the memorized positions are useless\verseline
The springs are rusty and they sing\verseline
I cannot make a word mean one thing\verseline
I am saying I love you
\end{stanza}

\begin{stanza}
Force what together\verseline
The raised rings of wood on the dresser top
\end{stanza}

\begin{stanza}
In this attic the sound of the radiator\verseline
Lift your arms above your shoulders\verseline
Cotton lifts above your breasts\verseline
Bright snow outside
\end{stanza}

\begin{stanza}
There is no need for unity\verseline
Relax you too will die\verseline
The circle will be within you\verseline
Ripples in the darkness silver as Sun
\end{stanza}

\begin{stanza}
Someone says ``I am''\verseline
And you are.%
\end{stanza}
\attribution{}
\end{poem}

\newpage
\section{Love's Laboratory}

\begin{poem}
\begin{stanza}
In the blackest room of all I stand,\verseline
I am the darkened mirror.
\end{stanza}

\begin{stanza}
But as God approaches me\verseline
My silver fog clears suddenly.
\end{stanza}

\begin{stanza}
For she is beckoning\verseline
With her golden candlestick.
\end{stanza}

\begin{stanza}
And the candle is a figurine of me,\verseline
The wick of my head's aflame,
\end{stanza}

\begin{stanza}
The wax of my flesh \verseline
Runs down her hand.
\end{stanza}

\begin{stanza}
With her other hand she opens the door.\verseline
Night's black cardboard falls gently over.
\end{stanza}

\begin{stanza}
It is day.\verseline
Crickets sound in the dry grass.
\end{stanza}

\begin{stanza}
Above the railroad ravine rise all\verseline
The tar-black turrets and chimneys faintly smoking.
\end{stanza}

\begin{center}\rule[3pt]{2in}{0.5pt}\end{center}

\begin{stanza}
So I found my Indian apartment\verseline
Where there are no more Indians,\verseline
On the dry hill over the old library\verseline
Where bums pass out on the tables muttering
\end{stanza}

\begin{stanza}
With the rot of dreams and broken names still crammed in their mouths.\verseline
And I catch them bowing under that weight\verseline
With the scorpion splendid in my plastic tieclasp\verseline
And the parakeet's bright cage on the windowledge in the sun high above me.
\end{stanza}

\begin{center}\rule[3pt]{2in}{0.5pt}\end{center}

\begin{stanza}
And I drive a silent cab in Hollywood.\verseline
Even the Scientologists never speak to me,\verseline
Though I know all the hotels and I know all the hills.
\end{stanza}

\begin{stanza}
For L.A. is still a walking town,\verseline
There are stairs full of dog piss between the dark junipers \verseline
Down the steep sides to Chinatown.
\end{stanza}

\begin{stanza}
And the building on the hill by the library\verseline
Has a glass scallop awning and wavy glass doors\verseline
That swing on grimy hallways that swim in yellow glare
\end{stanza}

\begin{stanza}
To the prow of the corner room where I sleep and sail \verseline
The sea of lights below me in the Valley of Angels\verseline
On a breeze of bougainvillea and exhaust and raw gasoline.
\end{stanza}

\begin{stanza}
And the red and blue and green eyelashes stroke down \verseline
The soft black cheeks of the streets.
\end{stanza}

\begin{center}\rule[3pt]{2in}{0.5pt}\end{center}

\begin{stanza}
These days I make love alone \verseline
Between sleepless white walls\verseline
With a name that does not love me.\verseline
Name, where did you come from, name, how do you live?
\end{stanza}

\begin{stanza}
Name with black roots searching my heart,\verseline
Name with the breath flower stolen and given,\verseline
Given and stolen, Oh God\verseline
Help me live and not die.
\end{stanza}

\begin{stanza}
For I cannot stand against memory \verseline
And I have no magic against the girl \verseline
Who invades my dreams, her face, \verseline
Her sweet narrow hands.
\end{stanza}

\begin{stanza}
She turns away from me,\verseline
She is milk and honey, marble and bronze\verseline
Looking down sunlit\verseline
Steps.
\end{stanza}

\begin{stanza}
Now I am only a broken mirror and the bodiless hand that always opens this door.\verseline
From the black cabinet \verseline
It seems I must always take my doll of flame,\verseline
My golden picture of eyes.
\end{stanza}

\begin{center}\rule[3pt]{2in}{0.5pt}\end{center}

\begin{stanza}
But when she comes at last to me, it is and is not she.\verseline
My solitary cell has become a laboratory\verseline
Where someone larger than I ever knew \verseline
Is working deep within me.
\end{stanza}

\begin{stanza}
For love has a laboratory abattoir\verseline
Where she takes your hair\verseline
Apart and the air is bright\verseline
With knives and lights.
\end{stanza}

\begin{stanza}
Her metal finger \verseline
In your silken brain\verseline
Stirs the dreams up \verseline
And the pain.
\end{stanza}

\begin{stanza}
Then on its column of steel her table tilts\verseline
Till the dead body spills \verseline
Onto black and white tiles\verseline
For love to dance with in high heels 
\end{stanza}

\begin{stanza}
And nothing else\verseline
Till love has done her dance and gone.
\end{stanza}

\begin{center}\rule[3pt]{2in}{0.5pt}\end{center}

\begin{stanza}
And even in death 
\end{stanza}

\begin{stanza}
I am being experimented upon by love.\verseline
The electrodes glitter, their sharp points needle my skin,\verseline
The electrical wheels begin to spin, faster and faster.\verseline
Love jumps up and down, chattering to her familiars,\verseline
The lightning she loves is gathering.\verseline
In her black and white nest of shadows and apparatus\verseline
Love will at last reveal her ghastly ancientness.
\end{stanza}

\begin{stanza}
So why should love fling away her white coat,\verseline
Why should love disdain the blood that covers her,\verseline
Why should love sing as she operates,\verseline
Why should love be so beautiful?\verseline
Why should love sew my parts back mixed\verseline
With entrails and limbs of creatures I hated,\verseline
Why should she shock me to life I can't live?
\end{stanza}

\begin{center}\rule[3pt]{2in}{0.5pt}\end{center}

\begin{stanza}
Love's laboratory has a golden key, fashioned of tears refined incessently, the tear within the tears, the tear of repentance, the tear of glory.
\end{stanza}

\begin{stanza}
In love's laboratory there are windows of snow, servants asleep in sunlit bottles, a teakettle nosecone toppling from the stove.
\end{stanza}

\begin{stanza}
In love's laboratory the instruments are shrieks on a black scale.
\end{stanza}

\begin{stanza}
In the alembic of emptiness, as a torch of black death, I acknowledge my sin, I bewail my deep malice.
\end{stanza}

\begin{stanza}
I sleep in the red oven, let the fat run off me and burn in the flames, become a black rock with a carved smile.
\end{stanza}

\begin{stanza}
And in love's laboratory I am crushed, in the crowded pages of her dictionary, by her thousand tons of history, in the diamond jaws of mystery, till the weeping cloud of me becomes a stone
\end{stanza}

\begin{stanza}
And I reflect\verseline
The sun alone.
\end{stanza}
\attribution{}
\end{poem}

\section{When I looked into the pit...}

\begin{poem}
\begin{stanza}
When I looked into the pit I saw my name weeping in captivity, but your name was in the pit also, silent, dark.\verseline
Your name would not permit me to escape until I felt the sorrow of my captor.  Your name was an infinity of sorrow, dead people in ditches, abortions, easy excuses.  Your name was an infinity of pity, even for evil which delights in evil.\verseline
Your name was dark to me, but there is no comfort anywhere else, so I took refuge in it even though it destroyed me.
\end{stanza}

\begin{stanza}
You rejoice with bankers, and exult with tank commanders, and drink beer with torturers.\verseline
You are silent with the silent ones, you weep inside the dead child's stopped heart.\verseline
You took the burden of human happiness upon yourself.  You experiment with it like a child.
\end{stanza}

\begin{stanza}
One word and winter would become spring, one word and time would freeze or run backwards.\verseline
Thieves steal from you to give to you, for you are begging for love in their thievish hearts.\verseline
Murderers trample on your breaking ribs to glorify you in the dead churches of their own hearts.
\end{stanza}

\begin{stanza}
Whole cities long to forget you, your name is not spoken there, it is put to uses of murder, burn them stone them shoot them, your name the rag of evil, your name the humility of death, your name that refuses to leave anyone, the towers of ignorance that stand because they stand on your name.
\end{stanza}
\attribution{June 12, 1990}
\end{poem}

\newpage
\section{Judge of Sleep}

\begin{poem}
\begin{stanza}
God really did appear on the side of men and women, the side of themselves they call nonbeing because they never dare look into it.
They're always ending the world in there.  I'm telling it to you straight as I can.  You're always ending the world in there.  I'm always ending in the world in there, and Jesus is always wrestling with Jacob and me in there.\verseline
You judge, you judge of sleep.  You executioner, you tender, tender executioner with your masklss mask.\verseline
My death insists on a second birth, and I will drag myself out of my own side with a chain of spit and blood.\verseline
I'm not self-willed, the self that's dragging me out of myself is the dark self in my dark side.\verseline
Even if I go crazy, I'll go crazy wrestling with love.\verseline
I won't go crazy at all, for God is on my side, hauling, hauling like a baby hauls.  It's Christmas, and the baby Jesus is hauling me through a fragrant world of straw.\verseline
Jesus and I, we remember you, we remember your world that comes before and after sleep, our tiny fists are curled in remembrance of the world that comes before and after.
\end{stanza}
\attribution{December 16, 1988...June 1, 1989}
\end{poem}

\part{Stories}
\chapter{The Unbelieving Singer}

	The house was a forty minute walk from the village. The land was gorse and pine, with cold mountain streams. At the edge of the bluff, meadows looked out over the Surmang Fjord to the ten kilometer wall of Palisade Range, dazzling in the light of the sun. The light was somewhat bluer and brighter than that of the Sun. 

	The largest stream, almost a river, pooled in a lake before pouring over the edge of the bluff and into the wind, which tore it into veils of mist before it could reach the salt arm below. 

	At the edge of the lake, on a ledge of weathered granite, stood a house of unpainted wood and piled slabs of stone, chinked with glass. There was a stable a few hundred meters from the house, a structure of fiber and plastic. A starship lighter perched at the far end of the meadow, in grass not burned for a long time by her jets.

	Birds, though not feathered, flew about in the meadow, seeking their prey, and singing. Their voices were more like bells than pipes. But the wind sounded exactly like the wind on Earth.

	The house contained four rooms: bathroom with sauna, kitchen with wood stove and freezer, bedroom with quilted bed and another wood stove, combination living room and study with yet another stove, sofa and armchair, an oak table with four chairs and a kerosene lamp with a milk-glass globe. A desk was set against the largest window. The back wall was bookshelves, filled with ancient books and music manuscripts, and a gun rack. Into the desk's slanted top of polished native hardwood was laid a computer surface. Large loudspeakers stood on stands to either side of the desk. 

	Everything in the house was clean, spotless, shining, and quiet. A robot stood in the kitchen, having finished its tasks, the slight scanning of its eye the only sign of life in the house. A breeze blew through an open window, smelling of something not too unlike mint, and pine.  The air was cool in the shadows of the house. A mechanical clock ticked in a corner. 

	On top of the clock was a picture of a woman, a painting not a photograph, taken on the terrace of an apartment in New York, on Earth. She leaned against the railing, her hair blown partly across her round-cheeked face, her eyes shadowed by the Sun.

	The air of the house was of intense quiet and solitude.

	The village was a collection of houses among the trees on the side of a hill, also overlooking the fjord and the Palisades. A church, a library / auditorium, and an armory faced each other across an area of beaten dirt. Two restaurants, a bar whose strongest drink was dark beer, and a bank shared the square. 

	The armory was the most substantial structure, of metal arches with fiber and plastic covering, dimly revealing the fighting vehicles within. 

	The church was the most beautiful building, a plain wooden structure with large arched clear glass windows and a delicate wooden spire. Its white-painted wood was almost too bright to gaze upon in the bluish sunlight. 

	Down the slope from the village was a large meadow. At one end stood a number of spacecraft, and at the other a few barns and stables.

	The village was four hundred years old, and its inhabitants, still including most of the original settlers, having achieved a way of life that suited them perfectly, had deliberately, almost completely, but not quite totally, forgotten about those of their children who usually, at the age of eighteen, left for livelier spots of the Universe.

	The village was the metropolis of its planet, and the planet the most notable settlement of humanity for a distance of 57 light-years. One hundred seven souls lived in the village, and another five hundred and twelve on Palisade Fjord, where the people preferred what they called a ``neighborly solitude.'' The planet boasted just barely enough people that they could not all know each other, even by sight; and the system, which contained some orbital mining stations as well as a fortified observatory of the Sacred Order of Navigators, supported a settled human population of four hundred thousand, two hundred and five, plus a much smaller, fluctuating number of transients. The population grew, slowly enough by the standards of Beyond, doubling every century.

	The transients included seekers of inhabitable planets, peregrine members of the Order of Navigators, rough trade fleeing justice or seeking easy pickings, which neither the town nor any other place in the system was, and the usual small fleets trekking Beyond for the usual reasons. The settlers, as they still called themselves after their four centuries, were more hospitable than outsiders guessed before arriving.  But, if the outsiders had been more numerous, the settlers might perhaps not have been so friendly.

	Across the Fjord from the village, in a high valley exposed to the cold breeze off the glaciers of the Palisades, stood a two storey building of hewed granite, a square of stone surrounding a herb garden, with four corner towers. This was Surmang Monastery, in which during any given year one fifth of the adult population of the settlers lived, praying, studying, and singing. The settlers took turns leaving their homes and families in the village or across the planet to seclude themselves in the Monastery, or, for a chosen few, in solitude.  Hermitages for one or several men or women were scattered across the face of the planet.

	The customary term of seclusion in monastery or hermitage was seven planetary years, slightly shorter than Earth years; the people would then live 35 years ``outside,'' enough time, as they considered it, to raise a family, engage in business, or even go traveling. The settlers, though not frequent travelers, were expert spacemen by necessity and found easy employment as masters, navigators, armorers, and skilled crew.

	The heart of human civilization in the system was the Monastary and its choir, which had deepened the art of song into an instrument of praise and enlightenment unequaled in human history, and raised the practice of religious song into an art beyond the conception of Earth.

	It should not be thought that the settlers were without passion, without sex, without violence; there was indeed a rough edge to their lives, deriving both from their sheer distance from Earth and the more populous settlements, and from the desperation of existence however armed with technology in the far Beyond, so deep in a cosmos that oscillates unpredictably between utter indifference and active hostility to human life. 

	But, walking from one house to another as the brilliant sun falls into an even twilight, a girl might raise her voice in lilting, almost wordless song in scales of twenty-two or thirteen tones. 

	A man, hunting with his dog, might hum five bars in multiphonics, or imitate the bells of the birds he sought, to draw them, and hide the name of God in every tone of his throat.

	To the house on the brow of the fjord now walked such a man, with gun and dog, although he was not a settler, and no believer.

	He was a visitor, a noted musician, a singer, an instrumentalist, a programmer, indeed a composer. Though not famous among the public, as musicians go, he was widely respected among musicians. His modest reknown eclipsed that of the even the greatest singers among the settlers.

	Bart Hwan had come to a turning point in his life on Earth, and having heard some recordings of the singing at Surmang, he left his children, his friends, and his artistic life behind him to come to Quire. There he found far more than he had bargained for. 

	The music of the settlers destroyed music for him. Four years he had lived in the house, and in that time he had begun many pieces,  in the first years, and finished none, and now he began none. He continued to listen, to the settlers and to the music on the radio from other stars, including the music of alien cultures, and to his old music, but he could make no more music now, for the music of Quire was beyond him. Bart was not unhappy. He did not know what he was. 

	Perhaps the music was working on him from within, unheard even to the listening ear.

	Bart walked the path, humming, came to his house, racked his gun and cleaned his birds, and fed his dog. He got in his car, flew across the fjord, and landed in the parking lot at Surmang Monastery, which he frequently visited although, as noted, he had not ventured to take his turn of seven years there. 

	It was twilight, and the shadows of the pines lenghtened against the stone walls. The building was silent yet, its door, as usual, open. Bart walked up the three granite steps, through the door, turned left, and entered the refectory as dinner was being served.

	The people of Surmang, unusually to the eye of a student of history, or institutions, or religion, lived in community and yet wore no uniform, at least no obvious one. The women had dresses, the men bluejeans and shirts.  One woman wore a plain grey silk kimono. Ten people sat in silence, at a long table. A young man and a young woman served without speaking. A woman stood on a platform at the head of the table, with a book open on a stand before her, head bowed. 

	The refectory had small but clear windows, and in the darkening air the glacier gleamed in them. Browsing animals called outside, and the wind, as always, blew cold and clean through the grass and through the refectory, bringing the scent of grass. 

	Bart took a chair from its peg on the wall and brought it to the table, sat and was served, and ate miso soup, barley bread, and salad with the others. After the people had eaten and the plates were removed, white wine was brought, freshening the mouth, and the people sat, sipping it, in silence. Without warning or signal the woman on the platform lifted her head and opened her mouth and sang. The people stood up one by one and joined their voices to hers, until only Bart Hwan remained in his chair. He closed his eyes.

	She began with the name of Allah, singing low in her throat, plain and simple, repeated over and over. 

	The refectory was a large room, stone, and shaped by scientific study and long experience to glorify the unaccompanied human voice; it was shaped like a curved wedge or a blunt boat of stone, and its walls were a grid of square pits of varying height, to diffuse and tune the sound held within. 

	Voices filled the room as a flock of birds would a small sky. 

	The people in the room, some of them, had been singing in this room for four hundred years. They knew every song of God from every culture of humanity and from many alien civilizations, and they made up new songs on the spot, as they sang. This they had been doing, for centuries, with the peculiar discipline of avoiding any style or distinguishing mark to their art. 

	At times the sound was hesitant, phrases of praise or even of singing in tongues; at other times an ancient hymn would gather shimmering in the cloud of sound until its lines were plain and unbearable, and every word assented to: Amazing Grace, Pange Lingua, songs to Rama, songs to Krishna, songs to the Hidden God of the alien Kzan. 

	The woman at the lecturn stood straight, eyes open but looking far past the room, and sang a raga in nine tones to the God of space and time, the Holy Spirit breathing the desert before time and without time, the vacuum before the universe blinked, the child Jesus playing in a dusty square.  Her voice was not remarkable, a mezzo with a faint burr, but it was loud, and dead on the subtle pitches of the raga, and her rhythm was so exact it brought tears to Bart's eyes.  But beyond anything was the purity of her song, its nakedness and its innocence, its desire and its desire fulfilled. As she sang her fists clenched at her sides, lifting and letting go, lifting and letting go.

	Four hours the settlers stood in their stone boat and filled it with their ocean of sound.

	Finally Bart Hwan stood, surprising himself, and his mouth opened of itself. His voice, more polished and brilliant than theirs, trained for the concert stage, was easily identifiable. For years he had struggled to sing like the settlers, to efface his conservatory training, but that evening he quit caring, so heart-broken was he. His voice was in most ways a finer voice, if not so pure, as theirs.

	He broke, he had given up hoping to sing as they sang, and somehow his voice did not break. He sang more loudly than they could sing, and he sang in the exactest pitch, and his melody was an artless variation of the raga of the woman with the purest voice, like a child playing nonsense games in the dusty square with the child Jesus. 

	Bart did not believe in God at all, but suddenly he grew up and sang the name of God and the name of the lover he had lost and who was the reason why he did not believe in God.  He sang his lost lover's name over and over while the others sang the names of God, supporting his voice and supporting her name, and finally he sang in her very voice, an octave above his own voice, wild and free and beyond him, and the name of God stole his / her voice, entered the voice, wore his / her throat like a burnished coat of sound, a loving wing, a lover's breath. He knew then that God lived, and loved him deathlessly, and loved the woman he loved even more than he did...  far more than he did.

	The young man and woman who had served sang with the others. After Bart sang he sat down, weeping, and the voices lowered, humming, and gathered around him. The woman with the purest voice came down from her platform and embraced Bart. He sobbed and clung to her. She stood up and dragged him out of the room, and out the open door. 

	It was dark, and cold, and he sat on the granite step while she sat with him. For the first time in many days Bart spoke.

	``Thanks.''

	``I was very glad you sang with us tonight. You sing even better than we do.''

	``That's simply not true, and you know it.''

	``Well, there was something different about your singing tonight.''

	``I finally forgot myself. God, it was hard... it seems so easy when you do it.''

	``It's always hard and it always seems easy when we do it.''

	One of the men came out to sit with them. It was Jason Smith, who had founded the settlement, and who did not sing well at all, though the woman with the purest voice was his daughter. He had, improbably, a snifter of cognac with him, which he offered to Bart and to his daughter.

	They sat in the cold wind off the glacier, listening to the browsers bark and almost sing, and the ringing of the night birds, sharing the generous snifter, warming their song-bruised throats.

	Overhead the stars burned, the Sun lost among them.

	``Thank you again, thank all of you. I've found what I came for.''

	``We're glad you came to find it,'' said Jason. ``We can't tell, you know, whether what we're doing is worth anything to anyone else, anyone outside.''

	Bart remembered how had sat up in bed, in his apartment in New York, when the boat of sound came over his radio.

	``What's that? Who are these people?'' he had asked out loud, alone in the room. Cars and voices had sounded from the street below, dance music from a club, and breaking glass.  The woman with the purest voice's voice had sounded from his radio, not breaking his heart, but showing how broken it was, and how it had been broken been for a long time.

	``I'm going to leave pretty soon,'' said Bart.

	``Back to Earth?'' said the woman with the purest voice.

	``Earth is over for me,'' said Bart. ``Farther Beyond, but a bigger settlement. I'm going to compose again. I can't do it here... I love your music, it has saved me, but I'm a different... a different kind...''

	``Come with me,'' said the woman with the purest voice. She took his hand and led him away from Surmang Monastery.

	The two moons were each smaller than Earth's but together they were brighter, two strange eyes roaming around in the face of the night, and they glittered twice from the Fjord and many times from the snows of the mountains across the Fjord. The pines, which were not pines, smelled minty and ammoniacal and alienly fresh.

	The woman, whose name was not spoken when she secluded herself in song, held Bart's hand and drew him along a path.  Bart lost track of the direction and the time, but he did not ask where she was taking him.

	Finally a light shone faintly between the trunks, and they came out into a meadow in the middle of the forest, where there was a small house, with a kerosene lamp burning on a table, visible through screen windows.

	They went up two wooden steps, opened a screen door, stepped within.  The house had only two rooms, of unpainted and faintly resinous planks.  A bed with a red blanket stood against one wall, a small table with two chairs against the other wall, and a sink and hutch across from the door.  The room was completely without decoration except for a few printed books piled on a night stand by the bed.  The light of the kerosene lamp, however, made Bart wonder whether he had ever seen light before.  He could almost smell the light, and he could definitely feel it on the skin of his face and hands, warm and clear.  What had happened during the singing was still alive in him.

	``He's not here,'' she said.  ``We'll sit awhile and wait.''  She went to the sink and poured a tumbler of spring water and handed it to him.  It was cold and clean.  She sat, not on one of the chairs, but on the edge of the bed.  He could feel the warmth of her face shining, the warmth of her body moving beneath its cotton dress.  His senses had never been so acute.  Bart felt no need to speak, or, for that matter, to think or not to think.

	He stood, not sitting, but listening.

	The wind blew over the pines outside, moving from one end of the sky to the other.  It was a marvelous sound, deep and broad and spacious, and wilder and lonelier than anything Bart had ever heard.  Animal sounds moved around within it, ferocious and glad and desirous.  The sound of a brook, too, threaded through the sound of the wind, cold and quiet.

	Finally he sat on the bed next to her and, to his surprise, she leaned against him, embracing him.  He felt as though he stood upon a divide; desire lay on one slope, and friendship on another, and God on both sides.  He put his arm round her waist, surprised by its elasticity; she was much older than he was. She looked directly at him, as a woman does who wants to be kissed.  He kissed her.  He knew, though without thinking about it, that the man they had come to see would be home soon.

	The woman drew back. ``I've often thought of leaving this place,'' she said.

	He was shocked, surprised, she had given no hint of not being completely at home among the settlers or in the monastery, but the glory of the singing was still so strong in him that he felt no need to react to what she had said; he knew, though, that she was counting on that feeling, that glory, to create a space in which he could actually hear what she was saying.  He waited, and listened.  Her eyes, into which he had never before looked, were a pale icy blue, her hair was blonde and thick, coarse even, cut short.  Bart listened, looking into her eyes, not even knowing if he sought anything there, merely looking.  She was happy, but unhappy.

	``I'm four hundred years old and I have never left this planet,'' she said.  ``I love it here, and I will come back some day, but if I do not leave and do something else for a very long time I am going to scream.''

	``What would you do?''

	``How should I know?  I know just enough to know how ignorant I am.  Go to school, I suppose, if there are schools for people like me.  I listen to the radio at night, explore the web...''

	``What about singing?''

	She laughed and leaned against him, then pulled back, smiling broadly.  ``You are a singer.  I merely go to church like a good woman.''

	``You are a far greater singer than I am...'' said Bart.

	``I am sure you are wrong, nobody here has heard anyone sing as you do.  But even if you are right, you are a musician and I am not.''

	A kind of sorrow awoke in Bart:  That she, and probably the other settlers, had no true estimation of the power of their singing, its worth, its excellence.  Maybe it took an outsider to see that.

	The door opened, and the occupant of the house came in.  He was a young man, probably no more than seventy or eighty, wearing a red and black mackintosh and bluejeans.  He nodded at the woman, and then at Bart, and sat in the chair, looking at them.  His reddish hair was cut so short it was almost stubble; he needed a shave; if it were not for the utter clarity of his yellowish eyes, Bart would have taken him for a laborer or even a thug.  

	Bart understood that the man was a hermit, and, possibly unlike the woman with the purest voice, the true article; living all by himself out in the woods decade after decade was exactly what he wanted to do, was what he should and must do.

	``This is my sister, Leah,'' said the young man to Bart.  ``My name is Liam.  I've heard a little bit about you... would you like something to eat?''

	``Do you have anything to drink besides water?''

	``Mint tea,'' said Liam, smiling.  He stood and poured a kettle without being asked further.  ``Leah comes and talks to me, although nobody else does.  She's been complaining about Surmang for decades.  I don't know why she hasn't left.''

	``I love it here, you know that,'' she said.	

	``It's stunting you, torturing you, you should go off and have lovers and raise families and fight in wars and make billions of dollars...'' said Liam, laughing.
	
\newpage
\chapter{Poison}
Phil first saw Shirley sidling along the wall of the Faith Church gym at coffee hour.  She had coarse black hair, black eyes, a dancer's body, and a spooked expression.  She approached Phil before he could approach her, informed him that she was a painter and a dancer, and darted away before he could finish telling her that he was a writer and a composer.

After that Phil saw Shirley at a crowded studio open house in Long Island City, but he did not have a chance to talk to her there.  Finally he invited her to join him with Steve and Joanne Robinson for brunch after church.

It was a warm April.  They sat in the courtyard of a cafe in the East Village and picked seeds with green wings off their food.  Phil and Steve argued about whether miracles are real in the sense of causing scientifically inexplicable but objective events.  Phil, the artist, attacked the objectivity of miracles while Steve, the physicist, defended it.  Shirley became interested and questioned Steve at length about his work.  The conversation then turned to the failure of Communism in Poland.  As if these topics were personal triggers, Shirley digressed into vague but long-winded revelations concerning her being poisoned, her apartment being bugged, and how her life even now was threatened due to her once having been involved with Ric.  He had, she said, been a computer scientist and nuclear physicist secretly working on Star Wars weapons for Israel.

By then Joanne was making faces at Phil when she thought Shirley was not looking.    

Later, on the way to the Whitney Biennial with Phil, who was half wishing he had not already agreed to go, Shirley asked what he thought of her story.  

He said:  ``Either you enjoy telling fairy tales, or you're paranoid --- as in schizophrenic.'' 

She smiled sadly.  ``Perhaps it's best for you to think that.  But I'm glad you're more honest with me than most people have been....''

In fact the exchange did not get in the way of their spending a pleasant day together.  In the Biennial, they both enjoyed the ghostly waterspouts and landscapes of April Gornik.  Shirley invited Phil to a Jewish festival on East Broadway, where she lived.  She was, she said, a true native; she had been born and raised on the Lower East Side only a few blocks from her present apartment, though her parents had recently retired to Florida.

By the time they got off the train and ate at a cheap Chinese cafe the festival was over, and the vendors were packing up their tables and goods.  So they bought ice cream and found a bench in the co-op park.  The trees were all in leaf and the blue of the sky was just beginning to deepen towards purple.

Shirley told Phil how she had become a Christian while studying in England, just from reading the New Testament, without talking to anybody.  Of course, she said, she had not therefore ceased to be a Jew.

They discussed the nature of romantic love.  She asked how it was possible to determine the Lord's will in love.  It was clear to Phil that Shirley still longed for Ric.  Thinking of the many failures, obsessions, and changes of heart in his own history, Phil pointed out that it was quite possible to experience false witnesses to God's will.  She looked down and said, ``Like always expecting a letter or a phone call?...''

	She asked Phil about his own life, and he confided that he too was recovering from an unfortunate relationship.  Shirley commented insightfully on his feelings and problems.  They prayed for each other, and she specifically asked Phil to pray for a new man in her life.  He thought she might well be inviting him to be the one.

\begin{center}\rule[3pt]{2in}{0.5pt}\end{center}

	Later, Phil ran into Joanne Robinson.  She looked up at him with a worried expression.  ``That woman we had brunch with last week, what's her name, Shirley?  Are you going out with her?''

	``No,'' said Phil.  ``Obviously she has some real problems.''

	``Oh, I'm so glad to hear you say that,'' said Joanne.  ``Steve and I were wondering what you were up to.''

	In fact Phil found Shirley quite attractive, but it was true that he did not want to become seriously involved with a woman who had mental problems.  He had done so several times in the past.  Still, he was tempted to find out whether Shirley would welcome a casual affair.  However, since becoming a Christian, Phil had more or less avoided things like that.  In any case, as he kept telling himself, such an affair could only hurt Shirley even more badly than she was already. 

	He fully intended to call her up, ask to see her paintings, go out with her as a friend, and so on, but he kept putting it off --- partly because he was not sure what he would do if she did want to go to bed with him.  He saw her in the crowd after church but did not say more than hello.  She looked lonely and out of place.  He found himself counting on the fingers of one hand the Jewish Christians at Faith Church --- which, despite being one of the oldest buildings in its neighborhood and a landmark of the city, often seemed more crowded with artists from New Jersey and South Carolina yuppies than with native New Yorkers.

	A month later, Phil heard from John, a friend of his who had also gone out once with Shirley, that she was dead.  

	John said that he had called Shirley's number, and a weeping woman on the other end of the line told him Shirley had been killed in a terrible car accident.  Half-seriously, John wondered whether Shirley's stories about spies and danger had not been true:  ``Maybe they killed her because she knew too much.''  

	Phil was quite sure the spies in Shirley's life lurked only in her mind.  He could not help thinking that Shirley must have killed herself in a fit of depression and loneliness, and that her mother could not bring herself to admit this to a stranger on the telephone.  Phil mentioned his theory to John, wondering as he did so if it were wise to open his mouth, or whether Shirley had not after all simply been run down by a taxicab.

	As Phil had feared, his skepticism started an unfortunate rumor.  In a week, people were coming up and informing him that Shirley had committed suicide.

	However, after a few more weeks, nobody talked about her any more at all.
	
\chapter{The Beginning of a Possible Reunion}

	The corridor was twice as tall as it was wide, with a skylike ceiling and walls that leaned forward.  Ornate windows glowed behind balustrades, vines cascaded down the walls, and rivulets ran beneath the vines.  There was occasional birdsong, and alien voices calling homely things, and young aliens.

	Some of the walls bounded rooms and structures of the city, and some of the walls were the hulls of spacecraft docked, sometimes for generation after generation, in the hull of the city.  At the intersections, where transit tubes looped and crossed, and brighter lights played above the fountains, was where the transient ships docked, and tonight one of them was human.

	She was scarred, black-hulled, and heavily armed.  Her robots crawled over her, repairing battle damage, polishing away micrometeorite holes, installing upgraded modules in weapons bays and drive pods.  Her observation blisters showed a bristling array of lenses and antennas.  But for all her weight she was lean and long, and there was only one light in her, on the bridge, up high in her bow.  

	That light winked out.  The lock opened, at corridor level, and a man stepped out, looking around.  He was of that indeterminate age granted most humans in the era of genetic engineering, though the wrinkles at the corners of his eyes, and the weight of his gaze, might have denoted centuries to those who also wore them.  He had black silk trousers, short supple boots, a dark grey tunic, a black belt with a silver buckle.   One of the small pouches on his belt was unmistakably a holster; the others might have contained tools, a lunch, a small computer.  The man's figure and face were not remarkable for anything but excellent posture and alertness.  Perhaps a profound sadness had settled so long and so deeply into his face that he himself was no longer aware of it, and only a complete stranger would see it.  The combined alertness and sadness denoted the man's most profound contradiction.

	There were no other human beings in the corridor.  The aliens were grey, slightly taller than a man, catlike, bipedal, and wore no clothes except for leather straps with jeweled buckles and fittings.  Another species, perhaps not biological but robotic, moved more slowly, on single broad feet with rippling cilia beneath, like silver columns.  The man set off down the hall without hesitation, following a map projected onto his mind's eye by an implanted computer.  At every intersection he took the smaller, darker street.  There was no real poverty in the city, but the beings who dwelled in these dimmer, narrower avenues, with night-flowering vines, cherished their darkness and their privacy.  Water ran in gutters along the side of the corridor.  Occasionally, a faint yowling music could be heard from some inner room.

	The corridor twisted and turned and emptied into an inner courtyard, with a small fountain and a ceiling perhaps thirty meters from the tiles.  Perhaps a third of the circumference was not walls but a screen of closely set pillars, through which could be seen one of the lakes that decorated the inner hull of the city.  The opposing shore of the lake, visibly higher than a natural horizon would have been, showed only a few lights.

	An establishment bearing some slight resemblance to a Greek taverna occupied one front on this plaza, though it had no customers at the sidewalk tables and only a few at the bar, which emitted wisps of aromatic smoke from some complicated apparatus.

	The man halted at the entrance to the plaza and looked about.  For the first time his manner betrayed his emotions.  He began to walk much more swiftly, then halted and muttered a few words to himself in English:  ``Teenaged fuckup.  If not tonight then another night.''  With that he took a bench at the circumference of the plaza, where he sat quietly, not looking at anything in particular.  The fountain and its resident birds made their small various noises.  Finally the man looked at a doorway among the other ways, and beneath the vines and the falling water he could read:  ``P. Melvish'' in Roman letters.

	Again the man spoke to himself:  ``Should have called.  Should call now.''

	For another few minutes he sat quietly, waiting for some inner signal.  The humor of the aliens, which sounded rather like spitting and coughing, drifted out of the taverna.  Their scent, too, a talc and ammoniac scent.

	``Fuck it,'' said the man.  He stood, and his irresolution disappared with a sigh and a squaring of shoulders.  He advanced upon the door, and administered three brisk knocks.

	``Who's there?'' called a woman, in English.

	The man jumped at the sound of her voice, then calmed himself.  ``Ross,'' called the man.

	The door opened, and its tenant appeared, leaning forward, mouth wide open, eyes round at first, then narrowing.  She shut her mouth, then opened it to speak.  ``Ross?  Ross?''  She stepped back.  Ross quivered involuntarily as she did so.  ``My God, Ross!  I thought I'd never see you again in my life!  What are you doing in Athekos?''

	``I was passing by,'' said Ross.  ``I looked in the directory and there was your name.  I had to see you.  Sorry if I surprised you by just dropping in on you.  Can't happen very often.''

	The woman retreated further into her doorway.  She was almost as tall as he, with a slim electric frame, wearing an alpaca sweater over levis.  Her eyes were green and her hair was black as space.  She stared at him, while her rapidly changing feelings pulled at the corners of her mouth.  ``God, Ross!  Well, come on in.  Have you had anything to eat?''

	The man, who had hovered at the entrance, now permitted himself to enter, somewhat warily.  The sadness that a total stranger might have remarked upon his face had completenely vanished, and although his face was not disturbed by any extreme expression, it managed somehow to convey a positively beatific state of ecstasy.  He was smiling slightly.  He looked around her apartment.  ``Thanks for asking me in, Penny.''

	The room was high and yet cavelike, its stone walls pierced by niches with wooden balustrades; the dark wood was heavily carved in feline figures and waxed to a fine sheen.  The furniture was alien except for the command chair off a human bridge and a table whose legs had been shortened to human height; it was piled with printout, books, scrolls, and small artifacts.  Books and other records filled a honeycomb of niches in the stone wall.  ``I see you're busy as always.''

	``It's why I came here,'' she said.  ``My work.''  They both reflected on this, which was the reason they had separated, a very long time before.  Penny whirled and clapped her hands.  ``Dinner!''  The sounds of cooking began to emanate from a kitchen in the back.  ``What do you want?''

	``I want what you want,'' said the man.  ``Make a double portion.''  She smiled at his words, and gestured at a feline couch, which was suitable enough for a man who didn't mind his feet dangling.  He took a seat, cautiously, keeping his eyes on her face; for her part, she did not glance away from his gaze.

	``Athekos,'' said the man.  ``How many humans live here?''

	``There were hundreds of thousands at one time,'' she said.  ``That was centuries ago.  I'm literally the only one, now.  I've been here about forty years.  Occasionally, a locator stops here on the way out, or back.  The city is a relic, really.  It used to have trillions of beings.  Now there are only a few billion.''

	``What are you writing?''

	``Philosophy, or theology, or something.  I like it here because this is the oldest city in our universe, as far as I know.''

	``Are you happy?''

	``In my work I am extremely happy.  The libraries here are intact from the beginning, they have files that were originally created before our universe.  I have to learn the languages.  You can't rely on translation programs for this kind of work.''

	``What about other scholars?''

	``I prefer not to meet them in person.  There's always the Net.  I don't even show a mask, though.  You have no idea how profoundly being away from humanity has benefited my work.''

	``Have you gotten close to these catlike people?''

	``I do have friends, and the Frila have their own scholars, who talk with me.  There are other peoples in Athekos, plus many who pass through, and a few other scholars who have come here because of the libraries.  But I'll go for weeks without speaking to anyone.''

	``You can get it all on the Net.''

	``I used to think that too.  But you can't.  Even if you can get the texts, and believe me there really are texts that were never logged onto the Net, it's different to be in the presence of the real thing.  There's an indefinable atmosphere.  Think of what has happened here, the eons of history, races rising and falling.  The city is virtually dead and yet it has been nearly this dead before.  Perhaps it will rise again, a million years from now.  What is your work?''

	``I don't do much of anything.  The truth is, I'm no genius like you.  I've been working as a locator.  That's lonely work too.  Sometimes I write a little music.  I wonder if anyone browsing the Net ever listens to my music, but I do it anyway.''

	``You must be pretty well off by now if you've been locating for a while.  You've survived.  I always knew you were good at it.  You could go back to the Expansion, even back to Earth... even back to New York!''

	They both laughed at the thought of New York, New York of the rattling trains.  ``I'd like to read your work,'' said Ross.  ``But I don't suppose you'll let me.''

	Penny smiled.  ``In fact I seem to have finally finished something.  Actually, I would love for you to read it, and tell me what you think.''

	``I have no scholarly training....''

	``I'm writing this for everyone.''

	``You think it's good.''

	``I know it's damned good or I've wasted hundreds of years.''

	``And what do you call it?''

	``Don't laugh... Theologies of Nothingness:  Non-Existence as the Semantic Pole for the Categories of Existence.''  She laughed.  In her laughter, which Ross leaned forward to see, a younger woman's face appeared, showing how old in fact Penny's usual face had become.  

	``God, it's good to see you, Penny,'' he said.

	``It's good to see you too.  I really thought I would never ever see you again.  The universes are so vast.  So much has happened.''

	``But people live a long time, and travel fast,'' he said.

	``That's true.''

	Her robots set the table, and they ate, drank red wine, talked.  After the main course Penny set down her fork and said, ``I have a confession to make to you.''

	Ross set down his wineglass, and looked down, for her words made him afraid.  ``Go ahead.''

	``I betrayed you.''

	Ross was silent.

	``I knew you really loved me, Ross.  I knew how rare that is, too.''

	Ross was silent.  He did not look up.

	``I was afraid of you.  I desired wisdom, and I was afraid your love would bind me in an unwise state.  So I betrayed you by making you leave me, even though I knew I could love you.  But it was too hard for me.''

	These were words that Ross had often dreamed of Penny saying, and now her voice, only slightly wispy with her great age, was actually saying them to him.  He bowed his head.

	``You were unwise to love me, or anyone, like that,'' she continued.  ``Most people stay in love for four or five years, if they ever really love at all, though I suppose everyone hopes to.  And when someone does stay fixed on another person for so long, there's usually something... obsessive... about it.''

	``I was obsessive enough,'' said Ross.  Other people might have made excuses or not talked so deeply, but these two had once been closer than brother and sister.  ``Perhaps I still am.  God knows that a day doesn't go by without my thinking about you.  I don't know how I do it myself.  But you know, I have loved other women.''

	``That's good,'' said Penny reflectively.  ``Well, let me go on with my confession.  Obviously there's something wounded in me, that seeks solitude.  I don't know how I ever let myself get so close to you in the first place.  You know how often I've preferred the love of women, anyway.''

	Ross was silent.

	``At any rate, one of the reasons I came to Athekos was to find out what I really thought about a lot of things.  About my parents, and the War, and the Expansion, and so many aliens, and human nature, and the vast emptiness of the universes in spite of so much history, and wars everywhere from before the beginning of time.  I may be solitary, you know, but I feel all of these things as if they were happening in my own body, in my own heart.''

	``You have a great gift,'' said Ross.

	``And so of course I found out what I feel about you.  I do love you, Ross, very dearly.  Wait.''

	Ross could not help himself, and had begun to weep, trying to keep it quiet.  She did not comfort him, but continued speaking.

	``I've done the thing I always longed to do, I have written a great book, and I really believe I have answered, or at least partly answered, questions that people have been asking ever since they became people.  They came first for me, Ross.  My ideas are my babies, and I protected them with my life, and with your life.  And of course it was so hard for me to be with anyone, even women like myself.  There's no such thing as telepathy, but I felt that there was, it was as if everyone's thoughts, which I could detect in the sound of their breathing or the color of their cheeks, were invading me with their pain and their demands, which I could not help but feel as my own.''

	She paused only for breath, but leaned forward and went on, her face now slightly rosy with wine and emotion.  ``I didn't think I could do it, Ross, marry you or be with you permanently I mean, I felt like an emotional cripple, like a selfish cold bitch.  I had to protect you as well as myself.  Wait.''

	Ross forced himself to take a drink of wine, and to smile at her.  Not once but several times he had opened himself completely to this woman, who had always, eventually, rejected him.  Never had he opened himself so wide as when he knocked on her door that evening, however.  Not to do it would have been a kind of death, and doing it had been another kind of death.  Ross profoundly knew that he was not of the same quality as Penny; she was of the order of Aristotle or Godel, and he was merely able, well-read, quick-witted, and the like.  ``Go on,'' he said.

	``I don't know what I'm saying.  God, don't think I haven't thought about this conversation many times, even though when I was thinking about it, I was thinking I would never see you again, that you were dead somewhere or married or forgotten me...''

	Ross shook his head slightly.

	``I had no right to think about it,'' she said.  ``But I did, and after many years I realized that even though I'm flawed and can't tolerate intimacy, that is not good.  You have no right to demand it of me, and it is not wrong for me to be by myself, but it would be better if we could be together.  That's all.''

	Ross looked at her.  She had said things like this in the past.  He was beyond caring and, at least, temporarily, beyond fearing.  She seemed then, as she had at times, weak and vacillating in comparison with his desire and his constancy; he feared that she would see herself as he saw her, and take it as a judgment upon her, though that would only be her own guilt speaking, wearing him like a mask.  He had no idea what to say, so he said cautiously, ``I'm not trying to make you feel guilty, Penny.  I just love you, that's all.''

	``Well, this is one of the things I realized.  Love can't help but make the loved one feel guilty.  Not to return love --- and I mean by doing, not merely by feeling --- is really not so good.  You know the difference between righteousness and goodness, between evil and badness?''

	``Goodness is without will, and righteousness is only by will.''

	``Very good... have you been studying?''

	``I've lived long enough to read a bit, Penny,'' said Ross drily.  They both laughed again.  

	She went on, ``I couldn't will to do something experience had taught me I'd fail at, so I can't be said to be evil, which is another way of saying you have no right to demand that I marry you, or anything like that.  But I know myself pretty well now, and I can say it would good for us to be together.  Wait.''

	``But you're not...''

	``Wait, I said!  I've been essentially by myself for over a hundred years now.  I've had no sex life except occasionally in my own head and by my own hand.  I've finished my book.  Who knows how long I'll live?  I don't take daring chances like you locators in starships.  I just sit at home and read and write, and go for a lot of walks in old cities.  But maybe there's some way around this.  We've gone back and forth so many times...''

	Ross sprang to his feet and turned away from the table.  ``Penny...''

	``I...''

	``You wait now.  You listen to me.  God, do we always have to start fighting the minute we see each other?  Does it really run that deep?''

	``We don't have to fight,'' she said in a small voice.

	``Listen to me.  You know I'm your slave, for all practical purposes.  But that's destructive to me unless you give yourself to me as much as I give myself to you.  It's not fair.  Again, I'm not trying to make you feel guilty, darling, I love you, I'd die for you.  But I have to respect myself.  I'm stuck on you.  I can't help feeling this way.  God knows I've tried.  There is something obsessive about it.  But I'm in control, in a weird way... I'd never hurt or insult you, and I can love other people... it's just that I can't help thinking, how much better it would be if only...''

	Penny stood up and came to him and put her arms around him.  ``Hush, honey...''

	``I didn't think we'd go through all this again in the first two hours,'' he said to the books in their niches, over her dark hair, streaked with grey, which smelled of soap and vanilla.  His hands caressed her back, and reached beneath her sweater to caress her back.  It was like touching God.

	``We're the only two human beings in a city of two billion,'' she said, ``and I've had plenty of time to think about all this.  I know I let you down, and I do love you.  It would be better if we could be together.  Maybe there is some way around the wall in my head.  I wrote about us, you know.''

	``You wrote about us?''

	``Psychology is ultimately of no use in these things.  It's really fundamental theology.  To say that one will not spend eternity with another person is to say that they are in fact not really another person, not a complete person, not trustworthy, not clever, not entertaining, not willing to change for the other, but rather an object of convenience, a slave or servant, one whose will and desire can essentially be set at naught.  All love stories in all cultures, and I don't mean only human cultures, are fixated on death and eternity, which are mirrors for each other and for the fulfilment of desire.  This is why true love if not requited always invites ridicule or makes the beloved feel guilty, why happiness is so difficult, why people are sinners before God whose very existence is love...''

	``But not impossible,'' said Ross, caressing Penny's naked back beneath her sweater.  He lifted it over her head, freeing her breasts.  She unbuttoned first his tunic, then his trousers.

	The dim stone room was swimming in light, and every time he touched her slight body, he was touching the sweet familiar flesh of God.

	Penny stepped out of her Levis, completely naked and only slightly less fit than she had been in his memories, which were frequently dreams.   Grasping his penis by the root she bent forward and took it into her mouth.  Ross closed his eyes so that he could better feel her tongue caressing him, and thrust gently into her.  But she released him, and took his hand,and led him to the couch.  She bent over it and helped him enter her from behind.  By bending her head to one side she was able to kiss him while he fucked her.

	``I do love you, Ross,'' she said.  ``You feel so nice in me.  Do you know I never wanted to have a baby in my life, but now I do?  For years I've been thinking about it, all alone out here.  I was wondering when I would finally leave Athekos and go find some man to do the job, but it's much better if you do it, at last.  Will you give us a baby, darling?  Make me pregnant.''

	He answered her in the only possible way.

	They lay for hours, not quite sleeping but hardly speaking, in the serenity that follows sex with a loved one.  Her warm slight form, again, resting against his side, breathing softly into his face, flinging an arm restlessly around his chest, was the very mind of God touching everything broken in him.

	In the years to come she might again become restless, and if she rejected him again, he might very well die, or be so wounded that he could never love anyone else no matter how long he lived; if he had not come to Athekos and knocked on her door, he might very well have died, or become so angry and dry that he would never love anyone no matter how long he lived.

	Equally, in the years to come if she rejected him, or if he despaired of her and sought another for that ultimate bond, she might dry up and become without feeling on that basic level, a woman of intellect and perhaps of affairs but never of passion, until she became obsessive about her personal habits and indifferent to others; or she might bend against herself and finally love, but only those who in their turn would reject her, until she herself despaired.

	From the taverna in the plaza sounded coughing and snatches of yowling music, and aromatic smoke drifted through the tapping vines at the open casement.  Outside the decayed city, in the cold black sky of space, starships older than Earth maneuvered for advantage in war.  Ross and Penny lay in each other's arms on a couch that smelled of alien fur, dreaming of their younger selves, and of their parents' sadnesses, and the mountains of Earth where they were born.  

	The book read in part:

	All cultures whose individuals grasp the oncoming of their own deaths must formulate the concept of nonexistence, to be sharply distinguished from the concept of nonbeing.  Nonbeing is not a fault, but existence is life as known, conscious life, and so to grasp the oncoming of nonexistence is to be afflicted with the despair radiating backwards from all failed possibilities.  Nonexistence is by definition not a positive quality or predicate, and therefore, in philosophical logic and existential thought, it bears the role of zero or the empty class, or of the identity operation.  All cultures, therefore, however much they differ in their fundamental categories and concepts, share in this most basic and undefinable one.  Through its use, as I shall explain, apparently contradictory and disparate systems of thought, feeling, judgment, and general culture can be placed into relationship so that their fundamental categories can be made mutually intelligible.

	I mean specifically defeat, despair, nihility, Hell, indifference, not mere lack or absence but the positive and yet qualityless frustration or defeat of existence.  In this category there is a subtle union of the normally distinct concepts of being and existence. That, of course, is what enables the conjugation of categories --- the veiled union of the positive concepts of being. Nonexistence, being empty, cannot perform this. But it can bring positive concepts into alignment.

	The culture that built Athekos, which descended into our universe from a prior and larger one, had among other religions a fundamental religion, in which God is not a category, but nonexistence is negated by its own universality.  This recognition, or wisdom, is metaphysics proper and the absolute movement of the spiritual intuition.  Insofar as concepts of God resist the solvent of philosophical logic, it is in this instability of universal negation that they rest.  There is a strict kinship, which I will develop in a later chapter, between the self-referential inconsistency of universal nonexistence and traditional and neo-traditional proofs of the existence of God by the necessity that perfection exist.  However, the argument from nonexistence is not modal, but appeals directly to indubitable experience.

	Human love perfectly expresses the meta-cultural dialectic that I am trying to illuminate.  In love, nonexistence is experienced as betrayal, failure to love or to be loved, and indifference; existence is experienced as participation in eternity, fidelity, victory, and even directly in the nature of God.  An asexual alien race will be able to enter into the human experience of love to the degree that they transpose their experience of nonexistence into the human biological basis of human existence.  This transposition is false and misleading couched in any positive terms, which by their nature offer a false analogy, but it is true and substantial if couched in the one term of nonexistence, which the pole of all existential dialectic of whatever qualities and categories.
	
	This is the inner meaning of Keats' concept of ``negative capability,'' namely, identification with the other through the common point of nonexistence.
	
\chapter{Jacob's Web}

	The old man pushed away the swinging arm that held his computer and lay back in the pillows, breathing with an involuntary rattle.

	Across the room, a bee crawled from the mouth of a pink snapdragon and flew into the windowpane with a thump, and the old man's grandson lifted his head from the eyepieces of a stereo microscope.  ``Johnny?  You all right?'' he called.

	``I'm all right, Jake,'' said the old man.  But his whisper was so slow and soft that the boy knew the old man was not all right.

	For several minutes the old man gasped for breath, while the boy stood watching carefully, neither advancing nor retreating.  Finally the old man sagged a bit and the sound of his breathing became more normal.

	The boy bent his head back down over the microscope.  In a Petri dish, tiny bits of bright metal, cast in precise shapes, lay jumbled in a shining pile.  The boy touched a switch and a vibrator hummed, agitating the thousands of pieces so that, like molecules in a cell, they tumbled this way and that, sometimes fitting together, sometimes breaking apart.  ``Cool, Johnny!'' called the boy.  ``There's this thing like a gold chain that's wiggling between these jaws --- and two chains come out the other side!  It looks kind of like a zipper!  What's that for?''

	``It's just a working model, Jake,'' whispered the old man.  ``The 'chain' is the control tape for the automaton.  The jaws make an exact, bit-for-bit copy of the tape.  It's the key to self-replication.  Tell me, Jacob, have the two tapes emerged from the copier?''

	``Yes, but they look more like chains or necklaces than tapes,'' said the boy.

	``'Tape' is a mathematical term, not a descriptive one,'' said the old man.  ``And are the tapes caught up by the control and assembly units?''

	``Are those the things that look like... like Mother's metal mesh handbag?  Yeah, each tape's got caught by one of the handbag things.  And now the handbags are spitting out charm bracelets... little lengths of chain with other pieces hanging off.  Wow!  One of the chains is folding up into a handbag!''

	``Very good, Jacob.  You've made me very happy.''

	``Why, Grandpa?  I didn't do anything but tell you what I saw.''

	``I can't see for myself any more.  I can't even move out of this damned bed.  The insides of my bones hurt too much.  And Jacob, not everyone can see accurately and report what they see.''

	There was a time of silence in the room.  The bee buzzed against the windowpane, rattling its wings.  Outside, willows and wisteria overhung a suburban stream, lined with boat docks and white cottages.  Brilliant sunlight streamed through the trees and glanced off the placid water.  An intense scent of greenery and life pervaded the room.  The boy was very glad and very sad at the same time.  He ran to hug his Grandpa.

	``It's real,'' whispered the old man.  ``It's real, Jakie.  I made it real.''

	``You said it was a model,'' said the boy.  ``Are the real ones a lot bigger?''

	The old man tried to laugh.  ``No... they'll be thousands of times smaller!  Ah... uh....''

	``Are you all right?'' cried the boy.

	``I'll... be.. all right,'' whispered the old man, straining up from his pillows.  The boy was frightened and took a few steps toward the door.  ``I'll go get Mother....''

	``No, Jakie!  Stay with me.''  There was a note of command in the whisper.  ``Please!''

	Reluctantly the boy returned to the bedside.

	``What will you be when you grow up, Jacob?'' whispered the old man urgently.

	``I want to be a scientist, like you,'' promptly returned the boy.

	``Don't say 'I want,' say 'I will,''' whispered the old man.  ``Will you?  Of course, I want you to be whatever you want to be, Jacob.  But will you?''

	``I will,'' said the boy.  He paused, considering.  In a low voice he confided, ``I want to discover the texture of space.  I think it's foamy, with bubbles that all reflect each other.  If space was smooth, how could things see where they are?''

	The old man was not really listening.  ``You've made me very happy,`` he whispered.  ''You know that I love you, don't you, Jakie?  But there's one thing... don't let them push you around... you do what you want, what you know is right!``

	''Yes, said the boy, alarmed by the intensity of his grandfather's whisper.  He edged towards the door, certain it was time to call his mother.

	The old man tried to sit up from his pillows.  His breath rattled.  The sunlight shone uncompromisingly on his gaunt, unshaved face, the sallow orbits of his burning, empty eyes.  Then he lay back down and forgot to move or breathe.

	Jacob Foster burst into tears.  He threw himself for a moment on the corpse, then ran from the room.

	The bee buzzed against the windowpane, weakening in the sunlight.

	The screen of the computer overhanging the bed continued to glow:  ``'Theory of Self-Reproducing Automata, Part II:  Optimal Nanotechnological Configurations in Superconducting Fullerenes,' by Jonathan Foster and R. A. Widding, October 12, 1991.''

	Several minutes passed.  The door opened and a tall woman in her mid-forties entered the room.  She went to the lab bench that had been set up along the wall and switched off the vibrator and the lamp in the microscope stage.  She knelt by the bed and lay her head on the old man's chest, listening carefully to the silence.

	``Good night, Father,'' she said, and pushed down his eyelids.  She stood and left the room.  ``Jacob,'' she called.  ``Jacob, we're leaving now.  Please get your things together and get in the car.''

	The boy stood for a minute in the sunlight by himself, watching the dead man.  ``How will I know what's right, Grandpa?'' he whispered.

\begin{center}\rule[3pt]{2in}{0.5pt}\end{center}

	The cabin was two meters by two meters by two meters.  Sunlight, weaker than from Earth but brilliant in contrast with the clouded limb of Saturn and the utterly black sky, swept every 30 seconds through the porthole.  The sweeping light picked out floating socks, food bar wrappers, empty soft drinks, crumpled printouts, motes of dust, and a sleeping bag curled in a net hammock.  
	The bag wriggled as its occupant manipulated invisible objects.  He mumbled to himself.  He appeared to be dreaming.  Behind his dark glasses, however, Jacob Foster was wide awake and hard at work.

	The virtual conference was set in a sunlit garden, with birds singing from espaliered trees and an ancient fountain purling down steps of mossy stone.  Of course there was no scent of water, no breeze, no warmth in the sunlight.  Then too overwork, irregular meals, lack of sleep, and free fall left Jacob feeling very much as if he were dreaming.  He pointed to an unsupported, unframed map of the space between the event horizons of his gravitronic gate. ``The violet lines show the curvature of space, and the yellow lines show the curvature of time.''

	Two men and two women watched.  Maria Giuliani and Wing-Tsit Cheng said nothing and scarcely moved:  they were mere icons denoting the fact that their actual persons, on Earth, would receive a broadcast of this meeting.

	Andrew Cadigan, Project Commander, Yasuo Hashimoto, Project Engineer, and Dodie Schultz, Project Technician, were virtually present in real time.  ``This is from the last probe to Tau Ceti,'' said Yasuo.

	``That's right,'' Jacob touched a button at the bottom of the image and green ripples appeared.  ``These are the anomalously regular metrical features I was talking about.  This surface here is the metric for our entire universe.  Of course, the perspective is really squashed.  Almost all our local space-time is in these narrow bands at the top and bottom.''

	``The ripples go all the way from the top to the bottom,'' said Yasuo.  ``I thought the overall metric of space-time was supposed to be extremely smooth.''

	``Well, it's foamy or grainy on a quantum scale, but on the macroscope scale it is extremely smooth.  On the cosmic scale we're looking at in a filtered view here these ripples represent a very, very small amount of energy --- less than the mass of a star, maybe much less.  What's important is how regular the ripples are.  I don't have the faintest idea what caused them.''

	``It seems that the crests are not quite parallel,'' said Yasuo.  ``I don't think the origin of their angle is on the torus.'' 

	``That's right,'' said Jacob.  ``Whatever caused these ripples was decelerating relative to our metric as a whole.''

	``The ripples are old,'' said Dodie.

	``I think the angle of ripple means that the source is older than the metric,'' said Jacob.

	``Older than our universe.''

	``Right.''

	There was silence.

	Andrew said, ``Well, whatever it was happened a long time ago, isn't that so?  Even if we don't know what caused this, it's not anything that's happening now.''

	``No,'' said Jacob.  ``There's no reason to change our plans for the manned test jump.  These ripples are far too old and small to affect our course.  I just wish I understood them, that's all.''

	``Maybe we'll understand them when we make our jump,'' said Dodie.

	``To find the origin of these features we'd have to jump outside our entire metric,'' said Jacob.  ``That's not exactly on the test agenda.''

	``Then we're still on schedule,'' said Yasuo.  ``I think we should all try to get some sleep.  This is a stressful time.''

	Andrew frowned and looked up into the blue sky, seeing something the others could not.  ``Uh-oh...''

	``What is it?'' said Dodie.

	``I'm not sure you want to see this....''

	``Of course we want to see it,'' said Jacob, with exasperation.

	``I'm receiving a Mayday call,'' said Andrew.

	``A Mayday call?  In Saturn orbit?'' said Yasuo.  ``There are no other ships out here.''

	``That's what I thought, too,'' said Andrew.  ``There's nothing we can do about it, but there's no ignoring it.  I'm changing our view to the telescope monitor.''

	The garden vanished, leaving the images of the conferees floating in the void, with the Sun hidden behind a black circle and the stars shining all round.  The Galaxy girdled them.  Saturn's moons spangled the heavens.  Despite his exhaustion and the anxiety of a new emergency, Jacob felt a moment of exhilaration.  He was so wrapped up in his work on the gate that he frequently forgot he was in space.

	``Zooming in,'' said Andrew.  Moons rolled away, the rings expanded.  It was dizzying.  Against the great yellow belly of Saturn was silhouetted a swarm of black dots.  At first Jacob thought they were ice chunks, or meteorites, but then he saw that some of the dots were ovoid, like grains of rice, blinking red and green at the ends.

	``Zooming in again.''

	The swarm expanded to fill their view.  They were visibly spacecraft, but there were chunks of ice and stone as well, a gravitational knot of floating junk.  The ships were feeding among it.  A tracery of Saturnian cloud formed the backdrop.  Lightning flickered beneath the clouds.  

	Yasuo and Jacob comprehended the situation at the same instant.  They looked at each other with sick expressions.  ``They're way too close to the planet.''

\begin{center}\rule[3pt]{2in}{0.5pt}\end{center}

	Richard was awakened by knocking at the door to his alcove.  It was Jacob, in a spacesuit liner and several days' beard, with two bulbs of coffee.  ``You wanted another interview before the jump.  I have half an hour before the final setup on the gate.''

	``Do come in,'' said Richard groggily, accepting the bulb that Jacob handed him.  The physicist curled up in the topmost corner of the alcove and shut the curtain behind him.  Richard sucked the warm bulb and felt himself coming awake.  In a few hours he would, with these others, pull open a hole in space and step through it into a new world.

	For a few moments the two men regarded each other.  Sharing ship's quarters during the two months out from Earth had brought them to know each other intimately, in a way, and even to like and respect each other; but they could never be friends, for their aspirations and judgments were mutually opaque.

	``We've already talked about everything,'' said Richard at last, ``yet there are still many things about you I just don't get.''

	``Such as?  Fire away.  I have nothing to hide.''

	``I know you don't, but I can never seem to grasp your motives.  You're far from ignorant of history and politics --- in fact, much though I hate to admit it, obviously you know a good deal more than I do --- yet I've never seen you give a moment's consideration to the political or social consequences of your work.  Surely you're aware that the world is in a crisis.  We're building up to some sort of general collapse, or perhaps there's a world revolution brewing.  Your scientific fame gives you enormous prestige --- the mantle of Einstein and Sakharov is yours for the picking up.  Don't you care what's happening in the world?  Don't you know you could do a great deal to help, just by speaking out?''

	``Of course I care what's happening, but you don't seem to realize two things.  First off, my vocation really is physics, not politics.  This jump we're making today is the culmination of my whole scientific career.  More than that, it will almost certainly prove or disprove quantum gravity, which is the final stage of physics.  This is infinitely more important than a world political revolution --- difficult though it must be for you to believe that.  Secondly, it's clear that we know much less about human nature and society today than we ever did in more traditional times.''

	This was the sort of dictum that Jacob was always letting drop with the most irritating possible tone of obviousness.  ``What do you mean?'' countered Richard.  ``We have even decoded the genome.  Obviously we know much more about human nature today than in any past age.''

	``Not at all.  You're some sort of neo-Marxist, so I suppose you grant the technological and economic foundations of society some degree of causation in political affairs.''

	``I don't believe, as the classical Marxists did, that the basis determines the superstructure,'' Richard said cautiously, ``but yes, the basis and the superstructure mutually condition each other, and the basis does energize the superstructure.''

	``Whatever.  I'm granting as much of your point of view as I think valid, to give us some common ground for discussion.  So you would accept that scientific discovery creates new technological potentialities?''

	``Yes...'' said Richard with a helpless feeling.  It would almost be preferable for Jacob not to respect his intellect.

	``And these technological potentialities, in turn, offer new social potentialities.  Cheap oil made automobiles possible, and mass ownership of automobiles made suburbs possible.  Then when the price of oil went back up, the suburbs died, or rather tried to turn back into cities.''

	``Surely.''

	``But the new technological potentialities can hardly be predicted in advance of the new scientific discoveries, can they?''

	``I suppose not,'' Richard said reluctantly, sensing the blow about to descend.

	``And the scale of scientific discovery has increased enormously, hasn't it?  Isn't quantum gravity the most enormous theory ever, with the most revolutionary possible technological potentialities?  I mean, just take military science.  The gravitronic gate is more than a means of faster-than-light travel, it's an irresistible weapon.  You can sneak up on any target without warning and crush it right out of existence, and there simply is no defense.  I'm only offering this as an example of a new technological possibility.  And it is the motivation for our funding.''

	``This is one of the main things I've been worrying about!'' began Richard.  ``You, of all people, out to be concerned about abuse of this enormous power ---''

	``Let's not get off the point.  There was no predicting this in advance of the theory, and even now that we have the theory there's no predicting how its technological potentialities will be realized.  I suppose a single tyranny may gain sway over the entire Earth, but then it seems equally likely that people will simply bolt the Earth and spread out into the universe like frightened lemmings, leading to a state of complete anarchy.  That's already happening, actually, on a local scale at any rate.  Think about those poor people who died last week --- there must be other squatter ships in Saturn space who did not answer that Mayday, who were luckier or wiser, who are still out there.  Can you say what will come?''

	``No.''

	``Well then, in traditional societies people had a much firmer grasp of human potentialities and of the future itself, didn't they?  The sons would live pretty much as their fathers did, and everyone knew it.  In that sense they knew much more about human nature than we do, with all our science.  So, one reason I don't bother myself with the political implications of my work is, that I simply don't know what those implications are yet.  And in the meantime, I have serious work to do --- work that perhaps only I can do.''

	``So you're just going to amuse yourself in your ivory tower while the world burns down around you?... wait a minute!  You're refuting yourself.  If people in the past didn't know what kind of mess we, their descendents, would get into, then they knew even less about human nature and potentialities than we do after all!''

	Jacob smiled gently.  ``Exactly right, but that hardly affects my answer to your original question, does it:  Why I'm not politically active.  Let me ask you a question in return.  What do you get out of writing articles and taking actions whose meaning you can't in the least determine, because it depends upon a future economic context you cannot know?  The house is burning down:  Do you really think it's more honorable to run in circles screaming instead of standing still and trying to think? --- to use your metaphor of burning.''

	``Do you really think it's more honorable to do nothing?  All right, I see you're not doing nothing.  I suppose that from your point of view, you're at least keeping a level head, keeping the scientific enterprise alive, keeping the universities going, that sort of thing.  For such a radically original intellect, you're a fucking conservative.  Let's change the subject.''

	``Be my guest.  It's your interview.''

	``Right,'' Richard muttered.  ``All right.  What will be the practical benefit of the gravitronic gate?  What are its uses?''

	``We've touched on the military uses.  Irresistible offense, no more defense.  Atomic weapons were bad enough, but despite its weakness gravity is the ultimate force, so control of gravity is the ultimate weapon.  Then of course, interstellar travel... do you know that the gate permits time travel as well?''

	``So you've said, though I don't see how that's possible.  I mean, couldn't I go back in time and kill myself?''

	``If you did and had you would never have existed, so there's no paradox.  We just need to add three perfect subjunctive tenses to the English language... at any rate, the gate offers practical, even cheap travel to any point in space or time.  You have to be out in space --- you don't want to be coming through a gate at nearly the speed of light and run into the side of a mountain, or a cloud, or anything else that would vaporize you --- but essentially, you can go anywhere in the cosmos you like.  If there are Earthlike planets elsewhere in the universe we will be able to find them and settle them.  It's a permanent solution to the population problem.''

	``If those planets aren't already occupied.''

	``That's a good point, but if they were occupied I presume we'd already have heard from the occupants... though this is getting pretty speculative.''

	``Yes, this all seems so Utopian, so far in the future, but all right.  I have to admit it, theoretical physics has solved the economic problem.  In principle.  In the long run.''

	``Absolute mastery of space, time, and energy,'' said Jacob thoughtfully.  ``I'm not sure I grasp what that all means, myself.  There won't be anything we can't do.  We'll be like gods.  Frankly, it scares me, since, as I said, we know ourselves so little.  But you could ask me a better question.''

	``What's that?''

	``What does quantum gravity mean to me personally?  What's my reward for this work?''

	``Fame, scientific glory?''

	``I've had that most of my life.  I'm not addicted to it.  You don't seem to realize that I have nothing left to do after this.  If the gate experiment works --- and I'm 99 percent sure it will work --- that's the end of theoretical physics.  We have the true theory of everything now.  End of story.  But I'm not yet an old man.  I have decades of productive life ahead of me.  What will I do next?  I'm not interested in merely burnishing my shield.''

	``What indeed?'' asked Richard, surprised.  ``What about the philosophical implications of the theory?''

	``I have the theory all right, but I don't in the least understand what it means,'' said Jacob, with a sort of perverse glee.  ``So I'm hardly able to draw out any philosophical implications.  I have no idea what I'll do next!  Perhaps I'll... go into politics!''

	The watch bell rang.  Captain Hansen's voice came over the intercom.  ``All right, everyone, stations please, to the bridge.  This is the big one.''

	Jacob clapped his hand to Richard's shoulder.  ``Here we go.  It should be just like the last jump:  completely uneventful.  Just a change of constellations.  The experiment itself should be equally exciting --- a photometric comparison of scattering angles.  Loads of excitement for a TV reporter.  Let's go.''  He uncurled himself, twitched aside the curtain, and swarmed up the pole that ran through the middle of the corridor.

	``You still didn't answer the question... what do you get out of it?'' muttered Richard, unzipping his sleeping bag, rubbing his face, reluctantly deciding he'd better do a quick shave before going on camera in front of the whole human race.
		
\begin{center}\rule[3pt]{2in}{0.5pt}\end{center}

	Even though it was the off-watch before the second jump, not the first, I still could not sleep.  For an hour or so I lay in my bag like a worm in a cocoon, eyes wide behind my cyberspace shades, going over and over our scans of the metrical regularities.  I was still unable to understand what had caused these gravitational waves that we had discovered between the event horizons of the gravitronic gate. The waves were so long that they must have been formed at nearly the same time as our universe itself; their angle indicated an origin near or even beyond the horizon of observability, which was impossible.  The waves did not contain a great deal of energy, but there was no explanation for them.

	Finally I tore off my shades, wriggled out of my bag, twitched aside the curtain of my alcove, and went swimming down the corridor, which was darkened for sleep.  We all slept on the same off-watch.  I stopped at Dodie Schultz's alcove and paused, considering.

	``I can hear you breathing out there, whoever you are.  Come on in,'' she whispered.

	I was not sure this was a recommendation, but I was in no mood to quibble.  I slipped through the curtain and found myself in the dark.

	``Hello, Dodie,'' I said.

	``It's you, Jacob.''

	Dodie was the only woman among five men; so far as I know she had formed no liaison with any of us.  It had been easier that way, but now I was beyond caring.

	``It's me all right.  I'll go back if you ---''

	``Christ no.  Come on, get in.''  There was a click as she turned on her reading light.  She unzipped her sleeping bag and waved me in.  I shucked my coverall, wiggled in, and hugged her.  Her warmth and feminine scent were overwhelming, and she laughed as she felt my immediate response.

	Dodie Schultz was not only the ship's Astronomy Officer, but a real astronomer, too.  She was considerably younger than I, but during the months we'd spent confined together aboard the \emph{Gagarin}, we'd become not only scientific colleagues but  conversational partners.  She was reasonably attractive, but until fifteen minutes ago I had never consciously considered her as a potential lover.  I had been too busy theorizing, building, and testing the gravitronic gate; that task had been the focus of my existence for twenty years.

	Perhaps, now that the gate had finally been proved to work, my unconscious was finally demanding its long-denied rights.  We made love without foreplay and without worrying about the noise.

	``Pipe down in there!'' yelled Captain Hansen from his hammock on the bridge.

	``Christ, how can anyone get any sleep around here?'' complained Yasuo from his alcove across the narrow corridor.

	We couldn't have cared less.  I discovered that I felt not only sleeplessness, doubt about whether the gate would really work a second time, lust, fear of death, eager curiosity to see what was on the other side of the universe, and the like, but considerable affection for Dodie.  When we were finished I fetched a bulb of cold wine from the cooler and we lay together talking.

	``Who's that?'' I asked, indicating a Polaroid of a black-bearded young man in a parka.

	``My brother Stevie.''

	``What does he do?''

	Dodie smiled sadly.  ``He's a musician, or so he says.  Last I heard from him he was crashing with a self-named 'band' in some squat in Manhattan.''

	``And this?''  An older couple, grey-haired, the man in tweeds, the woman in a dress and brooch.  ``Parents?''

	``My father and stepmother.  He was a Jew and an engineer and a New Society organizer.  She was an Arab from Beirut.  They died in the London riots.''

	``You must get loads of shit from all sides for being a noncomformist.''

	``Well, don't you?''

	``Since I lack the time, I am neither a conformist nor a nonconformist,'' I said.  ``Besides, everyone's in awe of me.  Aren't you?''

	``Not any more,'' she said, stretching against me.  ``What about your parents?''

	``My grandfather was Joseph Foster, who developed self-reproducing nanotechnology,'' I said.

	``I had forgotten.  Science runs in your family, obviously.  Not in mine.''

	I went on, ``My father and mother taught at Berkeley, biochemistry and history.  They're still there, actually, both Emeritus.  I had an uneventful, thank God, and therefore an extremely happy childhood.  I have no siblings and no children of my own.  The events of the century have passed me by.  I dwell quite contentedly in what remains of the ivory tower.  Do you have a lover, a boyfriend?''

	``What about a girlfriend?''

	``Do you?''

	``No, I just wanted to see if it would bother you.  I did have a boyfriend, a friend of my brother's, back in New York.  I haven't seen him for a year, obviously.  I suppose he's found someone else.''

	I looked at Dodie's other personal items.  A small Bible, which I wouldn't have expected, some plays and novels, even a collection of poetry.  No science.  It must all be on the computer.  A necklace of jade and agate on a silver chain.  Not much... but much more than I had.  ``Why did you end up in science?''

	``I have the brains for it and there was nowhere else to use them.  Being a bureaucrat would have killed me.''

	``We're all bureaucrats anyway.''

	``Not much longer.  The whole system is falling apart.  First Communism collapsed, now capitalism is going down the tube.  You know that.''

	I couldn't deny it.  In fact, though I made every effort not to be a political person, I knew that my work and, indeed, my whole family's work had contributed mightily to that downfall:  starting with the atom bomb, going on through genetic engineering and self-reproducing automata, and now ending up with a bang in gravitronics.  ``The more technological power society has, the weaker society gets.  How can that be?''

	``Technology is individual power, not social power,'' she said.  ``I'm a rebel myself, so I ought to know.  Where do you think it will end?''

	``I think computers will tie society back together again, someday,'' I said.  ``But everything has to change first.  Hierarchical bureaucracies simply don't work on computers.''  We had talked many times, but never like this --- I had never had a conversation like this with anyone.  Obviously it had something to do with sex.  ``We should get married,'' I said.

	Dodie laughed.  ``You're a funny man.  You hardly know me.  You didn't even know I had a brother or a boyfriend.''

	``You're not saying no.''

	``I'm trying not to let the fact that you're going to be one of the most famous men in history warp my judgment.''

	``I'm glad you're trying, but that's probably not possible.  Your judgment must be permanently warped now.''

	She laughed again.  ``We have a lot of work to do tomorrow.  Let's get some sleep.''

	Instead we made love again.  Finally we did sleep.

\begin{center}\rule[3pt]{2in}{0.5pt}\end{center}

	When I awoke I was tired but relaxed and quite cheerful.  I had not had a woman for years.  I should have done it long before.  However, our task soon occupied my mind again to the exclusion of all else.

	In retrospect it seems odd that we would have initiated interstellar travel without more extensive preparation and forethought, but the whole project was a side effect of gravitronics, which had originally been developed for other, that is military, purposes.  Space and time travel were byproducts of the search for a bigger bang.  It was always thus --- my great-great-grandfather had been asked to build the first working digital computer, the idea for which had been kicking around in several forms for decades, in order to compute the implosion for triggering the first atomic bomb.  However, the gravitron represented the end of this road; there was no bigger bang to be had.  The human race would now have to come up with some other motivation for doing science.  Perhaps, lacking one, governments would suddenly and simply lose interest.

	That would be fine with me.
	
\begin{center}\rule[3pt]{2in}{0.5pt}\end{center}

	The bridge of the \emph{Yuri Gagarin} was a featureless grey cylinder --- taking up the entire width of the ship --- filled with a half-circle of acceleration chairs.  Engineering Officer Yasuo Hashimoto's watch was ending.  He remained in his seat as the other others floated through the hatch in the middle of the floor and strapped themselves in place:  from left to right, Security Officer Andrew Smith, Astronomy Officer Dodie Schultz, Captain Anthony Hansen in the center, Professor Jacob Foster (acting as Gravitronics Officer), and finally reporter Richard Wickshire, with his camera perched upon his shoulder.

	``Officer Hashimoto, I'm taking your watch,'' said Captain Hansen.  ``We'll do this one just like the first run.  As before, I will handle no instruments or controls myself.  That will leave me free to observe, to think, and to give rapid verbal orders.  You will obey them immediately and ask any questions afterward.  Mr. Wickshire, you are not to interrupt me or ask me any questions under any circumstances during this flight, nor are you to speak with Dr. Hashimoto or Professor Foster.  You may ask questions of Astronomy Officer Schultz or Security Officer Smith as long as they are not busy with their duties.  Is that perfectly clear?''

	``Yes sir,'' said Wickshire, hating the unnecessary military formality.

	The Captain looked at the rest of his crew, who rendered more or less grudging acknowledgement of his authority.

	``Very well.  Bring up the external display.''

	The grey walls vanished, and the six chairs appeared to float in naked space.  A ribbon of instrument displays floated before them at railing height.

	Richard began to speak to his camera and, through it, to the live audience back on earth.  This was the moment that would make his career and even, if he played his cards, give him real power.  ``This is Richard Wickshire on the United Nations Spacecraft \emph{Yuri Gagarin}, in orbit around Saturn.  We are beginning humanity's second manned interstellar flight, and the longest voyage ever, out to the edge of the visible universe and back.''

	Captain Hansen said, ``Status on the pre-gate boost.''

	Yasuo said, ``Navigation and drive systems nominal, program tracking for boost at 24:13.''

	Richard said, ``Our ship is in a polar orbit just outside the rings of Saturn.  When we reach the pole of our orbit we will fire our main rocket and achieve escape velocity from Saturn.''  He licked his lips.  He wished he could speak to Professor Foster, who had pulled a cyberspace monitor down over both eyes and was sitting stock still, no doubt concentrating intently on initiating the gravitational gate.  ``Dr. Schultz, 

\begin{center}\rule[3pt]{2in}{0.5pt}\end{center}

	They were through the gate and in traffic.  The sky was very full.  There was a shimmering halo, a disc of shining stuff, surrounding a violet-white point of light; that was the black hole and whatever surrounded it.

	A blue white-diamond, a young hot star (how could there be young stars in this ancient galaxy?) coruscated fiercely not too far from the hole.  At its Trojan points, fore and aft in orbit, paced two white-yellow topazes, lesser suns each still larger than Sol.

	Otherwise they were englobed in stars, mostly white dwarfs, except that some twinkled, tinged with red, green, blue, purple....  And ships, like a cloud of unblinking fireflies, or burning gnats.

	Jacob realized that the twinkling stars were not stars --- they were cities.  Stars do not twinkle in space, only through an atmosphere.  The floating cities were twinkling because of the heavy traffic passing before them.

	``Namu amida butsu,'' said Yasuo over the radio.

	Jacob's mouth was dry.  ``Jesus Christ,'' he seconded.

	There were more cities than there are stars in Earth's night sky.  Somehow Jacob doubted they could see all, or even most, of them.  Only the local ones.

	``Look at the disc around the black hole,'' said Dodie.  ``It's full of what I assume are cities, swimming in the dust.  How can they do that?''

	Jacob swung his suit telescope in front of his eye and zoomed in on the disk.  As Dodie had noted, the black dust was sprinkled with shining, twinkling beads of light.  Some of the cities showed visible disks.  They were bigger than stars.  They must be hollow spheres, perhaps surrounding stars.

	``I'm filming all this,'' said Andrew.  ``I'm recording everything.''

	Jacob scanned in closer to the black hole.  The suit telescope blocked out the intense white-violet light where matter, heated to the point of annihilation, went over the event horizon of the hole.  The disk was gathered into braids of dust, with lanes of clear space between.  In the clear spaces, cities swarmed.  In fact, Jacob saw with wonder, the innermost perimeter of the accretion disk was lined with what appeared to be a solid wall of city; it had gaps or valves in it to permit the ropes of dust to pass through.  These people had built a dam to control the infall of the accretion disk of a Galactic-mass black hole.  

	The scale was enormous, unbelievable; the ring of structure surrounding the hole was nearly a light-week across.  It coruscated, glittered with energy.  How could anything live in that radiation, that gravitational turbulence?  Perhaps it was all machines down there.  Perhaps they had tamed the storm of energy around the hole.

	Jacob's mind, which had served him so well, did not snap; it recorded everything, thinking, drawing conclusions as rapidly as... thought.  What did they want with the hole?  It was a source of energy, a source of raw materials, and a freebie gate.  It was the ultimate oasis, the best port in the desert of space.

	He zoomed in closer.  The edge of the hole, blanked out by the telescope monitor to protect his retinas, was virtually a straight line, a vast moon of utter black.  The glowing space just beyond the event horizon twinkled with ships.  They were in organized columns, some emerging somehow tangent to the hole (Jacob's mind whirled giddily with the manifolds of that metric), going so fast he could see them moving; others, skimming in closer and closer.

	``Look as close as you can to the edge of the hole, Andrew,'' said Jacob.

	``I see them,'' said Andrew and Yasuo at the same moment.

	``How do they survive the radiation?''

	``What now?'' said Jacob.  ``This is completely, utterly beyond anything I expected.  My heartfelt apologies, Andrew.  I truly, truly appreciate your paranoia now.  The whole human race is just a drop in this bucket and this is obviously only one center of this civilization.  Let's quietly sneak back and think for a while, if we can.''

	``Great idea.''

	``I was going to suggest it myself.''

	They pulled themselves back into the ship, opened the gate as quickly and with as little fuss as they could manage, and slipped back to Jupiter orbit.

	The indicator chimed as space pinched off and the gravitronic gate evaporated into time to be no more.

	The four, taking off their suits, checking items off their lists by rote, sighed repeatedly as they released pent-up breath, feeling safer now that there was no path to mark the way back to what they had witnessed.

	``Beam that shit to your boss, Andrew,'' said Jacob, ``with my blessings.  I need a beer.''

	``Beer all around,'' said Captain Hansen, who had not spoken the whole time through the gate.  ``Excellent mission, crew.  Outstanding.  Nobody panicked.  Nobody forgot their job.  All the equipment seemed to work.  My personal respect and congratulations.  I know it seems a small matter after what we have seen, but we successfully completed humanity's second and longest manned interstellar voyage.''

	``Thanks, Cap,'' said Jacob, throwing the arms and legs of his spacesuit helter-skelter across the lock.  Then he was ashamed and bent down to stow the items by the book.

	Dodie and Yasuo were weeping loudly.  Jacob and Andrew found themselves joining arms around their mates.  Jacob was coming down from the calm of shock.  He had never been so frightened in his life.  People's teeth really did chatter.  Unlike the Indians who had met Columbus' ships, Jacob realized, he and his mates possessed just enough science to recognize that they had run into something immeasurably superior to their own civilization --- superior by orders of magnitude they had not even succeeded in imagining.  For a minute Jacob wondered whether the technological superiority they had witnessed necessarily implied cultural superiority, but he shuddred at the thought of comparing humanity's six thousand years of recorded slavery and warfare to whatever had spent, obviously, at least millennia building up that....

	Yet in spite of everything, a small voice at the bottom of his mind was repeating, very calmly, very happily, ``I wanted to do it and I did it.  I really did it.  I wanted to do it and I did it.  I made the gate and we went through it and back.''

	``And how long ago,'' whispered Jacob to that still small part of himself, ``did they make the gate and go through it?''

\begin{center}\rule[3pt]{2in}{0.5pt}\end{center}

	They got drunk --- Captain Hansen drank the most of all, encouraging them --- and were laughing hysterically at old college jokes around the wardroom table when Jacob found himself looking at himself.

	He was not looking in a mirror, for his other self's beer bottle was diagonal to his own, not straight across.  The other self raised the bottle and laughed just as he had laughed.

	Jacob had been frightened before, and then in conviviality he and his mates had encouraged each other to forget their fear.  Now his bowels loosened and dumped.

	His other self began slowly to change, becoming slimmer, paler, until it was an inhumanly beautiful androgyne.  Then it became more human again, until it was the identical twin of Dodie Schultz, shaking fingers through her frizzy black hair and scowling.

	There was dead silence in the wardroom.  It stank of human fear.

	Then the changeling became the beautiful androgyne, human but as if of every race and gender and age, and filled out stoutly to become Captain Hansen's other self, cradling his beer bulb in one crooked arm and laughing loudly.
	After becoming each of the four in turn, the alien became itself and remained itself.  It was the inhumanly beautiful androgyne, in a plain human spacesuit liner, with calm grey eyes and a feminine yet strong face.  ``There's no need to be frightened,'' it said.  ``I'm the Ambassador to Humanity.''  It lifted its empty slender hands.  ``Obviously, I have chosen not to do whatever harm I could have done.''

	Andrew spoke first:  Jacob admired him ever after, for that.  ``How long have you been watching us?''

	``Since you came through your gate to Athekos.  Five of your hours.''

	``In five hours you learned to speak English?''

	``I have been listening to your radio, examining your computer memories, reading your books, and consulting and studying with you and with other human beings.  This is a crime, but it is necessary, for now.  When I feel that I can speak as well as you do, I will no longer obtain information without first asking permission.''

	Jacob found his own voice.  ``Athekos is?''

	``The city you saw, the city folded between the horizons of the black hole.''

	Folded into.  Jesus.  He hadn't even noticed.

	Dodie spoke up, her voice cracking.  ``Why did you follow us?  Who are you?  What are you going to do with us?''

	``I followed you to find out who you are.  I am the Ambassador to Humanity.  I offer you Access to the Web  I do not yet have a proper name --- you will give me my name yourselves.  I was built just for this one job, which is my entire life.  I am neither a machine nor a person; I am an interface.  I am operated by a... committee, the Embassy.  You four, and other human beings besides, are on this committee in the future, you are a part of me in more than appearance.  Yes, we, you and I, have gone back and forth in time many years to find the time to study us so much in five hours.  So I doubt I'm as smart as you are thinking!  I remember what you are thinking now.  But I do speak with the full authority of the Web.  That is why I am called the Ambassador.  That is my title, not my name.  I will keep repeating this.  I have always done what you decide, what you wish, or perhaps it will be better to say that we will reach an agreement.  Perhaps we will agree that I will go away and never return.  Perhaps we will agree that I will stay with you forever.  We have not observed and caused the resolution of this loop yet.''

	Jacob's mouth was wide open; he shut it and leaned back in his chair.  He felt like a walking cliche.

	The alien paused.  It picked up a beer and gazed at the amber bulb, frosted with condensate.  ``I am thirsty... will you give me a drink?''

	``And if we don't?'' said Captain Hansen, glancing at Andrew.

	``Then I will die of thirst,'' said the alien entity, ``and I will be rebuilt by the Embassy, and I will go find some other human beings to talk to about Access to the Web.''

	``What is Access to the Web?'' said Yasuo.

	``Oh, you know,'' said the alien.  ``Trade, tourism, cultural exchange, mutual defense.  Politics.  You get the key to the Web, both Cyberspace and Gates, and when you are in our territory you are in the Judiciary and your serious weapons are quarantined.''

	``We have serious weapons?'' said Andrew wonderingly.

	Andrew was always irritating Jacob, who could see that Andrew could solve equations and give the right answers to problems, but never really seemed to believe the answers if they went beyond his personal experience.  Jacob said sharply, ``Haven't I explained several times how there is no defense against the gravitronic gate?  It's a million times worse than the atomic bomb.''

	``Precisely,'' said the Ambassador.  ``You, this small new ship, could destroy great old Athekos.  Not the Web, never the Web for the Web is infinite, but certainly you with this one ship could use the gate to dump us all into the middle of our own black hole.  We would get slower and slowed and never climbing out.  Except I think I could will stop you first.  And we, I, in this small body, can destroy your whole world.  Even your Sun.  Perhaps I will done it.  Perhaps we will agreed that I did not done so.  This is why I am the Ambassador.  It is been my entire life.  It was and will be very interesting, pathetic even.  I speak with the full authority of the Web, for what it's worth.''

	``Your tenses are confusing,'' said Dodie.

	``We are made new tenses in the English language,'' said the Ambassador.  ``You are travel with me in time.  You have looping three times with me, in the future.  If you think the tenses are bad when only one of us was loop, wait until we were all looping several times.  May I please have a beer?''

	``Of course,'' said Captain Hansen, extending a bulb.  ``Be my guest.''

	Although the Ambassador had claimed not to be a person, Jacob was astonished to find he had developed a clear and distinct impression of the entity's personality:  and liked it very much.

	Perhaps it was a damned good Ambassador.  The might of Athekos flashed across his mind's eye again.  Of course it was.  His feelings were reassured:  his intellect, not.
	
	\begin{center}\rule[3pt]{2in}{0.5pt}\end{center}

	After returning to Earth from Athekos, the Ambassador, Andrew, Yasuo, and Jacob are invited to speak to the United Nations General Assembly in a combination address and press conference.
	
	The context of this meeting includes the fact that the Gateway Project, while not exactly classified, had been little known until Contact.  This is mainly because political forums on Earth had been preoccupied with a host of crises including overpopulation, plagues, depression, and the spread of unlicensed nanotechnology.

	The Ambassador rose to speak.  The days when such occasions were marked by the lightning of dozens of flashbulbs were past, but certainly everyone in the General Assembly hall was aware that almost every human being on Earth was watching or listening.
	
	To Jacob, that attention was a physical weight, a wrestler joining his already angry conscience to force his protesting spirit to the mat.  He told himself it was not his fault, that Contact would have occurred without him; but there was no denying that in fact it was his work on the Gate that led directly to this moment, and his arrogance in jumping to Athekos without seeking permission or even discussion.
	
	Yasuo wondered if anyone among the General Assembly, the various heads of state, and their chosen scholars and advisers comprehended that the Ambassador was neither a person nor a machine, but a customized cyberspace interface, a super-sophisticated video game operated by a committee of beings from dozens of species and civilizations with the help of computer programs smarter than most geniuses.  Yasuo wondered if he comprehended it himself.  The Ambassador resembled an exceptionally intelligent, beautiful, and modest human androgyne.
	
	To Dodie, the strangest thing about the Ambassador was its suit, a dark grey worsted from Savile Row, cut in at the waist to suggest femininity.  The thing sported a red silk tie whose blue and green pattern spelled out words in Athask.  Dodie wondered if anyone else in the hall even knew the pattern was letters in an alphabet older than the human universe.  She wished she could read the text.
	
	To Andrew, this was the worst moment of his life.  He had been afraid, of course, ever since his boss had called up a picture of Jacob and said:  ``This young man is building a gravitational field generator.  It was originally intended to be a weapon but it seems it will also make it possible for us to fly to the stars.  We were wondering if you'd be willing to manage security for the project.''  But Andrew had not, at that time, had the imagination to know what to fear.  Visiting Athekos, and above all spending time with the Ambassador, had done much to rectify that; now Andrew was afraid of everything from the complete dispersion of the human race into unexplored interstellar space to a sort of permanently inescapable pet status in the Web.  Worst of all, Andrew knew that even Jacob could not really imagine what humanity was in for.  The only thing that was certain was that this was the most important moment in human history and he had absolutely no idea what came next or how to respond.  Wonderful.
	
	The appointed moment came.  Secretary General Ingit rose to introduce the Ambassador.  ``I will not spoil this moment with meaningless formalities.  Delegates, Heads of State, ladies and gentlemen, and all of you taking part in or watching this meeting, please join me in welcoming to New York the Ambassador to Humanity from the Web.''  Applause, somewhat longer than the standard interval, but not ecstatic.
	
	``I must confess that I myself do not completely understand the nature of the Web or the choices that humanity now faces.  However, let me reassure you all that Earth is now and will remain perfectly safe and free.  The Web offers no military threat to Earth or to the human race.  As long as we remain in our own space, it claims no political sovereignty over us.  As I understand it, the Ambassador is here for one purpose and one purpose only, namely to offer us Access to the Web.  From the political point of view, this is a treaty between equals and, even more, it is a treaty with individuals.  Let me emphasize this.  Even if the United Nations should agree to have Access with the Web, any individual citizen of a member state may opt out of access.  Equally, even if the United Nations and all member states should reject state Access with the web, any individual human being may opt into access.''
	
	``But do not think this meeting is a mere formality or a glorified press conference.  State Access is a very serious matter which, let me urge, we would do well to debate for several years before coming to a decision.''
	
	``That is all I have to say at this time.  The Ambassador to Humanity will now address us.  Ambassador, please.''
	
	The alien entity --- interface --- ascended the podium and gazed at humanity with calm grey eyes.  Seeing this emblem of Athekos, this extension of Athekos standing in a room in New York City, filled all of those who had been to Athekos was a wild mixture of exhilaration and dread.  Their sojourn had been overwhelming, but as long as it did not get mixed up with life at home, it had remained only an adventure.  Now that adventure was over, and real life would never be the same.
	
	``Thank you, Mr. Secretary General.''
	
	``Ladies and gentlemen of the human race.  It is my honor and privilege to serve as Access Ambassador of the Web to you.  My office and indeed my nature require a certain amount of explanation.  I resemble a human being but, as you may have guessed, I am not a human being at all.  My body is a machine, and I have no mind at all.  I am the outward form and interface of a cyberspace virtual reality, in which over three hundred intelligences from a number of species, civilizations, institutions, and systems are gathered in conference, just as you are gathered here today.  As my embassy prospers, some of your members may choose to experience my members from within, as it were, to meet us on a corporate and individual basis.  However, what I say to you is the result of discussions and decisions which do express the full authority of the Web.  That is why I am properly called an Ambassador.''
	
	``I am aware that humanity has only recently built its first gravitronic gate and that until now very few human beings have concerned themselves with interstellar travel or even the possible existence of extra-terrestrial civilization.  I now have your classic texts on the subject in my memory, and I am afraid that even your most imaginative thinkers have grossly underestimated interstellar civilization.''
	
	``This is a critical point.  Please bear with me as I expound concepts that may be difficult for you.  I do not mean to patronize you --- these concepts are difficult for many of us as well.''
	
	``The first thing to realize is that my civilization, the Web, is, so far as we know, literally infinite in size and history.  It has no beginning in time and no boundary in space.  Its numbers are countless.  In real terms, this means that if I wish to locate the earliest record in the Web, I can always find an earlier record; if I wish to locate the most distant city, I can always find a more distant city.''
	
	``How can this be?  Your own scientists have understood the basic facts of cosmology now for some decades.  This universe in which humanity dwells is only one universe in an infinite sea of universes each of which tunnels into actuality from the quantum potentiality, expands for a time, and either fades away or collapses again.  Each individual universe is vast indeed, but the total cosmos is unthinkably vast.  Each individual universe has a distinct beginning in time, but the cosmos as a whole, the quantum reality, is outside of time and space and has perhaps no beginning and no end.''
	
	``There is no way of numbering the universes, but it is possible to travel between them by means of the gravitronic gate.  Consequently, whenever a civilization achieves quantum gravity, it gains the power to travel instantaneously between any two points in space-time not only in its own universe, but in all the countless universes in the entire cosmos.''
	
	``Civilizations that travel will meet, and when they do they must treat with each other for trade, territorial protection, cultural exchange, and many other things.  The Web is simply the stable form these exchanges have assumed in history.  In other words, the Web is a network of treaties between autonomous civilizations, institutions, and individuals.  Over countless ages, this network and its central institutions have achieved a stable form.''
	
	``In human terms, the Web might seem like an autocratic empire, or it might seem like a democratic commonwealth, or it might seem like a mutual defense federation, or it might seem like a form of laissez-faire capitalism, or it might seem like an academic or even a religious bureaucracy.  It is all and none of these things.''
	
	``But before I explain further the nature and institutions of the Web, and the terms of the Access Agreement, you will need to know more about the fundamental cosmology of civilization.''
	
	``The Web is vast, but the cosmos is vaster yet.  Your scientists underestimated interstellar civilization, but they did ask one of the right questions:  'Where are they?'  If our civilization is so ancient and so vast and has such powerful science and technology, why did we not discover you at your very beginnings?  Why did you, only weeks after opening your first gate, discover us so quickly, long before we could discover you?''
	
	``The answer is that, although our Web has been and is now continually growing in numbers and power, the physical cosmos is growing even faster.  Over infinite time, this means that the Web has become almost infinitely thinly scattered.  In a very real sense, civilization is infinitely large and still growing, but it is also and at the same time and without contradiction infinitely small and still shrinking.  It is only gravitronic travel that gives access to all spatial locations and thus affords hope of locating other civilizations.''
	
	``One final fact must be grasped before I can begin to discuss the objectives and central institutions of the Web.  You would not have built a gravitronic gate if you had not developed quantum gravity, which is a reasonable approximation to what some of your scientists call the 'true theory of everything.'  Even our Web science is not certain that it possesses this true theory of everything, but we are certain that we possess something so close to it that we will probably never be able to find anything better.  Our physical theories, like our central institutions, have been stable for as long as we care to pursue the records.  And your own quantum gravity is very close to our physical theory.  What this means is that there are, culturally, what we call 'final stable forms' or 'irreducible forms.'''
	
	``These are facts, theories, or artifacts which can no longer be improved.  It is these final stable forms which determine the stability of the Web.  Let me hasten to assure you that knowledge of irreducible forms does not put an end to science or to any other evolution of culture.  It simply means that work in that particular field has been essentially finished, and those desiring to do original work must turn to other fields.''
	
	``It may seem that I have wandered far afield from diplomacy, but the relevance is to military science.  The fact is, certain weapons and tactics have long since achieved final stable form.  It is the sad truth that gravitational weapons are invincible.  Armed with the gravitronic gate, an attacker may appear without warning within the boundaries of any defense and release energies on any desired scale, including the sort of explosion that begins a new universe.  Under these conditions, an attacker may destroy any target desired and there simply is no defense.  Your own nuclear weapons are an earlier form of this permanent advantage of offense over defense.''
	
	``As a result of this unpleasant but very real fact, the continued existence of civilization cannot be assured by arms, but only by the dispersion of civilization to more centers than can be tracked or destroyed.  In addition, it has been found desirable to set up a sort of technological 'immune system' in which individuals or collectives desiring access to the Web must, as it were, surrender their guns at the saloon door.''
	
	This was the first point in the discourse of the Ambassador in which there was a pause designed to permit any sort of human response.  It was a gale of uncomfortable laughter.
	
	``Consequently you can now begin to understand the First Term of the Access Agreement.  It is designed to assure that the Web at least cannot be attacked on a large scale from within.  I quote from our formal English translation of the document:  'Term 1.  Web Security.  The Accessing Entity shall surrender all records, data files, computer programs, books, engineering drawings, genetic material, weapons systems, spacecraft, and any and all other realized or unrealized cultural artifacts to the Software Security Section of the Web Access Authority to be inspected for the presence of unlawful weapons or plans for unlawful weapons.  Such inspection does not constitute permission for other Entities to inspect those artifacts or records.  Unlawful artifacts shall be placed in quarantine so long as the Accessing Entity is within the territory of the Web and shall be returned to the Accessing Entity only upon exit from the territory of the Web.''
	
	The Ambassador raised its head and said, ``The meaning of 'weapons or plans for weapons' is an extremely difficult concept that will form the basis, I am sure, for much discussion and debate.  The point, less formally, is that anyone or anything entering the Web is subjected to a very thorough inspection and certain powers are quarantined or constrained, perhaps by implanted or accompanying machines or programs, so long as they are in the Web.  Persons or institutions of good will should not be hampered by this restriction.  Obviously, the cosmos is vast and those who would feel hampered by this restriction cannot be prevented from remaining apart from the Web and pursuing their own destiny in territory of their own choosing.  In fact, there are more individuals and even more civilizations outside the Web than there are inside it, although almost all civilizations with gravitronic gates do have some form of commerce or communication with the Web.''
	
	``It is somewhat safer inside the Web than outside.  You may be forming the view that the cosmos is a dangerous battleground.  Actually your own situation here on Earth is far more dangerous, as your own news media attest.  However, it is true that not even the greatest cities of the Web are assured of eternal security.''
	
	``Now on the second Term of the Agreement.  Again I quote from our formal translation.  ''Term 2.  Criminal Acts.  The Accessing Entity agrees without condition (a) not to kill or destroy any Accessing Entity except in proportionate self-defense and (b) not to knowingly state any falsehood in Web Cyberspace.  Entities violating this agreement are liable to action at the proper Court of the Web Judiciary for damages and/or restraint.``  These are simple concepts that I feel sure you all intuitively understand.  In political terms, it means that the Web is, above and beyond any other function, a court system.  Our highest officials are Judges, and our military and security arm is a police force.  Insofar as the Web has a constitution, it is a system of case law and predecents reaffirmed countless times.  I suppose this political basis could be considered, in terms of human political philosophy, a minimal state.  However, it would be very misleading to think of the Web as being such a state, as the institutions developed upon this apparently minimal basis are quite elaborate.  This brings us to the Third and final major Term of the Agreement.''
	
	``Term 3.  Access to Web Cyberspace.  The Web grants permission to the Accessing Entity to copy any record or data file from Web Cyberspace memory, expense of copying to be paid in the form of an Access Charge and optional Access Surcharge by the Accessing Entity.  The Accessing Entity may copy any record or data file to Web Cyberspace memory with or without an optional Access Surcharge to be charged in addition to the Access Charge and reimbursed to the Accessing Entity.''
	
	``This is the commercial foundation of the Web.  What it means is that you can sell anything you like via Web Cyberspace simply by copying the thing with a price tag attached.  Or you can give things away by copying things without price tags.  Virtually all of our property law is intellectual property law and it is much simpler than your human patents and copyrights --- on the surface.  In practice it is perhaps more complicated.''
	
	``Some among you may have been wondering what you, a new civilization with small numbers in a period of political and economic turbulence, have to offer a civilization older than your entire universe and of apparently infinite extent and numbers.  I do not know the answer to this question, but I suspect that those who copy human artistic, scientific, and technical works into Web Cyberspace will meet with some sort of reward.  I suspect the reward may be small in Web terms but not negligible in human terms.''
	
	``My final point before taking questions concerns freeware.  These are those records copied into Web Cyberspace without an Access Surcharge.  Almost every irreducible cultural form has been placed into Web Cyberspace free of any Access Surcharge.  In other words, the basic intellectual heritage of the Web is available virtually for free.  For example, those of you who elect to join the Access Agreement may download complete self-assembling plans for a nanotechnological starship with life-support systems suited to human beings, gravitronic gate and navigation system, a basic Web Cyberspace Access node, and all lawful weapons systems for the basic Access Charge alone, which in United States currency is about \$0.25.''
	
	``I am now prepared to entertain questions.  Mr. Secretary General, if we could begin with the heads of state?''
	
\chapter{The Third Opening}

	Later, when we compared recordings, it was actually I who turned out to be first to see the opening in space.

	My great-great-great-uncle Jack, who'd been born on Earth during the Collapse, his daughter Albertina, who I loved, and I, who like her had been born in the Pleiades, were in suits in open space, inspecting crops.  Jack's farm, seeded by robots on a comet long before my birth, finally had swung in close to our little sun Astyr, had sprouted and bloomed, and now was ready to harvest.  

	The scent of tomato sap and hydroponics remained in my suit as I cycled out through a greenhouse airlock.  Astyrlight dazzled along the space frame holding the greenhouses.  I'd finished my round of inspection an hour before dinner.  Since I'd been away at college and out of space for three years, I jumped off the frame behind the farm for a gaze.

	I drifted a while, bent at the waist to turn myself away from the green spiderweb of the farm, lifted my arms to halt my rotation, and ended facing into the central star-field of the cluster.  Astyrlight warmed my back.  My radio relayed the quiet music of electrons spiraling down lines of magnetic force.  

	The central Sisters were fiercely bright, some swathed in luminous veils, some half-hidden in their sacks of shadow.  Astyr's planets were visible, all three, the innermost one showing a tiny crescent and two moonlets.  I noted the five fuzzy green stars that were the other fields of Jack's farm.  I even fancied I saw the blue blink of the beacon on Jack's tug.

	Then something sparkled at ten o'clock low, moving at ten degrees per 
second to one o'clock even.  I couldn't quite see what it was, so I 
automatically closed my eyes to interface and stepped up the magnification on 
my suit telescope.

	My heart hammered.  I assumed I'd been passed by some vessel wantonly firing her jets without first broadcasting the required warning.  The three of us might already have been cooked by the radiation.  How close had the stray passed?  What I'd seen had been wide, wide in the sky of space.  

	But in the cyberspace window of my suit telescope I saw nothing at first --- then there it was --- a patch of different sky moving through the sky.  The sparkling occurred as the stars of my sky winked out at the edge of the circle and the stars of the other sky winked on.  That was all.  Suddenly the circle was moving very fast indeed, or perhaps it closed up, and it was gone.  

	I had never seen or heard of anything remotely like this.  In cyberspace, I brought up my comm window.  I opened my mouth to broadcast --- and then I shut it again.

	Come on, Paul, I told myself.  You saw Them, aliens at long last?  

	Sure.  

	That usually happens to rock rats who've been jacking off with cyberspace girls for about twenty years.  Or greenhorns from Novatown who get dizzy when they see the stars spinning under their feet for the first time in their lives.  Rather than yammering away on the emergency band, I forced myself to listen.  

	Nothing.  

	I sent a scanning program to track up and down all bands for keywords ``wide, fast, unidentified, circle.''  The query returned nothing.  

	Nobody else was reporting anything unusual.

	I forced myself to relax.  I dared to look again with naked eyes into the starfields.  They were as serene as they ever had been.  Well, I hadn't been in open space for three years.  It is spooky.

	I checked my watch.  Time for dinner.  Maybe I would mention what I had seen to Albertina and Jack.  Maybe there was no need to. 
 Perhaps the moving patch of different sky had been a hallucination.  Probably it had been.

	Then I noticed something odd in my navigation window.  The distance to the tomato field was ticking over rapidly.  I looked back.  The field was a mere emerald snowflake, swiftly diminishing.  I was moving away from it at nearly a klick per second.  

	And I had never once fired my jets.

\begin{center}\rule[3pt]{2in}{0.5pt}\end{center}

	At first Albertina was a red light winking among the other stars.  Then she was a wiggle of silver with a blinking red head.  Then she was a starfish.  Then she was recognizably a girl in a spacesuit.  Then we caught hands.  Our relative motion set us spinning, till our suit pilots stopped us with feathers of steam.

	``Merge suits?'' said Albertina's voice in my ear.

	I grasped the merge seam on her side and brought my own up to it.  ``Confirm merge suits,'' I said.  

	``Merging suits,'' said my suit.  There was a hiss and suddenly Albertina and I were in a tiny tubular tent.  She shook her straight black hair in my face to irritate me.

	``Your hair's too long for a spacer,'' I said.  I tried not to sneeze as Albertina's scent filled my nose.  It was a strong scent.  She had been living and working in her suit for days, as we all had.  However, her smell was far from ugly.  Under the clean sweat and dry soap was a heady vanilla musk.  Albertina had become a woman while I was away at my studies.  I had seen it in her calls, I had seen it under her suit, and now I was smelling it.

	``Who are you to talk?  You're no spacer any more,'' said Albertina, shoving me back so that I bounced off the wall of the tent.

	``Once a spacer, always a spacer,'' I retorted.  I smiled to see her growing up, as if I had finished that job myself.  She wore green shorts and a black T-shirt which set off her wide green eyes (I wore a one-piece suit liner of grey cotton).  

	Albertina and I were born in a sparsely populated region.  The biggest crowd I'd ever seen before leaving for college was at my mother's funeral --- 543 of the 600 or so people then living in the Astyr system had come.  There was no other boy closer to Albertina's age in the system, and only one other girl suitable for me.  So Albertina's parents, my father, and even I had all assumed that as soon as Albertina and I grew up, we'd get married.  Albertina never said anything about it one way or the other.  And I'd pushed this assumption to the back of my mind in Novatown, where I'd enjoyed a few student liaisons.  

	But now I realized that childhood had prepared me very deeply to desire the young woman who floated before me.  I wouldn't have hesitated to act on my desire in Novatown, but out in open space with Albertina things were different.  For the first time I realized in a conscious way that spacers have taboos about starting sexual relationships with people in close quarters, and I understood the reason why --- there's no crowded corridor on the other side of the city to move to when the relationship ends.  Any fooling around spacers do is with strangers --- we either never touch friends, or we marry them.  So the whole question of Albertina rushed in full strength to the forefront of my mind.  Did I want to marry her, or not?  More importantly, did she want me?

	I almost hoped Albertina wouldn't even notice my desire.  And yet I hoped she would.  ``It's good to see you,'' I said, smiling cautiously.

	``It's good to see you, Paul,'' she said, and kissed me.  I was so surprised I relaxed and enjoyed her eager mouth.  I felt a dizzying uprush of desire mingled with deep affection.  ``I love you, Allie.''

	``I love you too, Paul, I've missed you terribly, and now I'm a woman and we can get married.''

	I paused.  ``Don't you think we're too young to get married?''

	``That's what all the old fogeys say, but they also say a lot of our problems come from not getting married young enough --- it makes us too suspicious of the opposite sex.''  Albertina spent more time in the recondite reaches of cyberspace, even living on the farm, than I did at my institution of supposedly higher learning.  ``I'm going to skip my last year of high school.  We can live together and go to college together in Novatown!  Won't it be romantic?''

	I admitted to myself that the prospect was more than attractive.  But how well did Albertina know herself?  How long would her feelings last?  I wondered if she could be content with me for more than a few months, especially after she got a good look at some of the rather imposing guys in Novatown, where I was basically a hick college student.

	Or was I more than a little bit afraid of someone who seemed so much more sure of herself than I was?

	``Request merge suits,'' said Jack's voice in my ear.

	``Oh,'' said Albertina.

	I let go of her. ``Confirm merge suits.''

	``Suits merging,'' said the tent.  There was another hiss as Jack wriggled into the growing tent, which became an angular sort of ball.

	``Well, you two seem to be getting along rather well,'' said my great-great-great uncle.  He didn't seem terribly displeased.  I began to wonder if I was the object of some sort of conspiracy.  The tent enlarged itself, and the spacesuit faceplates migrated together to form a window.  

	Jack was a stocky, crew-cut man with a close-cropped grey beard.  He'd been many surprising things in his life, such as a military pilot on Earth who had actually killed people, and even a professional piano player --- or so he claimed.  I had never even seen a battle, much less a piano (except, of course, in cyberspace).  Jack had piloted one of the first starships to enter the Pleiades, and was over four hundred years old, even discounting the dilated time of interstellar voyages.  I liked him, and not just because he was Albertina's father.

	The tent got big enough to get comfortable.  We hooked our thumbs in our belt loops and sort of half-curled up.  The mini-robot in Jack's suit came out and began making dinner.

	I remembered what I'd seen, and watched the stars spin by our little window.  I noticed that Jack and Albertina were also gazing out.

	Jack said, ``I think I must've spent half my life in open space.  Did I ever tell you I was an astronaut in Earth orbit 78 years before interstellar flight?''

	``Only about a thousand times,'' said Albertina.

	Jack went on, ``Even way back then people were wondering where They are.  And ever since then it's just been one of those thoughts that refuses to go away, even though the human race has been expanding into space for twelve hundred years at almost the speed of light, and in all that time we've never found another intelligent species, much less one with radio or interstellar spacecraft.  But, you know, somehow people keep seeing things.''

	My stomach felt light.  Had Jack seen what I had seen?  Was this his way of checking out Albertina and me?  I shot her a look:  she shot me a look and raised an eyebrow:  so did I.

	``Seen what kind of things?'' I said.

	``Did you ever see anything, Dad?'' said Albertina.

	``Oh, flying saucers, little green men, moving lights...''

	Albertina made a face.  I repeated something a philosophy professor, Julie Wiggins-Chou, had said in one of those seminars that degenerates into an undergraduate bull session:  ``Well, where are they?  Wouldn't we have run into aliens by now, if there really are any, or at least picked up their broadcasts?''

	Albertina sniffed.  ``The universe is expanding, right?''

	``So they say,'' said pragmatic Jack.

	``And nobody can go faster than light, right?''

	``Well, there might be wormholes,'' I said, recalling my Quantum Gravity course.  For the first time I wondered if what I had seen had been a wormhole:  predicted by theory, but never observed.  Until now, maybe.  It was a thought that made me dizzy --- as if I might fall into it.

	``But you wouldn't go faster than light in a wormhole,'' said Albertina.  ``You'd just connect two corners of a timelike region in a spacelike way.''

	``What are you getting at?'' asked Jack.

	Albertina went on, ``Since we ourselves seem to be alone, intelligent species must be pretty rare, and anyway it certainly takes a long time for them to evolve, right?''

	``Obviously,'' I said.  ``Billions of years.''

	``If we can judge from just one case,'' said Jack.

	``So if the universe is expanding faster than intelligent species arise, then almost all intelligent species will be trapped in what amounts to their own universes,'' Albertina finished triumphantly.  

	Ha, Professor Wiggins-Chou, I thought.  ``Unless there are wormholes,'' I said.  The conversation was beginning to seem unreal.  Had I really seen anything, after all?  Maybe it was just a flying lens, a big chunk of ice.  But no, we were too close to our sun for that.  And anyway, there was the blasted speed I'd picked up out of nowhere --- that simply could not be explained away.

	``What would a wormhole look like?'' asked Jack.

	``I don't know,'' I said.  ``Maybe like what it is, just a hole in space.  And through it you see different stars, different constellations.''

	Jack and I looked steadily at each other.  ``Did you ever see anything like that?'' Jack asked me.

	``Did you?'' I asked him.

	``Oh, stop playing games!'' Albertina burst out.  ``I saw something exactly like that!  About an hour ago!''

	``So did I,'' I said.

	``I thought I saw something that looked like that,'' admitted Jack.

	The tent seemed very small then.  The robot finished our meal, and we ate in silence.  The smells of coffee and hot food mingled 
with Albertina's musk, Jack's smell, my own sweat, light machine oil, ozone:  the scent of humanity living in open space.

	Jack finished a piece of baklavah and washed it down with cappucino from a bulb.  ``Ah, that was good,'' he said.  ``You know, about a hundred years ago, I also saw something...''

	``Yeah?'' said Albertina and I.

	``Yeah, I was on a crew setting up farm frames.  We were working hard, twelve hours on eight hours off, sleeping in our suits, just like we are now.  One shift I couldn't quite get to sleep.  I was stargazing, half-asleep, and I saw a long, beautiful starship, all silver-hulled and gold-windowed, slide by smooth as a whale dreaming in a tropic sea.  She went by so close I could see right through her windows.  There were all kinds of people in there, in gowns and uniforms with medals and jewels, human and non-human, dining and dancing and waving at me as that ship went by.  And I heard the most unearthly music from within her hull...''

	``Come on, you can't hear anything in space,'' I said.

	``Obviously a beautiful dream or hallucination,'' said Albertina.

	``But I saw it with my own eyes,'' said Jack, with a smile.

	``Well, I bet you didn't get a recording of it,'' said Albertina.  She made the window of our tent into a cyberspace screen, and on that screen she replayed what I had seen with my own eyes:  a circular aperture in open space, showing a different pattern of stars within, moving slowly and then more rapidly across the star-fields.

	I had been so unwilling to believe what I saw that I hadn't checked my own suit's memory.  I did so then, however, and all three of us watched the recording.  Jack's suit had recorded his sighting, too, more or less identical with ours.  I also mentioned the 
speed I'd picked up.

	``Whatever it was, it's real enough, all right,'' said Jack.  ``Looks like you saw it first, Paul.  I suppose we'd better forward these recordings to the Institute for Interstellar Navigation.  They have a slush file for this kind of thing.  Maybe someone will even call us back in a few years with some questions.  Now, children, we have a lot of work to do tomorrow, and I for one am going to try to get some sleep.''  He pulled his suit on, detached from the tent, and jetted off into the distance.

	``Good sleep, Dad,'' called Albertina on the radio.

	We turned to each other.  We were both very excited, as much by the presence of the other as by what we had seen, whatever it was.  In the tiny tent, we embraced again.

	``Let's go sleep in the corn field,'' I said.  We would have to inspect it next anyway.  ``There's a lot more room there.''

\begin{center}\rule[3pt]{2in}{0.5pt}\end{center}

	Albertina and I made love for the first time in the corn field.  Then we bounced up and down the greenhouses with nothing on, splashing water on each other, laughing like maniacs, and talking about our plans for the future.  I found it easy indeed to set aside any reservations about our youth.  

	We would go to Novatown together.  Albertina would study music or mathematics:  she was still not sure which.  I had one more year left for my B.S. in Engineering, after which I would go directly into graduate school, specializing in astronautical engineering.  

	Sleep never came, of course.  Albertina and I floated between sunlit rows of corn, talking and talking.  Then we made love again.

	``That was a good idea you had about the universe expanding faster than intelligent species do,'' I said, feeling, after our second sweet bout, that she and I were at least equal to the universe.

	``It's pretty obvious.  The universe has a radius where its speed of expansion is effectively the speed of light.  That radius amounts to the edge of the universe, and technology can never go fast enough to reach it.''

	``But there might be wormholes, other universes...''

	``I think there are, and you may even have been the first human to see one,'' she said.  ``But, assuming we did see a wormhole, we have no way of knowing whether it was a natural one or an artificial one.  It's funny how we all seemed to assume it was artificial.  Maybe they just naturally spin off from black holes, or something.''

	I shivered.  ``I think that's physically impossible.  God, suppose it was artificial!''

	``Wouldn't that be wild?'' she said dreamily.  ``If my argument about the expansion of the universe is right, there might be a wormhole civilization, just like ours is a starship civilization.  It could be ancient, older than our universe even, with countless intelligent species.  Yet there might be many more intelligent species and civilizations that never even discover the wormhole people.  It might be a very lucky chance to meet them.  The cosmos is so very vast, dearest Paulie....''

	``It certainly is,'' I said, squeezing her hand and looking up, wide-awake, through the greenhouse roof into the dark and shining clouds of the Pleiades, past them to the star-milk of the Galactic limb, and past that to where other galaxies, mere faint and fuzzy stars, gathered in the black void.  Albertina's dear warm body seemed, to me, vaster than all that.  ``I love you, Allie.  Don't ever leave me!''

	``I love you too.  Oh, I just can't sleep...''

	The wakeup alarm chimed in my ear.  I groaned.  ``It's time to go to work again.  You can take this field, and I'll jet over to the orchard.''

\begin{center}\rule[3pt]{2in}{0.5pt}\end{center}

	It was a two-hour jet in my suit from the corn field to the orchard where Jack grew apple and pear vines.  As I coasted I sucked liquid breakfast from my suit and tried not to stargaze too much, for the truth was I felt a bit dizzy.  The peace I'd always known in open space was shot to hell.  I wondered if I'd ever get over this fear of falling through some sudden hole in existence.  

	My mood also jetted and braked as I thought how Albertina had become still more precious to me.  Evidently she'd made up her mind about me while I was away at school, but I had to wonder what in space she saw in me and when, probably soon after moving to Novatown, she would wake up and no longer be in love with me.

	Naturally, it was while I was in this unstable frame of mind that the wormhole appeared again.  I knew it was closer because it was much larger.  The stars at its edges were shifted in color to form a kind of thin, ghostly rainbow.  I felt no acceleration, of course, but I understood that the gravitational forces which had opened the wormhole must be pulling at me, just as they had set me moving the two shifts before.  I immediately interfaced.

	``Allie!  Jack!  I see it!''  I brought up my navigation window, managed to obtain some sort of fix on the edge of the wormhole, and transmitted it.  I shivered when I saw the numbers:  the opening was mere kilometers away.

	``It's back!'' cried Albertina.

	``I can see it, larger than yesterday,'' confirmed Jack.  ``Be careful!''

	I stepped the magnification on my suit telescope up to the maximum.  Within the opening I saw lights moving slowly, white and actinic violet lights with streaming tails --- not stars, but starship jets.  

	I groaned with fear and awe.  Alien traffic was passing by on the other side of the wormhole.  Within the one opening, too, I saw three others.  Even as I watched, a blinding needle of light emerged from one.  ``Oh God!  There are other holes in there!  A ship is coming through one!''

	``I'm transmitting the view from my scope,'' said Albertina.  Her voice was shrill with excitement, but there was no tone of fear.  ``We each have a different view through the wormhole.  It's a sphere, I bet, but to each of us it must appear as a circular opening.''

	A window appeared in my cyberspace interface to display her transmission.

	Two rough cones of glowing cloud spanned half the sky.  Where the points of the white cones met, there was a small but intensely brilliant violet disk.  I knew from my astronomy classes that I was looking at the accretion disk of a large black hole.  I guessed that this hole either generated the wormholes naturally, or more likely was used by the wormhole civilization as a source of energy for creating artificial wormholes.  By that time I was in a state of shock, beyond fear or awe, simply recording everything I could like a machine.

	I glanced from Albertina's transmission to my own scope.  In one corner it showed an indisputably artificial structure, a thin hoop encrusted here and there with brighter patches of light.  My scope returned a fix:  distance 10.2 million kilometers, diameter 450,030 kilometers, width of the hoop 15,200 kilometers

	The alien city or structure was wider than Earth, and created a circle in space almost the size of the Moon's orbit.

	``I'm viewing what appears to be an enormous city,'' I said, ``transmitting view and coordinates.''

	``I see it,'' said Albertina in a perfectly calm tone of voice.  ``I'm going in, I'm crossing the edge of the hole.''

	``Be careful, the hole appears to be moving,'' said Jack.

	``Stop, Allie!'' I cried.  I got out of cyberspace and looked with naked eyes.  The hole was moving directly towards me, getting larger; instinctively, I fired my jets to move aside.  I saw the tiny red blink that marked Albertina traversing the faint rainbow at the edge of the hole.  There came a brief flash of distant jets.  Then Allie's beacon was moving faster and faster as the gravity drew her in.  Even with unaided vision I glimpsed, also, the alien city, a minuscule circlet of artifice in the violent sky of the black hole.

	The wormhole moved past me, and became smaller, and closed.

\begin{center}\rule[3pt]{2in}{0.5pt}\end{center}

	``They'll be here in 17 hours,'' said Jack.  ``Buck up, Paul.  She's not dead.''  He sounded like he was trying to convince himself.

	We floated next to each other exactly where we had seen Albertina vanish, waiting for the wormhole to open again.  The tug floated behind us, programmed for intercept and rescue.  The ``they'' Jack meant would be Rangers, reporters, maybe professors from our local college, God knows who.  I just wanted the hole in space to open again so that Albertina would come flying safely out.

	``I wonder whether she fell in, or if she jumped,'' said Jack.  ``You couldn't quite tell from what she said.''

	``We don't know anything,'' I moaned, remembering the brief flash of Albertina's jets, and thinking how much like her it would be to jump --- out of sheer curiosity.  It was a thought I didn't want to look at too closely.

	In the brilliant Astyrlight, I could see Jack's face even through his reflective faceplate.  He had that distracted look that comes from listening to the radio.  His lips moved as he replied to some distant authority.

	I looked at my watch.  ``It's been 18 hours since the wormhole closed.  It was only 12 hours between the first one and the second one.  It should have opened six hours ago!''

	Jack glanced at his watch and frowned.  ``My watch says it's been 17 hours 20 minutes since the wormhole closed.''

	I grabbed him and pulled him closer.  I held my arm next to his.  His watch said ``10:53:42.''  My watch said ``09:13:16.''  We looked at each other.

	``The first time the hole opened,'' I said, ``It pulled me in space, pulled me toward it at 890 meters per second.''

	``It pulled me too,'' said Jack, ``but not as fast.  It must have pulled you right out of time, slowed time down.''

	``God,'' I said.  ``We not only don't know where Allie is, we don't even know when she is.''

	I fell silent.  I let Jack do all the talking to the authorities.  I stargazed.  I had odd thoughts:  We're going to be famous.  Or, why did we let Jack talk us into going back to work --- but that wasn't fair, for at that point we had seen nothing but odd twinkles in the stars.  Or just Allie, Allie, Allie, with a dead feeling.  Or I would even forget my grief and anxiety for a few minutes and ponder what I'd seen and recorded:  wormholes, the jets of alien starships, a city as large as a planet and quite likely more ancient.  Maybe it would somehow be possible to go in there, to visit that city....  Then I would remember Allie and feel guilty for thinking about anything else at such a time.

	After a time I slept, dreaming the impossible, that of course she would return, that the hole had opened twice and would open again, that things would be once more as they were before.

\begin{center}\rule[3pt]{2in}{0.5pt}\end{center}

	I awoke, not remembering at first what had happened.  I saw Jack next to me, upside down, his expression exhausted even in sleep.  I saw the tug, winking behind us.  Then I remembered.  I consulted my watch.  The Rangers would rendezvous in two hours.  

	I searched for the wormhole --- above, all sides, below --- though the tug would have awakened us if one had appeared.  

	My mouth was dry and foul.  I sucked water from my helmet nipple.

	Then motion beneath me caught my eye:  the nose of a ship was appearing out of nowhere.  The whole ship emerged from a blurry circle barely wider than itself, a ship two hundred meters long, as thin and bright as a needle.

	I tapped Jack's helmet.  He woke up, startled, and spun for a moment until his suit stabilized.

	I was already calling ``Albertina, is that you?,'' jetting toward the alien vessel.

	``Paul,'' answered Albertina's voice on the emergency band, though there was something different about it.  I did not think about that.  I was looking for an airlock, an entrance to the ship.

	``Allie, are you in there?'' called Jack.

	``Paul, Jack,'' said Albertina's voice.  ``I read you.  Come aboard, Paul.  Father, please stay outside for now.  I'll be with you in a short while.''

	A hangar door appeared in the side of the ship.  Orange lights burned within.  I jetted forward and was soon inside.  The stars vanished behind me as the door closed.

\begin{center}\rule[3pt]{2in}{0.5pt}\end{center}

	Wrapped in olive satin, a full-figured woman floated out of the hangar's inner airlock.  She had grey-streaked black hair and dark green eyes.  ``It's all right to unsuit,'' said Albertina, grasping my arm.  

	I gaped at her:  she was old --- not ugly, not senescent, but old.  Allie had been seventeen.  But this Albertina was more ancient than her own father.  Time had nested in faint lines at the corners of her eyes.  Time had pooled in the depths of her eyes.  She looked at me patiently, hopefully, sadly.

	As I peeled off my suit I was acutely conscious that I needed a shave, that I stank of sex and the sweat of fear.  Albertina did not appear to notice.  She drew me after her through corridors I scarcely saw.  We came to a library with recessed lighting, a high domed ceiling, black furniture, alien plants.  ``Not that one,'' said Albertina, moving me away from a rickety little stool with a slot for a tail.  She sat me down in an overstuffed armchair and poured tea from a pot that seemed alive, for it withdrew its snout with a purr after she set it down.

	``I remember being young with you,'' she said meditatively.  ``I remember it very well.  That is why I decided to meet you privately before speaking to anyone else, even Jack.  I really don't know where to begin, although I've been thinking about this meeting for years.''

	``How long have you been gone?''

	``I'm eight hundred forty standard years old, Paul.  You're twenty-two.''
	I was silent, looking at her, then at the moving pictures of alien soldiers on the inside of my teacup, then at her still face again.  My feelings, which had been quite chaotic, simply vanished.  I was calm --- for a while.  ``I'm very glad you're still alive.  Was it hard for you?''

	Albertina tilted her head and chuckled.  ``You have no idea how hard.  I've been a stranger in a very strange land far longer than I ever lived among humans.  You may be afraid of me, well, I'm afraid of you also.  It hurts me to see you.''

	``You've got to tell me the whole story.''

	``I will, I will, that's why you're here.  Listen, first of all, you have to understand that I went into the wormhole quite deliberately.  I had no idea whether it would open again, or where, and it just seemed obvious that it was worth my life to see if I could make it on suit jets to the city you showed me, to see if they would be interested in me there, if they could keep me alive.  To make contact!  And I'm sorry, but it also seemed worth giving you up to go there.  By the way, I've learned that the wormholes we saw, that I went through, were not exactly natural and not exactly artificial.  They were side effects of a new artificial wormhole.  Gravitational resonances, accidental openings.''

	I swallowed.  ``Was it worth it?''

	``Oh, yes.  You'll understand.''

	For the first time, I looked round the ship's library.  Between granite columns stood skeletons of red armor, twice the height of a man, whose eyes moved:  they were watching me silently.  Above crystal-fronted bookcases glowed abstract images that might be works of art, or perhaps data models.  A long worktable at standing height ran along a window onto the Pleiades.  The lamp on the table had been made from the papery skull of some creature with three eye-sockets, a huge brain-case, a round toothless hole for a mouth.  It cast an even white light.  Data moved on the surface of the table, which was piled with scrolls, codexes, printouts, small incomprehensible machines.  

	The room was serene, beautiful, disciplined, the study of a great scholar.  

	It was apparent that Albertina, who even as a girl had perhaps been beyond me, had now truly far surpassed me.  I was glad she was alive but in a very real sense I had lost her.  She had gone through the wormhole and become old and great, and I was small, one who had stepped aside from the opening onto a new creation, a callow youth, ignorant... for that matter, a mere human being.

	I rubbed my wet cheeks.  ``I'm so glad you're alive.  You seem to have done very well.  I'm proud of you.''

	``You couldn't know... I've done fairly well and I've done very poorly.  I've learned a great deal, but I made serious mistakes and I'm heavily in debt in ways you may never understand.  Possibly the human race will agree to pay off my debt.  If it can.  Coming back here, to this moment in time, for one thing.  That wasn't cheap.  They had to bend the wormhole.''

	``Who are 'they?'''

	``You'll meet them.  Many races.  Thousands.  Some older than our whole universe.  Some --- artificial.  It's wonderful and it's incredibly difficult.  Yet the city, Beis, it's largely abandoned.  The original builders left before humanity evolved.  I don't even know why.  There are various theories, contradictory ones.  Possibly there was or is a sort of war going on.  I don't know.  Beis is linked through the wormholes with many other cities, most of which are not abandoned.  I haven't visited any of the others yet.''

	``We have a lot to learn,'' I said, feeling how foolish this was to say even as I said it.  For the first time I realized my assumption that Albertina would even permanently return to the human universe was just that:  an assumption.  ``Are you coming back?''

	``Very good, now you're getting to the point,'' said Albertina, rising from her chair and pacing back and forth by her worktable.  ``That depends very much on you.''

	``On me?''  

	``Obviously I've traveled back in time specifically to meet you here.  Basically, when --- when Beis understood that I had left you behind, I agreed with it to go back and give you a chance to join my younger self in Beis, today.''

  	``'If you do, then I will have never existed, or rather I will still come to exist, but perhaps in a very different form than my present self.  What you and my younger self do after you join her, if you join her --- that is entirely up to the two of you.  Quite possibly you will both choose to return permanently to the human universe.''

	``If, on the other hand, you don't join my younger self, then I will continue to exist, perhaps in yet another different form.  In that case, I suppose I'd return to the future and to Beis.  I expect that other people, humans, would eventually join me through other wormholes I plan to open, later ones.''  

	``'Whatever you decide, I'm sure I won't always be the only one of my kind in Beis.''

	I was still in shock from gaining Albertina, and losing her, and regaining her, and losing her again, not to mention seeing into the wormholes.  ``Are there aliens on this ship?'' I asked nervously.

	``The ship herself is intelligent and has a kind of consciousness, but she is the only non-human with me.''

	I looked around.  ``Ship, are you listening to us?''

	``Not until you asked,'' said the air.

	I could not help but jump in my chair.  I felt intensely uncomfortable.  Suddenly I realized I was very tired.

	``Obviously this is a great deal to handle all at once,'' said Albertina.  She smiled, sort of.  ``Now you're feeling a bit of what I felt when I first landed in Beis.''

	``Jack's out there all by himself wondering what's going on,'' I said, ``and Rangers and reporters and professors and God knows who are going to be arriving any moment now.''

	``I do need to talk to my father, and to the others,'' said Albertina.  ``In the meantime, I suggest that you take a bath and get some sleep.  In here...''

\begin{center}\rule[3pt]{2in}{0.5pt}\end{center}

	I awakened alone, clean, and refreshed in a strange dark room.  But this time, even as I awakened, I knew exactly where I was:  on an alien vessel with a strange Albertina from the distant future.  I lay quietly, trying not to wake up all the way, for I was afraid that if I did I would not be able to breathe.

	The room had a window, real or cyberspace I did not know.  I turned my head to look out.  That was a mistake.  

	A chaotic jumble of human spacecraft glittered in the Astyrlight, swarming with people in spacesuits like golden flies on silver fruit.  I took a deep breath.  Sometimes my mind worked, and sometimes it didn't --- it advanced in jumps.  One thing I realized right away:  No matter what I did, I would never again lead a normal life.  

	But this momentous event had happened through sheer luck.  It had nothing whatsoever to do with my faults or virtues.  Somebody had to be first to go through this, and it just turned out to be Jack and Albertina and me, that's all.  As soon as I realized that, I felt a little better.

	I swung my legs around the side of the bed and looked for my suit liner.  It was tucked under a strap at the foot of the bed, as clean and wrinkle-free as new.  I put it on.  Another thing I realized:  I really could go to Beis and be with Allie.  Albertina had said she'd ``agreed with Beis'' --- the city, by implication, was intelligent, it was certainly ancient, presumably it was very wise, but did it mean well? --- to let me join up with Allie.  But was that what Allie wanted?  Was it even what I wanted?

	The enormous scale of events seemed to have crushed the sweet love I had felt for Albertina only two days ago completely out of my soul:  I felt nothing, or rather I felt such a rapid succession of conflicting emotions that they canceled each other out and amounted to nothing.  I supposed, abstractly, that I would possibly, even probably, feel my love for Allie come fully alive again in the future, especially if she returned it.  

	And as for my career!  Graduate school!  Astronautical engineering!  I stole another glance out the window.  The  Ranger cutter had arrived and was slowly warping towards the confused mass of tugs, shuttles, and lines.  The vessel I was on had been built by a tradition and science quite certainly older than my entire race.

	I resisted the temptation to pick up my calls, or to eavesdrop on the traffic outside.  Obviously Albertina was giving me time and privacy to consider the choice she had offered.  I returned to the bed, hooked a foot under the strap, put my head in my hands, and tried to think about time travel.  

	I found I did not want to think about time travel at all.  At least, I did not want to think about it by myself, not when there was a time traveler around to consult.

	``Albertina,'' I called.  ``Are you there?  Ship? -''

	The door opened.  Albertina entered and sat beside me.  I turned to her.  ``What does all this mean?  Is this meeting, my decision, and everything we ever do completely predetermined?''

	``Isn't it equally conceivable that every possible decision becomes real in some alternate world?'' suggested Albertina.

	``I don't know what's conceivable, but the idea of alternate worlds doesn't exactly reassure me,'' I said.  ``Why should I experience one of these alternate worlds, and not its opposite?  It seems to be sheer chance.  God, what about our love --- I mean, we were in love, we were going to get married, I suppose I still love you!  Even if you are older than your own father.  I don't know!''  

	``Anyway, what does love mean if my feelings are completely determined, or if all possibilities are real and it's sheer luck that I'm the one of me who loves you, instead of the one of me who hates you?''  I paused.  I had never felt so deeply hurt and bewildered in my life.  After a while I had to ask: ``Do you still love me?''

	``Yes,'' said Albertina.  ``...In a sense.  Don't forget that in comparison with me, you are an infant.  I do love you in my strange and distant memories of being a young human being.  And perhaps I love you in the man I think you might become, if you live as many years as I have.''  Seeing that her words wounded me, she said, ``Have some patience.  You still haven't considered all possible interpretations of time travel.''

	``But what else is there besides determinism and alternate realities?''

	``Think,'' she commanded.

	So I thought.  Albertina stood near the crossing of a loop that she herself had created in time.  Suppose she entered the actual crossing point.  Could she meet her younger self, convince herself not to enter the wormhole, or restrain herself by guile or even force?  If she did, her older self simply wouldn't exist.  That sort of thing seemed to be ruled out by sheer logic.  But I didn't see any reason why Albertina couldn't, for example, merely greet herself.  She might even be able to give her younger self certain instructions.  

	``I suppose,'' I said cautiously, ``that you can go back in time and do anything your older self doesn't undo, like by killing your earlier self.  Wait a minute --- I mean, you can only do things that you couldn't undo.  Yet there must be many things left to do, as long as all the remaining possibilities are consistent with this one moment -''

	I shook my head, put it back in my hands, and sighed.  My personal quandary certainly made the theoretical puzzles I had studied in Quantum Gravity a great deal clearer.  My decision in Albertina's present would determine her past.  Her decision in my future was determining my present.  And my decision in my present would certainly determine, as usual, my future!

	``You're beginning to see it,'' said Albertina.  ``It's very important to me that you realize this:  everything I have done and said to you since returning from Beis, I would have done and said exactly the same, whether you will decide and have decided to join my younger self there --- or not.  But of course, inside my own head I am a very different person, with completely different experiences and memories, if you join me, than I am --- if you don't.''

	``And you may have noticed that you haven't met your own older self.  That would prove that you had decided and could only decide to go to Beis.''

	``Why is it so important to you that I understand this?'' I asked.

	``I want you to be free,'' she said.  ``I want you to see and believe that I really do want you to be free.''

	``Ah,'' I said, feeling the beginning of some pretty vast relief.  ``I do see that.  But what do you want me to do?''

	``I want you to join me in Beis,'' she said.  ``Please.''

	``I will,'' I said.

	``Don't you want to know what it will be like?  And about my debt?  And the war, and all that?''

	``I'm sure I'll find out,'' I said.  ``Later.  Let's get this over with.''

	Albertina smiled for the first time.  She looked old and young simultaneously.  I liked that.

	``Let's leave right away, as soon as possible,'' I said.  ``I don't want to give myself a chance to lose my nerve.  I suppose we do have to go out and tell everybody.''

\begin{center}\rule[3pt]{2in}{0.5pt}\end{center}

	Telling everyone wasn't so bad.  A tent was rigged between the airlock of the Ranger cutter and the airlock of Albertina's ship.  Everybody had a distracted, important-looking expression because they were intensely conscious that they were present at an epochal moment in human history.  As for me, I just wanted to get it over with.  I tried to look alert and answer without mumbling as reporters literally clustered in a ball around me, staring with those impersonal eyes through which billions see.

	Finally it was over.  The last reporter and even the Ranger captain was off the ship.  Albertina, Jack, and I stood at the edge of her hangar.

	Albertina turned to me.  ``The ship is programmed to fly you to Beis,'' she said.  ``I am going back, I mean, I'm staying here.''

	One more galaxy-class surprise.  I shrugged.  I had almost expected it.  I could handle it.  Jack looked glad.  He winked at me.

	``But first, there's somebody I want you to meet,'' Albertina was saying.

	A tallish guy stepped out of the shadows.  He wore a spacesuit of unfamiliar design with a rich, abstract pattern.  He had sandy hair like mine, cut short to the scalp, and dark brown eyes like mine, with the faintly crinkled corners of age.  I looked into his steady gaze.  It was like looking into a mirror, except that my reflection was older and more powerful than I, and looked reversed precisely because he was not a reflection.

	This strange man, who was my older self, gripped my shoulder and shook my hand with a slow smile.  Then he slipped his arm round Albertina's waist.  She leaned into him.

	``That's one decision you got right, kid,'' I said to myself.  I had to ask him:  ``Were you waiting around just out of my sight the whole time?''

	My older self retorted:  ``Before Schrodinger opened the box to see, was his cat alive or dead?''  He went on, ``If you had not decided to go to Beis, I could not have been waiting here.  And if I had met you immediately, before Allie had a chance to explain the alternatives, would you have been free to make your decision?''

	``In a very real sense, the less I say to you now, the more freedom you'll have in the future.  So there's not much I want to or even can say.  But I will say that, when you finally find yourself standing in my shoes, many different possibilities suddenly seem to cancel each other out, and the right thing to do seems fairly clear.  And don't forget, even though it was a very long time ago, I remember asking me your question just now... that's all.''	

	``Let's go,'' said the ship.

	Albertina and my older self took Jack by the hands.  They jetted off the hangar deck and out into human space.

	I remained in the hangar doorway, watching them move towards the Ranger cutter.

	The ship from Beis said, ``You may remain in the hangar until I give warning for high acceleration.''

	``Thanks,'' I said.  ``I will.''

	The ship moved almost imperceptibly.  It rotated 180 degrees about its center of gravity until the wormhole was directly in front of its nose.  Then the ship fell swiftly through that hole in space.  I glimpsed human beings, tugs, and instruments gathered round the hole, and knew that countless people would view this moment in years to come.

	Then the ship was through, and I knew that the wormhole was closing behind us.

	Directly ahead, the city Beis rolled through a sky milky with stars, a bright hoop crusted with patches of light.

	``Please come to the bridge,'' said the ship.  ``I plan to accelerate in fifteen minutes.''


\part{Essays}

\chapter{The Human Future and the Cosmology of Civilization}

The progress of science has seemed, increasingly, to reveal the typicality of our galaxy, our star, our planet, and perhaps of ourselves.  If the human race exists, why not other races, other civilizations?  But where are they --- the signals from space, the spacecraft, the aliens themselves?  We should not be unique, yet we seem to be alone.

To this antinomy, for which I shall propose a simple solution, let us add the basic results of fundamental physics, cosmology, biology, and philosophical logic: the asymptotes of which may be appearing.  These considerations I take as constraints for my imagination of the future.  

Wait!  Don't free will, individuality, and chance make history a strictly indeterminate path?  I don't deny it.  Are the limits of technology, of humanity, of consciousness, and of history so apparent?  I agree, they're not.  And in fact, I will propose what I consider to be some very interesting choices, technologies, and historical singularities.  However, from the perspective of science fiction, the first principle for imagining the future is:   Imagine nothing contrary to the limits of science.  And to this principle I add a second:  Imagine as much as possible consistent with science.

\section{The Antinomy of Sapience}

Our search for extraterrestrial intelligence has only just begun, but it seems to me quite likely that, if any alien civilization had existed for very long in our galaxy, we should already have detected it, incidentally to astronomy.  Our broadcasts, and our modifications of the climate and surface of the Earth, must already be apparent across astronomical distances.  As discussed below, the feasibility of interstellar flight must be assumed.  So there is little reason to doubt that further technological progress will make our existence even more obvious.  I judge that the antinomy of sapience is a true one.  There are no extraterrestrial civilizations within our galaxy, or our neighbor galaxies.  

I agree, however, that humanity is unlikely to be unique in the cosmos.  However, let us recall that the cosmos is expanding.  Civilizations, too, are presumably expanding.  The proportion of the cosmos occupied by civilizations is a function both of their rate of origin, and of their rate of expansion.  If the compound rate of expansion of intelligent life is on average no faster than the rate of expansion of the cosmos itself, then most civilizations must be alone within their cosmic horizons.  If the expansion of civilization is at all slower than the expansion of the cosmos, then in fact each civilization must become increasingly thinly spread in space.  I propose that this is in fact the case.

Axiom: 	Civilizations occupy an infinitesimal volume of the cosmos.

\section{Faster than Light Travel}

Does this mean that each civilization is fated to remain alone throughout its history?  Not at all.  Being thinly spread is one thing, and remaining forever isolated is quite another thing.  It all depends on whether faster-than-light space travel is practical.  According to valid solutions of the equations of general relativity, both faster-than-light travel and time travel are theoretically possible via extreme gravitational torque.  At this time, all conceptual designs for gravitational torques involve objects of astronomical size and singular density, driven by energies of cosmic scale.  Mind-boggling though it may be, in keeping with the second principle of imagination, I must assume the validity of general relativity until such time as scientists accept a superior theory.  These then are the basic possibilities:

\begin{itemize}
	\item 
\item	Despite the theoretical possibility of faster-than-light travel, it is not practical for living human beings.

\item	It is not practical to build gravitational torques, but it is possible to exploit naturally occurring ones generated by celestial bodies, such as black holes.  Under these circumstances, civilizations would gradually migrate to these bodies, even if it took sublight spacecraft many years, or even generations, to reach them.  Flights between torques could only be accomplished between black holes, and would require starships and months or years of ship time.

\item	It is possible to build gravitational torques, perhaps by exploiting a unified field theory to build an electro-gravitational transducer, and therefore to build faster-than-light spacecraft, or even locally stabilized gates.  These would be gates that maintained a fixed position local to the frame of reference on each side of the gate.  However, gravity is monopolar, and so weak compared to other forces that I have great difficulty imagining a torque sufficient for a gate on the surface of a planet.
\end{itemize}

Axiom:	Gravitational torques can be built, but not locally stabilized gates.  Faster-than-light flights, which also may be time travel, can be accomplished anywhere in space, but still require starships and months or years of ship time.

\section{Extent and Structure of Cosmic Civilization}

Modern cosmology suggests, on quantum mechanical grounds, that the Big Bang was a rather arbitrary event, an instability of the vacuum that... just grew one day.  There is no particular reason to suppose that it only happened once.

Axiom:	The cosmos is infinite in time and space, and so is cosmic civilization, for even an infinitesimal portion of infinity is still infinity.

It follows from the preceding axioms that civilizations that survive long enough do contact other civilizations ---  by FTL starship, not by radio.  Most such contacts will occur between two civilizations that each possess starships, most of them will occur as relatively newly starfaring civilizations contact the existing web of much older starfaring civilizations, and most of them will occur as those newly starfaring civilizations fly to investigate scientifically interesting celestial bodies, where they will find starships from other civilizations.  These bodies will serve both as natural beacons and as the true crossroads of interstellar civilization.  There may also be artificial beacons.  However, it does not necessarily follow that the greatest cities will be located at such crossroads.

If civilizations last very long, most of them will be part of the Web.

\section{Military, Economic, and Political Structure of the Web}

Human history is generally written as political history determined above all by war, and there is no reason to think that cosmic history should be any different.  It is said that liberal democracies do not make war on each other, but there is no particular reason to believe that liberal democracies will replace all other forms of civilization.  Even if they did, one evil being armed with a gravity torque could destroy just about anything.

Axiom:	Offense has an overwhelming and permanent advantage over defense.

Under these circumstances, the Web cannot invest its safety in any one fixed city.  It follows that the Web must be founded primarily on starships.  It is infinitely old, and yet none of its cities are infinitely old.

It is to be expected that the Web has pushed technology, and biotechnology, to their theoretical limits.

Axiom:	In additional to gravity torques, the Web has direct conversion of mass to energy, self-reproducing nanotechnology, and complete genetic engineering.  It also has cyberspace that interfaces directly with the brain.

It follows from these axioms that each individual civilization, or for that matter each individual starship or even clan, faces a fundamental dilemma:  remain isolated in the infinite wilderness of space for the sake of safety, or join the Web for the sake of civilization.

It might be wondered whether the Web has already achieved all that can be achieved.  However, the existence of computational irreducibility guarantees that, although the Web possess our hoped-for true theory of everything and has pushed fundamental technologies to their theoretical limits, there is no limit on the development of higher levels of technology and culture.  Even if the basis of civilization has achieved a stable form, the superstructure elaborated thereon never will.

Axiom:	The technological foundations of the Web are infinitely ancient, and are in the public domain.  The only basis for capital under these circumstances is original intellectual work and services.  The medium of exchange is credit for access to files in Web cyberspace.

Irreducibility (incompleteness) also guarantees that intelligence, whether natural or artificial, is self-transcending and self-determining.  The political dialectic of the Web is now becoming clear.  There is incredible diversity, so the only possible stable foundation for intercourse in the Web is liberty and the Categorical Imperative, resulting in a fairly minimalist governmental structure.  Beyond this, there is a fundamental need to balance security against freedom of travel and trade.  There can be no fixed center of authority, but there are more or less stable forms of adjudication.

Axiom:	The current fundamental institutions of the Web, in order of logical priority, are the Process of Judgment, the Cyberspace Foundation, the Software Security Force, the Exchange, and the Interstellar Militia.  All of these are based primarily on starships and starship computers.

Because of the dialectic between the stable technological basis and the evolving cultural superstructure, these institutions are probably metastable.  There is plenty of scope for narrative here.

\chapter{Background to the Aperture}

	I may write more science fiction.  The following is a statement of the scientific and philosophical assumptions upon which I choose to base this writing.  I state these assumptions as facts in the history of philosophy, science, and technology.

\begin{center}\rule[3pt]{2in}{0.5pt}\end{center}

	A number of physical theories have been proposed, all of which unify gravity and quantum mechanics, each of which covers all known observational and experimental data, any one of which may well be the ``one true theory of everything.''  To test these theories decisively would require truly cosmic energies and is thus beyond human, or for that matter nonhuman, power.  Indeed, after the Aperture, it was discovered that interstellar civilization, for all its unlimited antiquity, has done essentially no better in this respect than human science.

	Although they make the same empirical predictions, the unified theories differ somewhat in their philosophical presuppositions and implications.  It is also too difficult to explicitly solve the equations for systems beyond a rather modest limit of complexity.  And as for nonlinear systems, almost all of them, even the simple and solvable ones, exhibit chaotic behavior.  All three of these fundamental limitations of cognition may be considered aspects of computational irreducibility.  Every civilization known to have achieved unified theory has thereafter and therefore undergone a fundamental change of direction, away from fundamental physics and towards either computational simulation or the cultivation of intuitive traditions or, more commonly, a synthesis of simulation and intuition.

	The unified theories have similar consequences for the philosophy of mind.  The most probable interpretation is that self-consciousness requires an actual infinite regress of awareness, which is possible as a fractal structure in the nonlocal view but impossible, due to phase changes enforced by Planck's constant, in the local view.  Reflective self-consciousness, then, is generally held to consist of nonlocal self-awareness acting both into and by means of local reflection.  Interstellar civilization, insofar as its members share this particular structure of reflective self-consciousness, tends to share this model of consciousness.  Its major consequence is that no computer, i.e. no universal Turing machine, can truly be self-conscious.  This has not prevented the construction of truly conscious artificial persons; but they are not Turing machines.  

	Another consequence is that telepathy is impossible, not to mention the transference of consciousness from one body to another, though sensory interfaces so complete and intimate as to create a convincing simulacrum of these impossibilities may certainly be contrived.  It is also posssible to transfer or instill recorded or artificial memories.

	Unified theories overcome the temporal and causal paradoxes and singularities of relativity theory, now called the local view, by an extension of the notion of phase cancellations in the Schroedinger wave equation, now called the nonlocal view.  As general relativity predicted, it is possible to travel both forward and backward in time.  The causal loops or contradictions that might seem to follow from this cancel each other out in the nonlocal view, and therefore do not actually arise.

	Unified theory made it possible for engineers to construct the gravitron, for generating gravitational potentials by means of electromagnetic energy.  The gravitron, in turn, made it possible to open up artificial Kerr horizons, or gates, for instantaneous travel in space and time.  The cost of such travel is little more than the difference in gravitational potential between the starting point and the end point.  (Of course, this represents a large amount of energy indeed! --- in time travel, somewhat exceeding the mass being transported).

	It is possible to open a gate anywhere, even on the surface of a planet, because the gravitron itself supplies a means for creating the absolutely tide-free gravitational field required.  But actual passage through the gate is necessarily ballistic --- either the gate itself or the vessel must move --- so it is not possible to calculate one's vector of arrival with absolute precision.  Therefore, a vessel may leave the surface of a planet with confidence, but can only arrive safely by making a series of closer and closer jumps in the relatively unobstructed field of space.  Even in space there is a small but definite risk of emerging on a collision course with, or even into the space occupied by, another object, which may well destroy both in a violent explosion.  This is in fact the major risk of space travel.

	The gravitronic gate makes possible not only faster-than-light and temporal travel, but also the conversion of mass into energy with at least almost 50 percent efficiency.  This is fortunate, for otherwise time travel would be prohibitively expensive, and even mere interstellar travel would have to be reserved for government emergencies.

	The final consequence of gravitronics is the duplication of arbitrary objects, including persons, by sending them just slightly backwards in time, although of course at the cost of somewhat greater mass than the original.  This kind of work is permitted only in interstellar space, where the accidental failure of the process will not destroy any planets.

\begin{center}\rule[3pt]{2in}{0.5pt}\end{center}

	The first historical consequence of the gate was human contact with interstellar civilization --- the Aperture proper.  The term applies both to the moment of first contact, and to interstellar civilization itself, which considers itself to be always an opening upon another aspect of itself.  

	Efforts in the late twentieth century to detect or contact interstellar civilization by radio had necessarily failed, because interstellar civilization is quite sparsely distributed and uses radio, if at all, only for local communications.  As for gravitronics, it seldom creates cosmic fireworks; and when it does, they resemble the natural ones.

	Before describing interstellar civilization, it is well to consider other purely human fundamental technologies.

\begin{center}\rule[3pt]{2in}{0.5pt}\end{center}

	The other fundamental achievements of human technology, which were begun but not perfected before the Aperture, are genetic engineering, nanotechnology, and cyberspace.

	Genetic engineering is now complete in the sense that the human genome has been completely transcribed, and any mutation in it can be simulated.  Most somatic and genetic diseases have been eliminated.  Those that remain do so because their genes also perform some vital function, and no substitute for them has yet been contrived.  Many of the genes responsible for human aging and death have been eliminated or redesigned.  The baseline human life span is now in the neighborhood of 800 years.  After that time, the effects of the remaining ambiguous genes, the accumulation of ineradicable errors in DNA replication, and the memory constraints of the human brain enforce a gradual senescence, which can go on for centuries.

	The general capabilities of humanity have been elevated, so that what used to be the second standard deviation and a half of endowment has become the median.

	Some hold that as a result of these many changes, the biological substratum of Homo sapiens no longer merits the term ``human,'' but should be called ``transhuman'' or ``neohuman.''

	Genetic engineering has also been applied to plants, animals, and the construction of artificial ecologies for spacecraft and orbital dwellings and cities.

	A number of existing or designed creatures, including dogs, cats, apes, horses, even some larger birds, have been endowed with intelligence, speech, and hands.  ``Humanity'' has thus become a cultural rather than a strictly biological term.

\begin{center}\rule[3pt]{2in}{0.5pt}\end{center}

	Nanotechnology has achieved the construction of quantum von Neumann machines.  These are universal Turing machines with universal constructors, and thus capable of self-reproduction after the manner of living things, built on a molecular scale, whose operation is based upon quantum mechanical principles and therefore is as efficient as physics permits.

	Most of the vessels, instruments, and weapons of the Aperture are nanomachines.  Artificially intelligent machines are also of this type.

	In the comparison of nanomachines with living things, the components of living things are amino acids and ribonucleic acids, whereas certain vital components of nanomachines are crystals that must be of slightly greater size.  Thus living things tend to be more compact or complex, but slow and subject to heat and noise, whereas nanomachines tend to be simpler but somewhat larger, and much faster and more efficient.  

	When it comes to thinking, the organization of biological brains, evolved over geological epochs, has a weight of design that tends to compensate for less subtle programs running on much faster hardware.  The non-Turing machine aspects of consciousness favor neither biology nor electronics.  This is true in human civilization.  

	In the Aperture, however, nanomachinery is billions of years old and has fully realized its inherent computational advantages (not the least of which is a longer life span).  There are artificial intelligences in the Aperture whose intelligence is far vaster than human.

\begin{center}\rule[3pt]{2in}{0.5pt}\end{center}

	Cyberspace is a universally distributed, switched-access network of computers and communications media, utilizing a common storage format and a virtual reality interface.  The physical interface is generally nanotechnology worn upon or within the body.  It provides a sensorium that either overlays, or tiles (e.g., a view out the back of one's head), the natural one.

 	Cyberspace has completely replaced the previous publishing houses, recording industry, telephone system, broadcast networks, and computers of humanity with a single medium of communications, transmission, and storage.

	Cyperspace both understands human speech and can respond in kind.  There are also many other perceptual interfaces for cyberspace, but the most commonly experienced are those for the telecommunications services, the librarian, and the simulation utilities.  

	The telecommunications services can produce a simulacrum of each party in a conversation that appears to occupy the real space of each of the other parties, or all the parties may meet as telepresences in a virtual space.

	The simulation utilities do what their name portends, and tend to be enclosing spaces reached by doorways from the librarian.

	The librarian, naturally enough, occupies what appears to be a very large, beautiful, and yet comfortable library, which contains many other virtual spaces.  This space is also used for telecommunications and thus can be considered the central space of cyberspace.

	Almost all machinery in the Aperture is operated by means of a cyberspace interface, frequently in the form of a command language.

\begin{center}\rule[3pt]{2in}{0.5pt}\end{center}

	Humanity had begun to develop these critical technologies prior to the Aperture, but upon joining interstellar civilization found them in their mature, and quite overwhelming, forms.  

	If the fundamental limit to technical cognition is computational 
irreducibility, consider what billions of years of practical experience must 
add to an engineered life form, nanomachine, or complex algorithm.  An analogy: 
 two schools of the dance.  In the one, each generation of dancers is active 
for fifteen years, then dies after laboriously passing on its teaching to the 
next generation.  In the other, each generation of dancers passes its skills in 
toto and instantaneously to the next, which can then occupy itself with 
learning an entirely new style of dance without losing any skill of the others; 
furthermore in this school, each generation of dancers is active for seven 
hundred years.  It's obvious which school will dance best, at least from the 
technical point of view.

	Now consider that what applies to these dancers can apply to an intelligent blade, a spacecraft navigation system, a couch for reading and sleeping.

	With regard to any definite tasks, there is an irreducibly optimal  
algorithm, which cannot be deduced from first principles,but is inevitably 
discovered after sufficient experience.  The Aperture has in fact discovered 
the optimal algorithms for most humanly definite tasks.  Humanity, believing 
itself with good reason to have achieved the fundamental limits of technology, 
found itself in contact with a civilization whose products, both technically 
and artistically, were demonstrably and overwhelmingly superior.

	It took some time for humanity to accept the weight of sheer experience; to realize that the acceleration in technology between the twelfth and twenty-first centuries was a catastrophic collapse from one stable regime to another.

\begin{center}\rule[3pt]{2in}{0.5pt}\end{center}

	Like the system of interconnecting universes that it explores, 
interstellar civilization has no known beginning in time or circumference in 
space.  Those who devote sufficient effort have always been able to find an 
earlier ancestor or a further outpost.  However, humanity knows that the 
Aperture extends at least five hundred billion years into the past (at least 30 
times older than our local universe) and occupies millions of universes, or 
more.

	Equally however, not only the local universe but the entire system of universes is on the whole expanding, and that faster than the Aperture; even though it, like recently joined humanity, is in the throes of a permanently exponential population explosion.

	As a result, although the Apeture is a network of unthinkable population and density, it is located in physical space with almost complete sparsity.

	This sparsity is relative; some universes are ancient and crowded, but then these dense universes are themselves widely separated.  In short the distribution of the Aperture is a curdled fractal, and unpopulated wilderness is far from almost no one.

	The shock of the Aperture to humanity was double:  on the one hand the billions of years of culture; on the other hand the permanent and inexhaustible frontier.

	As for the politics of the Aperture, they are somewhat incomprehensible, but in their humanly relevant sections guarantee freedom of information, of travel, of trade, and of residence throughout the Aperture; prohibit war, murder, slavery, and ``unlawful use'' (i.e. duplication of a person without their permission, rape, slander, etc.).  The machinery of law enforcement is spread, unfortunately, as thinly as the Aperture itself.  

	In centers of civilization, one is quite safe from the grosser crimes, but subject to levels of manipulation and persuasion that most human beings probably cannot begin to suspect.  In the wilderness, one remains oneself, but had best go armed.

\begin{center}\rule[3pt]{2in}{0.5pt}\end{center}

	The development of late technological capitalism was an explosive crisis with a radical denoument.  Between the fall of Communism in 1991 to the Aperture itself in 2048, only 57 years obtain.  Within those years humanity mastered genetic engineering, conceived the fundamentals of nanotechnology, constructed the initial spine of the human cyberspace network, put forward several of its own unified theories, and developed the gravitron.  At the end of that period, anyone who wished to could obtain a gravitronic starship built of self-reproducing and self-maintaining nanotechnology, and containing the essential core of the human library in toto; this vessel could construct copies not only of itself but of any other object whose design was stored in its library.
	
\end{document} 
