\documentclass{beamer}

\mode<presentation>
{
    %\usetheme{Warsaw}
    \definecolor{links}{HTML}{2A1B81}
    \hypersetup{colorlinks,linkcolor=,urlcolor=links}
    \usetheme{Frankfurt}
    %\usecolortheme{seagull}
    % or ...
    \setbeamercovered{transparent}
    % or whatever (possibly just delete it)
}

\usepackage[english]{babel}
\usepackage[utf8]{inputenc}
\usepackage{times}
\usepackage[T1]{fontenc}
\usepackage{fancyvrb}
\usepackage{listings}
\usepackage{graphicx}
\usepackage{attachfile}
\usepackage{ifthen}

\newboolean{localPieces} %Declaration, defaults to false
\setboolean{localPieces}{false} %Assignment

\title{Advanced Csound}

\author[Gogins] % (optional, use only with lots of authors)
{Michael Gogins \\ \url{http://michaelgogins.tumblr.com} }

\institute[Irreducible Productions] % (optional, but mostly needed)
{
    Irreducible Productions\\
    New York
}
\date[NYCEMF 2022] 
{NYCEMF 2022}

\subject{Computer Music}
\expandafter\def\expandafter\insertshorttitle\expandafter{%
    \insertshorttitle\hfill%
    \insertframenumber\,/\,\inserttotalframenumber}
\begin{document}
    \lstset{basicstyle=\ttfamily\tiny,commentstyle=\ttfamily\tiny,tabsize=2,breaklines,fontadjust=true,keepspaces=false,fancyvrb=true,showstringspaces=false,moredelim=[is][\textbf]{\\emph\{}{\}}}

\frame{\titlepage}

    \begin{frame}{Outline I}
        \begin{itemize}
            \item These are slides, with links to resources, for a workshop on the
            \emph{advanced} use of Csound.
            \item The intended audience is anyone using Csound to \textit{actually make
                music}, whether they are a raw beginner or an experienced user.           
             \item \emph{There are no compromises here}, but I will try to keep things
            as simple as possible.
            
            \item I provide pre-written examples and exercises, so you won't have to debug much 
            during the workshop.
        \end{itemize}
    \end{frame}
    
    \begin{frame}{Outline II}
        \begin{itemize}
           \item First I review "best practices" using \textit{built-in features of
            the current release of Csound}.
            \item Then I review some extensions to Csound: 
                \begin{itemize}
                    \item Development environments.
                    \item Plugin opcodes.
                    \item Hosting VST plugins.
                    \item Using other languages in Csound.
                \end{itemize}
        \end{itemize}
    \end{frame}
    
    \begin{frame}{Basic Pre-Requisites}
    	Install with brew on macOS, apt on Linux, vcpkg on Windows, or 
	google for downloads.
        \begin{itemize}
            \item Install Csound from
            \url{https://csound.com/download.html}. For the most current release on
            Linux, build from sources
            (\url{https://github.com/csound/csound/blob/develop/BUILD.md}).
            \item Install the Risset package manager for Csound plugins from 
            \url{https://github.com/csound-plugins/risset}. 
            \item Install the CsoundQt Csound editor from
            \url{http://csoundqt.github.io/pages/install.html}.
            \item Install the Audacity soundfile editor from
            \url{https://www.audacityteam.org/download/}.
        \end{itemize}
    \end{frame}
    
    \begin{frame}{Optional Pre-Requisites}
        Some of these have their own additional pre-requisites.
        \begin{itemize}
            \item ABX comparator from \url{https://github.com/gogins/python-abx}.
            \item Python 3.9 from \url{https://www.python.org/downloads/}.
            \item Faust DSP language from
            \url{https://github.com/grame-cncm/faust/releases}.
        \end{itemize}
    \end{frame}

    \begin{frame}{Agenda}
        \tableofcontents
        % You might wish to add the option [pausesections]
    \end{frame}

    \section{Hearing Electroacoustically}
       \begin{frame}{Hearing}
    	\begin{itemize}
	  \item Before anything else one should know that one is hearing \textit{objectively}.
	  \item One should establish the limits of one's hearing because the computer 
	  will exceed them.
	  \item One should establish the least noticeable differences for important features of sound. 
	\end{itemize}
    \end{frame}
\begin{frame}
	\begin{example}{Hearing}
		\begin{itemize}
			\item Ideally, get a hearing test from an audiologist.
			\item Frequency sweep in Audacity.
			\item Dynamic ranges in Audacity.
			\item Wave table sizes in Audacity.
			\item Time/frequency uncertainty relation.
			\item Digital artifacts in Audacity: clicking, aliasing, smearing.
		\end{itemize}
 	\end{example}
 \end{frame}

    \section{Introduction to Csound}
    \begin{frame}{Csound}
        \begin{itemize}
            \item Csound is a \textit{programmable} software sound synthesizer.
            \item Csound dates from 1985, yet is still being developed (with almost
            complete backward compatibility).
            \item All SWSS have digital oscillators, filters, envelopes, etc...
            \textit{because} it is older, Csound has \textit{more} unit generators.
            \item Csound can be extended with plugins, some of which bring other
            programming languages into Csound. 
            \item Csound has a flexible API, and can be embedded in other
            programming languages.
            \item Csound began with "off-line rendering," but now does real-time
            audio and interactive performance very well.
            \item Csound runs on the desktop, on mobile devices, and in Web
            browsers. Here, we focus on the desktop. 
        \end{itemize}
    \end{frame}
    \begin{frame}{An "All-In" Piece}
        \begin{itemize}
            \item Here I will perform a real-time, interactive, algorithmic piece
            that uses much of what we will work with today.
            \item This piece is one .html file that incorporates  HTML, JavaScript, and Csound.
            \item But it is still just one .html file that is edited in a code editor
            like any other code, and that runs in one process like any other program.
        \end{itemize}
    \end{frame}





\end{document}
