\documentclass[12pt,letterpaper,onecolumn]{scrartcl}
\usepackage{tocloft}
\usepackage{stdclsdv}
\usepackage{comment}
\usepackage{vmargin}
\usepackage{t1enc}
\usepackage{fancyvrb}
\usepackage{url}
\usepackage{calc}
\usepackage{array}
\usepackage{scrpage2}
\usepackage{graphicx}
\usepackage{color}
\usepackage{listings}
\usepackage[latin1]{inputenc}
\usepackage[english]{babel}
\usepackage{supertabular}
\usepackage[pdftex,
            pagebackref=true,
            colorlinks=true,
            linkcolor=blue,
            pdfpagelabels,
            pdfstartpage=3
           ]{hyperref}
\pagestyle{scrheadings}
\makeindex
\definecolor{LstColor}{cmyk}{0.1,0.1,0,0.025} 
\setcounter{tocdepth}{2}
\begin{document}
\lstset{language=c++,basicstyle=\ttfamily\scriptsize,commentstyle=\ttfamily\small,tabsize=8,breaklines,backgroundcolor=\color{LstColor},fontadjust=true,keepspaces=false,fancyvrb=true,showstringspaces=false,moredelim=[is][\textbf]{\\emph\{}{\}}}.
\title{A Module System for Csound}
\author{Michael Gogins \\ \texttt{michael.gogins@gmail.com}}
\maketitle

\section{Introduction}
\label{sec:Introduction}

This document presents a system for writing Csound orchestras that I have 
developed over a number of years to support my work as a composer. This 
document comes with a complete Csound piece that illustrates all features of 
the system, \texttt{module\_system\_example.csd}. The main purposes of the 
system are:

\begin{enumerate}
	\item To enable the maximum possible reuse of Csound instrument and 
	user-defined opcode definitions. Any instrument or opcode created in one 
piece can immediately be used without modification in any other piece whose 
score uses the same pfield conventions. Any instrument can be used without 
modification for standard Csound score input, for MIDI file input, for MIDI 
real-time input, or for real-time API input from a host application, whether 
that input consist of note on/note off pairs or whether that input consist of 
notes with predefined durations.
    \item To enable the use of an arbitrary number of real-time control 
    channels for instrument definitions.
	\item To enable moving modules between various user interface and external 
	control systems, such as CsoundQt, the Csound for Android app, or 
csound.node.
	\item To implement a comprehensive, flexible, and high-quality system of 
	spatialization, including Ambisonic periphony, arbitrary speaker 
arrangements, spatial reverberation including diffuse and specular early 
reflections from various surfaces, and distance cues including attenuation, 
filtering, Doppler effects, and head-related transfer functions. The 
spatialization system is adapted from Jan Jacob Hofmann's excellent work. 
	
\end{enumerate}

\noindent These goals are achieved by strictly following, insofar as the 
Csound 
orchestra language permits, that fundamental principle of software engineering 
known as \emph{encapsulation} or \emph{data hiding}. 

A Csound instrument definition or user-defined opcode definition is called a 
\emph{module}. These modules are ``black boxes'' that expose only the following 
standard interfaces for interactions with the rest of Csound. As a result, 
modules may be defined in include files (\texttt{.inc} files) and 
\texttt{\#include}d as required by the Csound orchestra. Here is an example of 
a module (with comments added):

\begin{lstlisting}
; Global control variables for STKBowed, named to prevent collisions.
; The init values are usable defaults that pieces, hosts, and applications may 
; choose to override.

; Controllers range from 0 through 127.

gk_STKBowed_bow_pressure init 110
gk_STKBowed_bow_position init 20
gk_STKBowed_vibrato_level init 2.8 
gk_STKBowed_vibrato_frequency init 50.2

; Level ranges from -40 to +40 dB, 0 is nominal.

gk_STKBowed_level init 0

; The actual instrument definition.

instr STKBowed

; Author: Michael Gogins

; More or less standard "prolog" for handling standard pfields.

; Setting negative p3 to a large value enables normal arithmetic based on 
; duration so that "note-on/note-off" performance works just like 
; "score-driven" performance.

if p3 == -1 then
p3 = 1000000
endif
i_instrument = ; 
i_time = p2
i_duration = p3
i_midi_key = p4
i_midi_velocity = p5
i_depth = p6
i_pan = p7
i_height = p8
i_phase = p9
i_frequency = cpsmidinn(i_midi_key)

; Adjust the following value until "overall amps" at the end of solo 
; performance with midi velocity 80 is about -6 dB.

i_overall_amps = 70
i_normalization = ampdb(-i_overall_amps) / 2
i_amplitude = ampdb(i_midi_velocity) * i_normalization
k_gain = ampdb(gk_STKBowed_level)

; The body of the instrument definition.

asignal STKBowed i_frequency, 1.0, 1, gk_STKBowed_vibrato_level, 2, gk_STKBowed_bow_pressure, 4, gk_STKBowed_bow_position, 11, gk_STKBowed_vibrato_frequency

; More or less standard "epilog" for sending audio out.

; To get note-offs to work without clicking, there must be a releasing 
; envelope generator with a suitable de-clicking envelope. If the note is too 
; short for the attack and release, p3 is extended.

iattack = .005
isustain = p3
irelease = 0.1
p3 = iattack + isustain + irelease
a_declick linsegr 0, iattack, 1, isustain, 1, irelease, 0
aleft, aright pan2 asignal * i_amplitude * a_declick * k_gain, i_pan
outleta "outleft", aleft
outleta "outright", aright
prints "STKBowed       i %9.4f t %9.4f d %9.4f k %9.4f v %9.4f p %9.4f #%3d\n", 
p1, p2, p3, p4, p5, p7, active(p1)
endin

\end{lstlisting}

\begin{enumerate}
	\item Standard pfields. Please note, in comparison with my past convention 
	for pfields, I have changed slightly the meaning of the spatial pfields to 
more closely follow the conventions of Csound's spatialization opcodes. Also 
note that the exact same set of pfields and instrument definitions may be used 
either for score-driven performance, or for real-time, MIDI-driven performance, 
if the \texttt{-\--midi-key=4 -\--midi-velocity=5} command-line options are 
used. The first three pfields are mandatory except for \texttt{alwayson} 
modules, which do not require any pfields.
        \begin{description}
			\item[p1]	Instrument number or MIDI channel number (may be a 
			fraction; the fractional part can be used as a globally unique 
identifier for the note; a negative value means "note-off" for instrument 
instances running with the same absolute value of p1).
			\item[p2] 	Start time in beats (by default, a beat is 1 second).
			\item[p3] 	Duration in beats (a negative value indicates 
			indefinite duration).
			\item[p4] 	Pitch as MIDI key number (middle C is 60, may be a 
			fraction).
			\item[p5] 	Loudness as MIDI velocity number (\emph{forte} is 80, 
			may be a fraction).
			\item[p6] 	Cartesian $x$ coordinate in meters, running from in 
			front of the listener to behind the listener.
			\item[p7] 	Cartesian $y$ coordinate in meters, running from the 
			left of the listener to the right of the listener. This is the 
same as stereo pan.
			\item[p8] 	Cartesian $z$ coordinate in meters, running from below 
			the listener to above the listener.
			\item[p9] 	Phase of the audio signal in radians (in case the 
			instrument instance is, e.g., synthesizing a single grain of sound; 
this can be useful for phase-synchronous overlapped granular synthesis).
        \end{description}
	\item Standard outlet and inlet ports. All modules must send or receive 
	audio signals only via the signal flow graph opcodes. The instruments 
	defined in this repository suppport both plain stereo output and 
	spatialized output using the following logic. It is assumed that the 
	instrument's output signal is in \texttt{a\_asignal} at unity gain.
	
\begin{lstlisting}
i_attack = .002
i_sustain = p3
i_release = 0.01
p3 = i_attack + i_sustain + i_release
a_declicking linsegr 0, i_attack, 1, i_sustain, 1, i_release, 0
a_signal = a_signal * i_amplitude * a_declicking * k_gain
#ifdef USE_SPATIALIZATION
a_spatial_reverb_send init 0
a_bsignal[] init 16
a_bsignal, a_spatial_reverb_send Spatialize a_signal, k_space_front_to_back, 
k_space_left_to_right, k_space_bottom_to_top
outletv "outbformat", a_bsignal
outleta "out", a_spatial_reverb_send
#else
a_out_left, a_out_right pan2 a_signal, k_space_left_to_right
outleta "outleft", a_out_left
outleta "outright", a_out_right
#endif
\end{lstlisting}

\noindent In other words, if the macro \texttt{USE\_SPATIALIZATION} is defined 
in the orchestra header, the instrument will send Ambisonic B-format audio to 
a 16 channel outlet named \texttt{outbformat}, along with a mono signal to 
\texttt{out} that can be used as a reverb send; if 
\texttt{USE\_SPATIALIZATION} 
is not defined, the instrument will send stereo audio to \texttt{outleft} 
and \texttt{outright}. 

This enables the same modules to be used in any sort of Csound 
orchestra and for any audio output file format or speaker rig. The B-format 
encoded audio signal can be decoded to mono, stereo, 2-dimensional panning, or 
full 3-dimensional panning using first, second, or third order decoding. 

Of course, the declaration of audio connections and ``alwayson'' instruments 
will be different for plain stereo versus Ambisonic orchestras. See 
\url{SpatializedDrone.csd} for an example of an Ambisonic piece, and comments 
in \url{Spatialize.inc} for documentation of the spatialization system.

		\item Standard control variables. The modules themselves do not define 
		or directly use input or output channels or zak variables. They use 
global control variables with the following naming convention: 
\texttt{gk\_InstrumentName\_variable\_name}; \texttt{i}-rate variables may also 
be used. These variables must be declared just above the instr or opcode 
definition and initialized there with default values. If the range is not [0,1] 
then comment to document the expected range.
		
In addition to using only these standard interfaces, all modules that use 
function tables must define their own tables within themselves using the 
\texttt{ftgenonce} opcode. Then there is no dependence of the module upon the 
external score or orchestra header.
		
Usually, a modular Csound piece will generate a score in external code or in 
the orchestra header, \texttt{\#include} a number of instrument and effects 
processing patches, connect their outlets and inlets using the signal flow 
graph \texttt{connect} opcode, and turn on effects modules with the 
\texttt{alwayson} opcode. For spatialization, the signal flow graph will 
terminate in a module that will perform Ambisonic decoding to the specified 
speaker rig and/or output soundfile.

Finally, any external controls, such as CsoundQt widgets, OSC signals, MIDI 
controllers, or whatever, will be received in another custom input module and 
mapped to the relevant global control variables.

Because control channels are compute-intensive, it is advisable to receive all 
control value changes in a global, always-on instrument, like this:

\begin{lstlisting}
; Set default values of control channels,
; this is necessary in case the user of this orchestra does not
; otherwise set these values.

chnset .95, "gk_Reverb_feedback"

instr Controllers
gk_STKBowed_level chnget "gk_STKBowed_level"
gk_STKBowed_level chnget "gk_STKBowed_bow_pressure"
gk_STKBowed_level chnget "gk_STKBowed_vibrato_level"
gk_Harpsichord_level chnget "gk_Harpsichord_level"
gk_Reverb_feedback chnget "gk_Reverb_feedback"
gk_MasterOutput_level chnget "gk_MasterOutput_level"
endin
\end{lstlisting}

\subsection{Real-Time Notes}

The exact same instrument definitions can be used for score-driven, 
MIDI-driven, or API-driven performance; and for notes with predefined 
durations, or notes that come in note-on/note-off pairs.

To enable this interoperability, bear in mind that for MIDI note-on events, 
Csound creates \texttt{i} statements with an indefinite duration. If the MIDI 
interop command-line options or opcodes are used, then the MIDI key number is 
filled in to a configurable pfield (here, p4) and the MIDI velocity number is 
filled in to another configurable pfield (here, p5). Also, it may well be 
desirable to create an application that uses the API to perform a MIDI-type 
real-time performance driven by note-on/note-off pairs. For real-time 
interoperability to work you must:

\begin{enumerate}
    \item For API-driven note-on/note-off usage only, create a fractional 
    instrument number to act as a globally unique identifier for each note on 
\texttt{i} statement.
    \item For API-driven note-on/note-off usage only, to turn the note off, 
    create a new \texttt{i} statement that is identical to the note-on 
\texttt{i} statement except that the sign of p1 is negative.
    \item In the instrument definition, detect if p3 is -1 and, if so, reset 
    it to a large value such as 1000000. When the note-off event occurs, 
Csound will end the instrument instance but continue the release section of 
any releasing envelope generators. Using a large positive p3 for indefinitely 
held notes ensures that envelopes and other features that do arithmetic 
expecting a positive p3 will continue to work properly.
\end{enumerate}
		
\end{enumerate}
\section{An Example}
\label{sec:AnExample}

\begin{lstlisting}
<CsoundSynthesizer>
<CsLicense>

Author: Michael Gogins
License: Lesser GNU General Public License version 2

This file demonstrates a module system for Csound.

</CsLicense>
<CsOptions>
-d -m163 -odac
</CsOptions>
<CsInstruments>

; Initialize the global variables.

sr = 44100
ksmps = 100
nchnls = 2
0dbfs = 1

; Connect up instruments and effects to create a signal flow graph.

connect "STKBowed",     "outleft",      "Reverb",     	"inleft"
connect "STKBowed",     "outright",     "Reverb",     	"inright"

connect "Harpsichord",  "outleft",     "Reverb",     	"inleft"
connect "Harpsichord",  "outright",    "Reverb",     	"inright"

connect "Reverb", 		"outleft",     "Compressor",    "inleft"
connect "Reverb", 		"outright",    "Compressor",    "inright"

connect "Compressor",   "outleft",     "MasterOutput",  "inleft"
connect "Compressor",   "outright",    "MasterOutput",  "inright"

; Turn on the "effect" units in the signal flow graph.

alwayson "Controllers"
alwayson "Reverb"
alwayson "Compressor"
alwayson "MasterOutput"

#include "Harpsichord.inc"
#include "STKBowed.inc"
#include "Reverb.inc"
#include "Compressor.inc"
#include "MasterOutput.inc"

; Set default values of control channels,
; this is necessary in case the user of this orchestra does not
; otherwise set these values.

chnset .95, "gk_Reverb_feedback"

instr Controllers
gk_STKBowed_level chnget "gk_STKBowed_level"
gk_STKBowed_level chnget "gk_STKBowed_bow_pressure"
gk_STKBowed_level chnget "gk_STKBowed_vibrato_level"
gk_Harpsichord_level chnget "gk_Harpsichord_level"
gk_Reverb_feedback chnget "gk_Reverb_feedback"
gk_MasterOutput_level chnget "gk_MasterOutput_level"
endin

</CsInstruments>
<CsScore>

; Not necessary to activate "effects" or create f-tables in the score!
; Overlapping notes to create new instances of instruments.

i 1 1 5 60 85 0 .25
i 1 2 5 64 80 0 .25
i 2 3 5 67 75 0 .75
i 2 4 5 71 70 0 .75
e 10
</CsScore>
</CsoundSynthesizer>
\end{lstlisting}

\end{document} 

