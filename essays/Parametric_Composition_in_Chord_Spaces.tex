\documentclass[english,11pt,letterpaper,onecolumn]{scrartcl}

%\usepackage[utf8]{inputenc}
\usepackage{babel}
\usepackage{amsmath}
\usepackage{amsfonts}
\usepackage{mathptmx}
\usepackage{enumitem}
% Extra leading.
\renewcommand{\baselinestretch}{1.125}
\usepackage{tocloft}
% \usepackage{fancyhdr}
\usepackage{scrlayer-scrpage}
\usepackage{ifthen}
\usepackage{keyval}
\usepackage{geometry}
\usepackage{url}
\usepackage{calc}
\usepackage{array}
\usepackage{graphicx}
\usepackage{color}
\usepackage{listings}
\usepackage{supertabular}
%\usepackage{scrpage2}
\usepackage[pdftex,
           colorlinks=true,
            linkcolor=blue,
            pdfpagelabels,
            pdfstartpage=3
           ]{hyperref}
% \usepackage{poemscol}
% \global\verselinenumbersfalse
\makeindex
\definecolor{LstColor}{cmyk}{0.1,0.1,0,0.025} 
\setcounter{tocdepth}{9}
\newcommand\floor[1]{\lfloor#1\rfloor}
\newcommand\ceil[1]{\lceil#1\rceil}

\usepackage[
backend=biber,
style=numeric,
sorting=ynt,
hyperref=true,
backref=true
]{biblatex}
 
\addbibresource{gogins.bib}
\begin{document}

\title{Parametric Composition}
\author{Michael Gogins \\ \texttt{michael.gogins@gmail.com}}
\maketitle
%\pagestyle{scrheadings}

%\lohead{Parametric Composition}

\section{Introduction}

We define a \textit{score space} as a space in which a set of points represents 
a piece of music, e.g. the grand staff, a piano roll, some more abstract 
space, or even a representation of sound such as a sonogram. We define a 
\textit{score generator} as a computer program that generates a musical score 
in some score space. We define a \textit{parametric score generator} as one 
whose behavior is completely specified by adjusting numerical parameters. Then 
\textit{parametric composition} is the art of composing music by exploring the 
parameter space of a parametric score generator. Such exploration can be 
performed by literally zooming around in a colored map of the parameter space, 
by interpolating between two parameter points in that space, or by evolving 
parameters using the genetic algorithm. 

Parametric composition has previously been investigated with respect to 
the generation of pieces in spaces that directly represent either sound, 
e.g. in the form of a grid of sound grains (Gabor transform) \cite{obsessed}, 
or scores, e.g. in the form of a grid of notes (piano roll) \cite{ifsmusic}. 
Here, pieces are generated in a score space constructed from the basic 
symmetries of chord space identified by Callender, 
Quinn and Tymoczko \cite{callender:346}, along with revoicings and 
rearrangements. The dimensions of this score space are: set class 
(the most basic form of a chord) $P$ (i.e. Callender et al.'s OPTIC), 
inversion $I$, transposition $T$, octavewise revoicings $v$, and 
rearrangements of voices $a$. In $PITva$ space, a piece of note-based music may 
be considered a succession of more or less fleeting chord points or, in other 
words, as the graph of a vector-valued function of time in the score space, 
$$(P, I, T, v, a) = r(t).$$ The notes and chords do \textit{not} need to be 
confined to 12-tone 
equal temperament.

The score generators used here are \textit{iterated function systems} (IFSs) 
\cite{barnsley1985iterated, 10.2307/24893080, fractalseverywhere} composed of 
\textit{fractels} (fractal elements) such that, by construction, the fixed 
point or attractor of the IFS is the graph of the vector-valued function $r$ 
\cite{2016arXiv161001369B}. An IFS that computes the graph of a function in 
this way is called a \textit{fractal interpolation} or \textit{fractal 
approximation} (FA) \cite{Barnsley1986, fractalseverywhere, 
navascues2014fractal} and the function for that graph is called simply a 
\textit{fractal function}. It is interesting that all continuous functions are 
fractal functions \cite{2016arXiv161001369B}. The advantage of using fractal 
approximations of the graph of $r$ is that \textit{any} score can be so 
generated and \textit{every} FA of this type generates a graph of $r$.

A FA may be completely specified as a set of fractels each of which, 
in turn, may be computed by a Read-Bajraktarevi\'{c} operator. These can be 
completely specified as a set of numerical parameters. Thus, any piece of 
note-based music may be approximated as closely as desired by a fixed size set 
of numerical parameters, a list of the matrix representations of the 
Read-Bajraktarevi\'{c} operators.

Although there might be dozens or hundreds of numbers in such FA parameters, 
each parameter set can effectively be represented as a single real or complex 
number by using a recursive indexing scheme such as a Hilbert index 
\cite{hamilton2006compact}. We call this number the \textit{effective 
parameter} of the FA because the \textit{actual parameters}, i.e.\ the 
complete representation of the Read-Bajraktarevi\'{c} operators, can be 
recovered by decoding the index. 

Then, using the effective parameters, it is simple to compute a parametric map 
of all pieces within a given range of FAs, or to interpolate between two 
pieces by interpolating between their effective parameters. It is also, of 
course, possible to evolve pieces using the genetic algorithm on sets of 
actual parameters.

The remainder of this paper develops the mathematical background in 
somewhat more detail; discusses the implementation of fractels, fractal 
approximations, Hilbert indices, and the genetic algorithm for FA parameters 
in the Silencio library for algorithmic composition in JavaScript; and 
finishes with examples of each of the three methods of algorithmic 
composition, implemented in Silencio and Csound.

\section{Mathematical Background}

This section is not intended as a complete, self-contained exposition but 
rather as providing entry points for readers who have some exposure to 
mathematical music theory or fractal geometry. I have tried to supply 
references not only to original publications of ideas, but also to recent 
reviews or summaries.

\subsection{Chord Spaces}

A chord space is simply a space in which each voice of a chord is represented 
by a different dimension. A single voice is represented by a point on a line; 
an interval, by a point on a plane; a triad, by a point in a 3-dimensional 
space; and so on. The concept of chord space has many uses. For example, 
Tymoczko \cite{tymoczko2006geometry, tymoczko2011geometry} showed that 
perceptually smoother voice-leadings between chords are represented by shorter 
paths through chord space. Here the octave is always 12, so that pitch-classes 
are always integers and compatible with MIDI.

Even unschooled musicians are aware that pitches transposed to different 
octaves are somehow the same. This is expressed in music theory by saying that 
pitch-classes, e.g. C or F\#, are pitches under \textit{octave equivalence}. 
Octave equivalence transforms the line of pitch into a circle, an 
\textit{orbifold} or \textit{quotient space} $\mathbb{R}/12$, by 
gluing the lowest pitch under octave equivalence to the same pitch an octave 
higher. There are other implicitly familiar equivalence classes in music: 
\textit{permutational equivalence} (a triad is somehow the same whether the 
notes are assigned to violin, trumpet, and flute or to flute, trumpet, and 
violin), \textit{cardinality equivalence} (a chord is somehow the same if 
voices are doubled in different octaves), and so on. A little more theoretical 
sophistication reveals \textit{inversional equivalence}: a major triad is 
somehow the same as a minor triad. Callender, Quinn, and Tymoczko 
\cite{callender:346} concisely analyzed these equivalence classes, or 
symmetries, as different quotients of chord space.

Here, we do not use chord space for composing; rather, we construct a new 
space in which each of the irreducible symmetries of chord space identified by 
Callender, Quinn, and Tymoczko is a separate dimension, on which operations 
are indexed. These are:

\begin{enumerate}
 \item $P$ (Callender, Quinn, and Tymoczko's OPTI space) -- Set-type, the most 
abstract conception of a chord; all major and minor triads are equivalent. The 
number of elements in P depends upon the maximum number of voices and the 
number of divisions of the octave.
 \item $I$ -- Inversion.
 \item $T$ -- Transposition.
\end{enumerate}

\noindent To these we add:

\begin{enumerate}[resume]
 \item $V$ -- octavewise revoicing (within some specified range) of the 
$PIT$ chord, through all permutations of the possible voicings.
 \item $a$ -- Voicewise re-arrangement of voices or instrument, through all
permutations of the possible arrangements.
\end{enumerate}

\noindent The purpose of this odd $PITva$ space is to permit composing by 
moving a single point through the space and thus controlling at once 
set-class, inversion, transposition, voicing, and arrangement; another way of 
saying this is that any piece of note-based music can be represented by the 
graph of a vector-valued function of time, $$(P, I, T, V, a) = 
r(t).$$

A further purpose is that, by assigning each symmetry to an orthogonal 
dimension of the $PITva$ space, the IFSs to be developed for composing can 
more easily be adapted to control the dimensions independently, e.g.\ by 
generating arpeggiations without affecting chord type.

\subsection{Fractal Approximation}

A fractal is simply a set of points that fills some definite fraction of its 
space. A line on the plane occupies no space. A snowflake curve iterates 
bending segments of the curve to infinity, and thus comes to occupy some 
fraction of the plane \cite{Mandelbrot:1982:FGN}.

One basic way to generate fractals is with the multiple copy reduction 
machine (MCRM). Imagine a copier with more than one lens, each equipped with 
adjustments (sliding, shrinking, or rotating the image of the original). Make 
a copy, replace the original with the copy, and repeat to infinity. Just as 
with the snowflake curve, the final copy comes to occupy a definite fraction 
of the picture plane and thus is a fractal \cite{chaosandfractals}. 

Mathematically, the MCRM is an iterated function system (IFS). Each lens 
represents an affine transformation of the plane. Each lens, chosen at random 
or in turns, transforms the original set, and the transformations are summed 
to replace the original; if the transformations are on the whole contractive, 
the Banach Fixed Point Theorem proves that iterating this process to infinity 
brings the original to a fixed point, the attractor of the IFS 
\cite{chaosandfractals, barnsley1985iterated, 10.2307/24893080, 
fractalseverywhere}. \textit{Any} original will end up producing the 
\textit{same} attractor. 

Furthermore, Barnsley's Collage Theorem \cite{barnsley:1986:solution} proves 
that any original set can be approximated, as closely as desired, by some IFS. 
This is done by arranging the affine transformations to cover the original set 
with transformed copies of itself, minimizing overlaps. \textit{The Collage 
Theorem is the key motivation for using IFS in algorithmic music composition}: 
IFS are simple yet extremely powerful. 

The remainder of this paper concerns how to control this power: how to 
filter out as many of the musically useless productions, which of course are 
overwhelmingly more numerous, as possible; and how to explore the 
remaining vast parameter space of IFS in a \textit{musical} way.

It would be possible to use IFS to generate sound grains on a grid; to 
generate notes on a grid; or to generate a path through a different chord 
space, e.g. use Callender, Quinn, and Tymoczko's OPC $\times$ revoicings 
$\times$ rearrangements instead of $PITva$. Again, the purpose is to filter 
out unmusical productions. Since by the Collage Theorem IFSs are universal, we 
cannot exclude such productions; but we can simply make it harder, take more 
steps, to generate then. This is not an esthetic judgement. We are 
simply trying to make the generative space more like the space of human 
musical perception, as its geometry is implied by music theory.

So, we have a piece as a sequence of points in $PITva$ space, or, 
alternatively, the graph of a function of time in that space. IFSs can 
be used to approximate graphs of functions by applying the Collage Theorem 
as follows. This applies in spaces of any dimension, i.e.\ to vector-valued as 
well as to single-valued functions:

\begin{enumerate}
 \item Define the domain of the function, an interval $[a, b]$.
 \item Identify a set of interpolation points, samples of the value of the 
function, within $[a, b]$, 
\end{enumerate}










\subsection{Hilbert Indices for Fractal Approximation Parameters}

\section{Implementation Notes}

\section{Musical Examples}

\subsection{Parametric Mapping}


\subsection{Interpolating Between Pieces}


\subsection{Using the Genetic Algorithm with Fractal Interpolation Functions}

\section{Future Directions}

It is interesting that, as Massopust notes \cite{massopust2017}, 
\begin{quote}D. Hardin proved in 2012 that every compactly supported 
refinable function is a piecewise fractal function. In particular, the unique 
compactly supported continuous function determined by the mask of a convergent 
subdivision scheme is a piecewise fractal function. \end{quote} 
This implies that the same technique of parametric 
composition based on FAs and described here, would also work for composing 
sound directly using refinable sets of sound grains, e.g. wavelets.

It also is interesting that Barnsley, Hegland, and Massopust 
\cite{2013arXiv1309.0972B} discuss how to derive fractal approximations of 
existing data sets. This implies that it is possible to automatically derive 
FA parameters for existing works of note-based music, and then to use the 
resulting parameter map or parametric interpolation to analyze the 
relationship between the works, or to compose new works intermediate in form 
between them.

% \bibliographystyle{ieeetr}
%\bibliographystyle{ieeetr}
%\bibliography{gogins}{}
\printbibliography

\end{document} 
