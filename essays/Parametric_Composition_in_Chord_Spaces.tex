\documentclass[english,11pt,letterpaper,onecolumn]{scrartcl}

%\usepackage[utf8]{inputenc}
\usepackage{babel}
\usepackage{mathptmx}
% Extra leading.
\renewcommand{\baselinestretch}{1.125}
\usepackage{tocloft}
% \usepackage{fancyhdr}
\usepackage{scrlayer-scrpage}
\usepackage{ifthen}
\usepackage{keyval}
\usepackage{geometry}
\usepackage{url}
\usepackage{calc}
\usepackage{array}
\usepackage{graphicx}
\usepackage{color}
\usepackage{listings}
\usepackage{supertabular}
%\usepackage{scrpage2}
\usepackage[pdftex,
            pagebackref=true,
            colorlinks=true,
            linkcolor=blue,
            pdfpagelabels,
            pdfstartpage=3
           ]{hyperref}
% \usepackage{poemscol}
% \global\verselinenumbersfalse
\makeindex
\definecolor{LstColor}{cmyk}{0.1,0.1,0,0.025} 
\setcounter{tocdepth}{9}
\newcommand\floor[1]{\lfloor#1\rfloor}
\newcommand\ceil[1]{\lceil#1\rceil}
\begin{document}

\title{Parametric Composition in Chord Spaces}
\author{Michael Gogins \\ \texttt{michael.gogins@gmail.com}}
\maketitle
%\pagestyle{scrheadings}

%\lohead{Parametric Composition}

\section{Introduction}

Parametric composition is the art of composing music by exploring the 
numerical parameter space of some score-generating algorithm. Such exploration 
can be performed by literally zooming around in a colored map of the parameter 
space, by interpolating between two parameter points in that space, or by 
evolving parameters using the genetic algorithm. 

Parametric composition has previously been investigated with respect to 
the generation of pieces in spaces that directly represent either sound, 
e.g. in the form of a grid of sound grains (Gabor transform) \cite{obsessed}, 
or scores, e.g. in the form of a grid of notes (piano roll) \cite{ifsmusic}. 
Here, pieces are generated in a chord space constructed from the basic 
symmetries of chord space identified by Callender, 
Quinn and Tymoczko \cite{callender346}, along with revoicings and 
rearrangements. The dimensions of this $PITVi$ space are: set class 
(the most basic form of a chord) $P$ (i.e. Callender et al.'s OPTI), 
inversion $I$, transposition $T$, octavewise revoicings $V$, and rearrangements 
of instruments $i$. In $PITVi$ space, a piece of note-based music may be 
considered a succession of more or less fleeting chord points or, in other 
words, as the graph of a vector-valued function of time, $$(P, I, T, V, i) = 
r(t).$$ The notes and chords do \textit{not} need to be confined to 12-tone 
equal temperament.

Here, the advantage of $PITVi$ is simply that \textit{shorter moves 
in chord spaces are easier for audiences to hear}. In other words, score 
generators in chord spaces are more likely to produce what musical tradition 
considers ``well-formed'' pieces.

In addition, the score generators used here are \textit{iterated function 
systems} (IFSs) \cite{barnsley1985iterated, 
10.2307/24893080, fractalseverywhere} composed of \textit{fractels} (fractal 
elements) such that, by construction, the fixed point or attractor of the IFS 
is the graph of the vector-valued function $r$ \cite{2016arXiv161001369B}. An 
IFS that computes the graph of a function in this way is called a 
\textit{fractal interpolation} or \textit{fractal approximation} (FA) 
\cite{barnsley1986, fractalseverywhere, navascues2014fractal} and the function 
for that graph is called simply a \textit{fractal function}. It is interesting 
that all continuous functions are fractal functions \cite{2016arXiv161001369B}.

A FA may be completely specified as a set of fractels each of which, 
in turn, may be computed by a Read-Bajraktarevi\'{c} operator. These can be 
completely specified as a set of numerical parameters. Thus, any piece of 
note-based music may be approximated as closely as desired by a fixed size set 
of numerical parameters, a list of the matrix representations of the 
Read-Bajraktarevi\'{c} operators.

Although there might be dozens or hundreds of numbers in such FA parameters, 
each parameter set can effectively be represented as a single real or complex 
number by using a recursive indexing scheme such as a Hilbert index 
\cite{hamilton2006compact}. We call this number the \textit{effective 
parameter} of the FA because the \textit{actual parameters}, i.e.\ the 
complete representation of the Read-Bajraktarevi\'{c} operators, can be 
recovered by decoding the index. 

Then, using the effective parameters, it is simple to compute a parametric map 
of all pieces within a given range of FAs, or to interpolate between two 
pieces by interpolating between their effective parameters. It is also, of 
course, possible to evolve pieces using the genetic algorithm on sets of 
actual parameters.

The remainder of this paper presents the mathematical background in 
somewhat more detail; discusses the implementation of fractels, fractal 
approximations, Hilbert indices, and the genetic algorithm for FA parameters 
in the Silencio library for algorithmic composition in JavaScript; and finishes 
with examples of each of the three methods of algorithmic composition, 
implemented in Silencio and Csound.

\section{Mathematical Background}

This section is not intended as a complete, self-contained exposition but 
rather as providing entry points for readers who have some exposure to 
mathematical music theory or fractal geometry. I have tried to supply 
references not only to original publications of ideas, but also to recent 
reviews or summaries.

\subsection{Chord Space}

\subsection{Fractal Interpolation Functions}

It is interesting that, as Massopust notes \cite{massopust2017} 
\begin{quote}D. Hardin proved in 2012 that every compactly supported 
refinable function is a piecewise fractal function. In particular, the unique 
compactly supported continuous function determined by the mask of a convergent 
subdivision scheme is a piecewise fractal function. \end{quote} 
This implies that the same technique of parametric 
composition based on FIFs and described here, would also work for composing 
sound directly using refinable sets of sound grains, e.g. wavelets.

\subsection{Hilbert Indices for Fractal Interpolation Functions}

\section{Implementation Notes}

\section{Musical Examples}

\subsection{Parametric Mapping}


\subsection{Interpolating Between Pieces}


\subsection{Using the Genetic Algorithm with Fractal Interpolation Functions}


\bibliographystyle{ieeetr}
\bibliography{gogins}{}
\end{document} 
