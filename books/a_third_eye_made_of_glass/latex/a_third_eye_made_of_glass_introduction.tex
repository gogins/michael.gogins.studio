\chapter{Introduction}

This book is one volume in a series consisting of photographs I have taken, selected from a lifetime of taking pictures. 

I have never been a professional, but I was and am committed to the art of photography. I am a ``street photographer'' open to abstraction. I shoot what catches my eye. I prize beauty. When I take pictures of people, I usually prefer they don't notice. 

I take pictures so I can see more clearly what God has created.

These pictures were variously taken with 35 mm film cameras, a variety of digital cameras, and smartphones. They were taken in my native state of Utah, other states including Washington, California, and New York, and countries around the world. I have tended to use the sharpest possible camera that is small enough to carry with me at all times. For some years now I have been using the Sony RX100 series. Increasingly, however, I am finding that my smartphone is a real camera.

The images are all in natural color. They are all used as far as possible without image manipulation; in a few cases, I have leveled horizons or removed spots. I shoot in JPEG rather than RAW. Selected metadata from the photographs is printed. If the photograph is a digital scan of a 35 mm slide, the creation date is the date of that scan. Captions are incomplete and cryptic. I have \emph{tried} to keep the pictures in chronological order.

\emph{All pictures are published in the full resolution with which they were taken}. In other words, this e-book provides more detailed pictures than all but the  largest and finest printed books. Thus, you can zoom into any image to see much more detail.

 As the file size of this book would otherwise be unmanageable thanks to the use of uncompressed images, it has been split into volumes of no more than about 500 megabytes each. Even so, the files are huge. Metadata from the images, where readable and relevant, is printed beneath the caption. If you want to know more about the technicalities, you can look up the meaning and units of the values from the EXIF tag name in \href{https://www.cipa.jp/std/documents/e/DC-008-2012_E.pdf}{\emph{\textbf{Exchangeable image file format for digital still cameras: Exif Version 2.3}}}.

I invite the reader to copy pictures out of this book for printing, but not for commercial use or re-publication. This book does not have digital rights management enabled. Kindle apps do not provide any way to save one image out of a book, but you may be able to find the book file on your device, open it with Calibre, convert it to a .docx file, and open that in Microsoft Word or LibreOffice. Then you can save and print individual images out of the .docx file, in full resolution.

\begin{table}
\centering
\captionsetup{labelformat=empty}
\caption{\textbf{Maximum Recommended Print Sizes}}
\begin{tabular}{lrr}
\hline
\emph{Source of image}                                  & \emph{Close up} & \emph{On a wall} \\
\hline
35 mm Velvia slide (to compare) & 16" x 20" & 18" x 24" \\
My scans of slides                            & 11" x 14"  & 16" x 20"  \\
20 megapixel digital camera             & 16" x 20"  & 18" x 24"  \\
High-end smartphone                      & 8" x 10"   & Up to 16" x 20"  \\
\hline
\end{tabular}
\end{table}

As I take more pictures that I think are good enough to be in this series, I will add new volumes.
