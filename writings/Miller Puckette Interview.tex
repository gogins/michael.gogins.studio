\documentclass[a4paper,10pt]{article}
\usepackage[utf8]{inputenc}

%opening
\title{On the Miller Puckette Interview by Teddy Ryles}
\author{}

\begin{document}

\maketitle

\begin{abstract}

\end{abstract}

\section{}
This came up in conversation with my friend Fred Szymanski. Fred presented to me his take on some of the things that were said in this interview, and gave me a copy of the issue of \textit{\textbf{Journal SEAMUS}}, Spring/Fall 2016, Volume 26, Number 1-2, that contained the interview. Both the conversation and the interview were intensely interesting to me. I will respond to all three musicians.

Miller Puckette is a composer, performer, and software developer who created the Max patching environment that become both the commercial software system Max published by Cycling 74 and the open source software system Pure Data. Both are systems for writing programs in a graphical patching language that will synthesize and process sound as they are being controlled in real time by a performer. Teddy Ryles is a composer, performer, and sound technician. Fred Szymanski is a sound designer and composer of electroacoustic music. I am a composer of electroacoustic music specializing in algorithmic composition.

The issues covered in the conversation and the interview included the purpose of computer music software, whether or not it can or should be neutral, the place of corporations in computer music software, the distinction between composing by writing code and composing by operating preset controls, the musical training of younger composers, and cliquishness and community in the world of electroacoustic music. This is a lot of issues and yet there is a common thread: composer software is the apotheosis of the written word, an actualized logos, a mathematical sequence. And yet, it produces a perfectly alienated simulacrum. Software is like a magic spell that one recites without necessarily understanding it. The computer runs so fast, that an image, a gestalt, a sound emerges without any obvious relationship to the incantation that of course completely determined it.

Miller Puckette wants musicians to master the logos, to write code even if they do so visually, and to understand the technicalities of digital signal processing and synthesis. The kids in classes just want to know to use Max so they can get a job as a sound designer, game composer, or production assistant in a recording studio. For many of them anything beyond learning how to operate the preset controls, to press the buttons, is of only passing interest. I have seen this myself in talks I have given to undergraduate and graduate students in music technology and electroacoustic composition. Correct me if I am wrong, Fred, but you seemed to be saying that you perceived a bias among electroacoustic musicians towards those who do not possess a technical understanding, and that you feel you are doing good work without needing to do that. I know that you do have some of the technical understanding. We also talked about Angelo Bello, who produced a CD of electoacoustic music (which both Fred and I liked) using software derived from algorithms by Iannis Xenakis, and in a review of this CD, the reviewer stated that the composer was not Bello, but Xenakis, since that software was simply operated by Bello.

My conviction is that it is always good to know the technical details but that, thanks to the mathematical brute fact of computational irreducibility, knowing the details will not and \textit{cannot} tell you what the code will actually sound like. For that, you need hands-on experience, what used to be called practicing. I also know that very few electroacoustic composers actually know more than the key concepts of digital signal processing, and that many of them are perfectly happy to play around with their buttons and sliders and loops in a program like Ableton Live or even in a plain old audio editor, without writing a single line of code, and that some of them can make very good music indeed by doing only that. 

Puckette expressed the view that what he and his interviewer perceived as a slackening of fundamental creativity over the past 30 years will only resume when a great disaster comes. I do not share this perception, I think that there are major breakthroughs in machine learning, mathematical music theory that enables more sophistication in algorithmic composition, and also an immense flood of work by non-academic, popular, and amateur musicians that is forming a lingua franca of electroacoustic music just by listening, which is the only way to really do it, and out of this a variety of native and popular styles of \textit{creative} electroacoustic music have emerged and continue to emerge. But if Puckette's thesis is true after all, then we are sure in for another major period of creative ferment because of multiple catastrophes and upheavals that have been underway for that entire period and are now becoming obvious, including anthropogenic gobal warming, migration away from the hot middle east, a change in the structure of capital to the disadvantage of most of us, robot soldiers, yada yada yada. So I'm not too worred about creativity. But I do share Puckette's distrust of corporate involvement in software development.

I will now mention a thought I had that does not arise from my conversation with Fred. The neutrality that Miller Puckette seeks in software for music cannot exist, precisely because of computational irreducibility. Anyone with a passing acquaintance with coding can read and understand most code one line at a time. If the code were to scroll under one's eyes as fast as it actually runs, it would be going far faster than the speed of sound, indeed at a truly astronomical velocity. The computer on which I am writing this is running at 4 GHz. A line of C code compiles down to 2 or 3 instructions. The computer executes the average instruction in 4 or 5 clock cycles. So lines are whizzing by at 4000000000/(2.5*4.5) or 355555555.5555556 lines per second. Or 355555555.5555556/8/12/5280*60*60 or 701.45903479237 miles per second.

Of course, people use high-level languages to code, and it is even possible for people to improvise music by coding in real time, as in live coding. But each such high-level line hides thousands of other lines. 

What I am getting at here is that software, as it is actually developed and used, is a Janus with two faces: the logos, the actual code, which can indeed be completely understood but only one line at a time; and the simulacrum, the artifact presented to human perception when the code runs. The first completely determines the second, which is thus utterly malleable to changes in the code. And thus, we compose by stumbling through the forest of the logos, seeing each tree very clearly but the forest only very dimly as a dark and even threatening mass in which we are likely to get lost.

But there is a thing that links the logos to the simulacrum in a way that can be comprehended. And that is mathematics. Which, of course, is also the link of music to its parts.


\end{document}
