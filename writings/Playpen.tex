\documentclass[a4paper,10pt]{scrartcl}
\usepackage[utf8]{inputenc}

%opening
\title{Playpen}
\author{Michael Gogins}

\begin{document}

\maketitle

\begin{abstract}
``Playpen'' is the word I use to denote a computer environment that makes it 
easy, fast, and fun to generate new algorithmic compositions. I started using 
this word to reflect my initial experiences playing with the Mandelbrot set, 
chaotic dynamical systems, cellular automata, and Lindenmayer systems in 
the Basic programming language, inspired by Martin Gardner's columns in the 
Scientific American. A few lines of code would generate a whole piece, and it 
was all new. Since then I have tried to replicate this experience with more 
powerful programming languages, and with more sophisticated algorithms, but 
with mixed results. A key component of the playpen experience was, and 
continues to be, computational irreducibility, which guarantees that small 
changes in algorithms will have big effects on the generated pieces. Here, I 
investigate the issues and means of returning to a better version of my 
compositional playpen.
\end{abstract}

\section*{Why}

Because it is fun, and even more, because it makes it easier to compose. For 
me, it is vital to be able to make a few edits in code, and then to 
immediately hear the resulting new piece. Furthermore, it must be possible to 
think musically about what I am doing without, of course, being able to 
imagine every note in advance.

\section*{Why Not}

The thorns that have cropped up in my little garden (to change the metaphor) 
are, in \emph{increasing} order of importance:

\begin{enumerate}
 \item The increasing complexity of the code, which brings in...
 \item Third-party dependencies, which increase the amount of time required to 
maintain the code.
 \item The swampland of algorithmic composition languages and systems. Every 
composer seems to re-invent the wheel, and systems that work for one composer 
are not necessarily adopted by other composers.
 \item Perhaps most importantly, it has not been possible for me (so far, I 
haven't quit trying) to implement my idea of a Mandelbrot set for music, where 
each point in a parameter space would represent the parameters for generating 
a separate piece of music, and all possible pieces could be thus generated. I 
will say more about this. To be precise, I have been able to implement the 
\emph{idea}, but I have not been able to implement it in a \emph{usable} way, 
for important reasons that I will discuss, along with possible solutions and 
workarounds.
\end{enumerate}

\section*{\emph{Mea Culpa}}

To address the less important issues first, I made some huge mistakes that I 
still have not fully corrected.

It was a huge mistake to get as involved as I did with the development and 
maintenance of Csound. I wanted to make Csound usable as a library rather than 
as a command line program in my algorithmic composition system. I did that, 
it works great, and this idea has been adopted by the Csound maintainers. 
\emph{But I should have stopped right there}. Further contributions were not 
as important, were not all adopted by the Csound maintainers, and turned into 
a huge time sink.

It was a huge mistake to develop my own algorithmic composition systems from 
scratch, and an even huger mistake to create versions of this system in so 
many different languages. I had reasons, of course, every step along the way, 
but the whole business has been a huge time sink.

I have mixed feelings about the HTML5 approach. I do \emph{not} think it was a 
mistake to try to get algorithmic composition going in HTML5 with Csound,  
it did not really take that long to do, and the WebAssembly build of Csound is 
being maintained and extended. It may still be possible to move my C++ 
CsoundAC code into WebAssembly. 

Hindsight is so great. With it, I would have learned Common Lisp, and 
developed my own algorithms in it, especially the parametric Lindenmayer system 
and the chord space library. Of course, the wider Lisp community has been no 
particular help, and neither has Ircam. Only recently has it become possible 
to treat Common Lisp as a standard language with cross-platform, 
cross-implementation library support. And it may still be possible either to 
port the best parts of CsoundAC to Lisp, or to provide a Common Lisp CFFI 
binding for CsoundAC.

\section*{Parametric Composition}

The big problem is this. Obviously the whole idea of computational 
irreducibility does mean that it makes sense to compose with algorithms. 
Otherwise I would simply not have been able to make pieces that other composers 
respect. Equally obviously, the Mandelbrot set for music idea has validity, as 
I have proved mathematically that it does. 

But there are serious problems that I have not yet been able to solve. These 
are:

\begin{enumerate}
 \item The mathematics that are used in the literature for analogous tasks 
produce results that do not map naturally to musical scores.
 \item Perhaps partly as a result of that, the scores that are generated 
contain relatively few usable ones.
 \item This all takes a \emph{lot} of computer power.
 \item I'm no mathematician, and working with postgraduate level mathematics 
doesn't always work for me.
\end{enumerate}

I'm happy to say that I have recently made some progress on some of these 
problems. 

\begin{enumerate}
 \item I have identified some mathematics, namely fractal functions and 
fractels, that in theory will map naturally to musical scores. 
 \item I have developed a better understanding of the iterated function 
system and its algorithmic implementation. The code runs faster as a result.
\end{enumerate}

Unfortunately, right now I'm stuck on fractal functions and fractels because I 
just haven't been able to unpack the math to get at how to implement them.

\section*{Next}

There are two sets of issues that I should work on.

\subsection*{System}

It remains a vexing issue to have all these languages. I'd prefer everything 
to work in HTML5 because it comes with so many goodies and runs everywhere, 
but currently, the real firepower in algorithmic composition is in Common Lisp.

I need to monkey around with these questions:

\begin{enumerate}
 \item Does it make sense to translate some or all of CsoundAC to Common Lisp? 
 \item Does it make sense to ultimately target WebAssembly?
\end{enumerate}

\subsection*{Mathematics}

I really must just bear down and try to implement fractal functions and 
fractels in C++. And I need to keep tracking the literature.

\section*{Trucking}

In the meantime I must keep composing. Since I don't really know what system 
to use I shall keep trying to use nudruz. I will also look at other Lisp stuff.










\end{document}
