\documentclass[english,11pt,letterpaper,onecolumn]{scrbook}
\usepackage{tex4ebook}
\usepackage{babel}
\usepackage{mathptmx}
% Extra leading.
\renewcommand{\baselinestretch}{1.125}
\usepackage{tocloft}
\usepackage{ifthen}
\usepackage{keyval}
\usepackage{geometry}
\usepackage{url}
\usepackage{calc}
\usepackage{array}
\usepackage{graphicx}
\usepackage{color}
\usepackage{listings}
\usepackage{supertabular}
\usepackage[pdftex,
            pagebackref=true,
            colorlinks=true,
            linkcolor=blue,
            pdfpagelabels,
            pdfstartpage=3
           ]{hyperref}
\usepackage{fancyhdr}
\usepackage{poemscol}
\global\verselinenumbersfalse
\makeindex
\definecolor{LstColor}{cmyk}{0.1,0.1,0,0.025} 
\setcounter{tocdepth}{9}
\setlength{\headheight}{14pt} 
\setkomafont{pageheadfoot}{\usekomafont{title}}
\begin{document}
\pagestyle{plain}
\coverimage{verses-cover.jpg}
\frontmatter

\title{Selected Essays}
\author{Michael Gogins \\ \texttt{gogins@pipeline.com}}
\maketitle

\tableofcontents
\chapter{Preface}
These essays mostly reflect my thinking on plausible long term futures for the human race and for other intelligent species.

\mainmatter
\pagestyle{headings}



\chapter{The Human Future and the Cosmology of Civilization}

The progress of science has seemed, increasingly, to reveal the typicality of our galaxy, our star, our planet, and perhaps of ourselves.  If the human race exists, why not other races, other civilizations?  But where are they --- the signals from space, the spacecraft, the aliens themselves?  We should not be unique, yet we seem to be alone.

To this antinomy, for which I shall propose a simple solution, let us add the basic results of fundamental physics, cosmology, biology, and philosophical logic: the asymptotes of which may be appearing.  These considerations I take as constraints for my imagination of the future.  

Wait!  Don't free will, individuality, and chance make history a strictly indeterminate path?  I don't deny it.  Are the limits of technology, of humanity, of consciousness, and of history so apparent?  I agree, they're not.  And in fact, I will propose what I consider to be some very interesting choices, technologies, and historical singularities.  However, from the perspective of science fiction, the first principle for imagining the future is:   Imagine nothing contrary to the limits of science.  And to this principle I add a second:  Imagine as much as possible consistent with science.

\section{The Antinomy of Sapience}

Our search for extraterrestrial intelligence has only just begun, but it seems to me quite likely that, if any alien civilization had existed for very long in our galaxy, we should already have detected it, incidentally to astronomy.  Our broadcasts, and our modifications of the climate and surface of the Earth, must already be apparent across astronomical distances.  As discussed below, the feasibility of interstellar flight must be assumed.  So there is little reason to doubt that further technological progress will make our existence even more obvious.  I judge that the antinomy of sapience is a true one.  There are no extraterrestrial civilizations within our galaxy, or our neighbor galaxies.  

I agree, however, that humanity is unlikely to be unique in the cosmos.  However, let us recall that the cosmos is expanding.  Civilizations, too, are presumably expanding.  The proportion of the cosmos occupied by civilizations is a function both of their rate of origin, and of their rate of expansion.  If the compound rate of expansion of intelligent life is on average no faster than the rate of expansion of the cosmos itself, then most civilizations must be alone within their cosmic horizons.  If the expansion of civilization is at all slower than the expansion of the cosmos, then in fact each civilization must become increasingly thinly spread in space.  I propose that this is in fact the case.

Axiom: 	Civilizations occupy an infinitesimal volume of the cosmos.

\section{Faster than Light Travel}

Does this mean that each civilization is fated to remain alone throughout its history?  Not at all.  Being thinly spread is one thing, and remaining forever isolated is quite another thing.  It all depends on whether faster-than-light space travel is practical.  According to valid solutions of the equations of general relativity, both faster-than-light travel and time travel are theoretically possible via extreme gravitational torque.  At this time, all conceptual designs for gravitational torques involve objects of astronomical size and singular density, driven by energies of cosmic scale.  Mind-boggling though it may be, in keeping with the second principle of imagination, I must assume the validity of general relativity until such time as scientists accept a superior theory.  These then are the basic possibilities:

\begin{itemize}
\item	Despite the theoretical possibility of faster-than-light travel, it is not practical for living human beings.

\item	It is not practical to build gravitational torques, but it is possible to exploit naturally occurring ones generated by celestial bodies, such as black holes.  Under these circumstances, civilizations would gradually migrate to these bodies, even if it took sublight spacecraft many years, or even generations, to reach them.  Flights between torques could only be accomplished between black holes, and would require starships and months or years of ship time.

\item	It is possible to build gravitational torques, perhaps by exploiting a unified field theory to build an electro-gravitational transducer, and therefore to build faster-than-light spacecraft, or even locally stabilized gates.  These would be gates that maintained a fixed position local to the frame of reference on each side of the gate.  However, gravity is monopolar, and so weak compared to other forces that I have great difficulty imagining a torque sufficient for a gate on the surface of a planet.
\end{itemize}

Axiom:	Gravitational torques can be built, but not locally stabilized gates.  Faster-than-light flights, which also may be time travel, can be accomplished anywhere in space, but still require starships and months or years of ship time.

\section{Extent and Structure of Cosmic Civilization}

Modern cosmology suggests, on quantum mechanical grounds, that the Big Bang was a rather arbitrary event, an instability of the vacuum that... just grew one day.  There is no particular reason to suppose that it only happened once.

Axiom:	The cosmos is infinite in time and space, and so is cosmic civilization, for even an infinitesimal portion of infinity is still infinity.

It follows from the preceding axioms that civilizations that survive long enough do contact other civilizations ---  by FTL starship, not by radio.  Most such contacts will occur between two civilizations that each possess starships, most of them will occur as relatively newly starfaring civilizations contact the existing web of much older starfaring civilizations, and most of them will occur as those newly starfaring civilizations fly to investigate scientifically interesting celestial bodies, where they will find starships from other civilizations.  These bodies will serve both as natural beacons and as the true crossroads of interstellar civilization.  There may also be artificial beacons.  However, it does not necessarily follow that the greatest cities will be located at such crossroads.

If civilizations last very long, most of them will be part of the Web.

\section{Military, Economic, and Political Structure of the Web}

Human history is generally written as political history determined above all by war, and there is no reason to think that cosmic history should be any different.  It is said that liberal democracies do not make war on each other, but there is no particular reason to believe that liberal democracies will replace all other forms of civilization.  Even if they did, one evil being armed with a gravity torque could destroy just about anything.

Axiom:	Offense has an overwhelming and permanent advantage over defense.

Under these circumstances, the Web cannot invest its safety in any one fixed city.  It follows that the Web must be founded primarily on starships.  It is infinitely old, and yet none of its cities are infinitely old.

It is to be expected that the Web has pushed technology, and biotechnology, to their theoretical limits.

Axiom:	In additional to gravity torques, the Web has direct conversion of mass to energy, self-reproducing nanotechnology, and complete genetic engineering.  It also has cyberspace that interfaces directly with the brain.

It follows from these axioms that each individual civilization, or for that matter each individual starship or even clan, faces a fundamental dilemma:  remain isolated in the infinite wilderness of space for the sake of safety, or join the Web for the sake of civilization.

It might be wondered whether the Web has already achieved all that can be achieved.  However, the existence of computational irreducibility guarantees that, although the Web possess our hoped-for true theory of everything and has pushed fundamental technologies to their theoretical limits, there is no limit on the development of higher levels of technology and culture.  Even if the basis of civilization has achieved a stable form, the superstructure elaborated thereon never will.

Axiom:	The technological foundations of the Web are infinitely ancient, and are in the public domain.  The only basis for capital under these circumstances is original intellectual work and services.  The medium of exchange is credit for access to files in Web cyberspace.

Irreducibility (incompleteness) also guarantees that intelligence, whether natural or artificial, is self-transcending and self-determining.  The political dialectic of the Web is now becoming clear.  There is incredible diversity, so the only possible stable foundation for intercourse in the Web is liberty and the Categorical Imperative, resulting in a fairly minimalist governmental structure.  Beyond this, there is a fundamental need to balance security against freedom of travel and trade.  There can be no fixed center of authority, but there are more or less stable forms of adjudication.

Axiom:	The current fundamental institutions of the Web, in order of logical priority, are the Process of Judgment, the Cyberspace Foundation, the Software Security Force, the Exchange, and the Interstellar Militia.  All of these are based primarily on starships and starship computers.

Because of the dialectic between the stable technological basis and the evolving cultural superstructure, these institutions are probably metastable.  There is plenty of scope for narrative here.

\chapter{Background to the Aperture}

	I may write more science fiction.  The following is a statement of the scientific and philosophical assumptions upon which I choose to base this writing.  I state these assumptions as facts in the history of philosophy, science, and technology.

\begin{center}\rule[3pt]{2in}{0.5pt}\end{center}

	A number of physical theories have been proposed, all of which unify gravity and quantum mechanics, each of which covers all known observational and experimental data, any one of which may well be the ``one true theory of everything.''  To test these theories decisively would require truly cosmic energies and is thus beyond human, or for that matter nonhuman, power.  Indeed, after the Aperture, it was discovered that interstellar civilization, for all its unlimited antiquity, has done essentially no better in this respect than human science.

	Although they make the same empirical predictions, the unified theories differ somewhat in their philosophical presuppositions and implications.  It is also too difficult to explicitly solve the equations for systems beyond a rather modest limit of complexity.  And as for nonlinear systems, almost all of them, even the simple and solvable ones, exhibit chaotic behavior.  All three of these fundamental limitations of cognition may be considered aspects of computational irreducibility.  Every civilization known to have achieved unified theory has thereafter and therefore undergone a fundamental change of direction, away from fundamental physics and towards either computational simulation or the cultivation of intuitive traditions or, more commonly, a synthesis of simulation and intuition.

	The unified theories have similar consequences for the philosophy of mind.  The most probable interpretation is that self-consciousness requires an actual infinite regress of awareness, which is possible as a fractal structure in the nonlocal view but impossible, due to phase changes enforced by Planck's constant, in the local view.  Reflective self-consciousness, then, is generally held to consist of nonlocal self-awareness acting both into and by means of local reflection.  Interstellar civilization, insofar as its members share this particular structure of reflective self-consciousness, tends to share this model of consciousness.  Its major consequence is that no computer, i.e. no universal Turing machine, can truly be self-conscious.  This has not prevented the construction of truly conscious artificial persons; but they are not Turing machines.  

	Another consequence is that telepathy is impossible, not to mention the transference of consciousness from one body to another, though sensory interfaces so complete and intimate as to create a convincing simulacrum of these impossibilities may certainly be contrived.  It is also posssible to transfer or instill recorded or artificial memories.

	Unified theories overcome the temporal and causal paradoxes and singularities of relativity theory, now called the local view, by an extension of the notion of phase cancellations in the Schroedinger wave equation, now called the nonlocal view.  As general relativity predicted, it is possible to travel both forward and backward in time.  The causal loops or contradictions that might seem to follow from this cancel each other out in the nonlocal view, and therefore do not actually arise.

	Unified theory made it possible for engineers to construct the gravitron, for generating gravitational potentials by means of electromagnetic energy.  The gravitron, in turn, made it possible to open up artificial Kerr horizons, or gates, for instantaneous travel in space and time.  The cost of such travel is little more than the difference in gravitational potential between the starting point and the end point.  (Of course, this represents a large amount of energy indeed! --- in time travel, somewhat exceeding the mass being transported).

	It is possible to open a gate anywhere, even on the surface of a planet, because the gravitron itself supplies a means for creating the absolutely tide-free gravitational field required.  But actual passage through the gate is necessarily ballistic --- either the gate itself or the vessel must move --- so it is not possible to calculate one's vector of arrival with absolute precision.  Therefore, a vessel may leave the surface of a planet with confidence, but can only arrive safely by making a series of closer and closer jumps in the relatively unobstructed field of space.  Even in space there is a small but definite risk of emerging on a collision course with, or even into the space occupied by, another object, which may well destroy both in a violent explosion.  This is in fact the major risk of space travel.

	The gravitronic gate makes possible not only faster-than-light and temporal travel, but also the conversion of mass into energy with at least almost 50 percent efficiency.  This is fortunate, for otherwise time travel would be prohibitively expensive, and even mere interstellar travel would have to be reserved for government emergencies.

	The final consequence of gravitronics is the duplication of arbitrary objects, including persons, by sending them just slightly backwards in time, although of course at the cost of somewhat greater mass than the original.  This kind of work is permitted only in interstellar space, where the accidental failure of the process will not destroy any planets.

\begin{center}\rule[3pt]{2in}{0.5pt}\end{center}

	The first historical consequence of the gate was human contact with interstellar civilization --- the Aperture proper.  The term applies both to the moment of first contact, and to interstellar civilization itself, which considers itself to be always an opening upon another aspect of itself.  

	Efforts in the late twentieth century to detect or contact interstellar civilization by radio had necessarily failed, because interstellar civilization is quite sparsely distributed and uses radio, if at all, only for local communications.  As for gravitronics, it seldom creates cosmic fireworks; and when it does, they resemble the natural ones.

	Before describing interstellar civilization, it is well to consider other purely human fundamental technologies.

\begin{center}\rule[3pt]{2in}{0.5pt}\end{center}

	The other fundamental achievements of human technology, which were begun but not perfected before the Aperture, are genetic engineering, nanotechnology, and cyberspace.

	Genetic engineering is now complete in the sense that the human genome has been completely transcribed, and any mutation in it can be simulated.  Most somatic and genetic diseases have been eliminated.  Those that remain do so because their genes also perform some vital function, and no substitute for them has yet been contrived.  Many of the genes responsible for human aging and death have been eliminated or redesigned.  The baseline human life span is now in the neighborhood of 800 years.  After that time, the effects of the remaining ambiguous genes, the accumulation of ineradicable errors in DNA replication, and the memory constraints of the human brain enforce a gradual senescence, which can go on for centuries.

	The general capabilities of humanity have been elevated, so that what used to be the second standard deviation and a half of endowment has become the median.

	Some hold that as a result of these many changes, the biological substratum of Homo sapiens no longer merits the term ``human,'' but should be called ``transhuman'' or ``neohuman.''

	Genetic engineering has also been applied to plants, animals, and the construction of artificial ecologies for spacecraft and orbital dwellings and cities.

	A number of existing or designed creatures, including dogs, cats, apes, horses, even some larger birds, have been endowed with intelligence, speech, and hands.  ``Humanity'' has thus become a cultural rather than a strictly biological term.

\begin{center}\rule[3pt]{2in}{0.5pt}\end{center}

	Nanotechnology has achieved the construction of quantum von Neumann machines.  These are universal Turing machines with universal constructors, and thus capable of self-reproduction after the manner of living things, built on a molecular scale, whose operation is based upon quantum mechanical principles and therefore is as efficient as physics permits.

	Most of the vessels, instruments, and weapons of the Aperture are nanomachines.  Artificially intelligent machines are also of this type.

	In the comparison of nanomachines with living things, the components of living things are amino acids and ribonucleic acids, whereas certain vital components of nanomachines are crystals that must be of slightly greater size.  Thus living things tend to be more compact or complex, but slow and subject to heat and noise, whereas nanomachines tend to be simpler but somewhat larger, and much faster and more efficient.  

	When it comes to thinking, the organization of biological brains, evolved over geological epochs, has a weight of design that tends to compensate for less subtle programs running on much faster hardware.  The non-Turing machine aspects of consciousness favor neither biology nor electronics.  This is true in human civilization.  

	In the Aperture, however, nanomachinery is billions of years old and has fully realized its inherent computational advantages (not the least of which is a longer life span).  There are artificial intelligences in the Aperture whose intelligence is far vaster than human.

\begin{center}\rule[3pt]{2in}{0.5pt}\end{center}

	Cyberspace is a universally distributed, switched-access network of computers and communications media, utilizing a common storage format and a virtual reality interface.  The physical interface is generally nanotechnology worn upon or within the body.  It provides a sensorium that either overlays, or tiles (e.g., a view out the back of one's head), the natural one.

 	Cyberspace has completely replaced the previous publishing houses, recording industry, telephone system, broadcast networks, and computers of humanity with a single medium of communications, transmission, and storage.

	Cyperspace both understands human speech and can respond in kind.  There are also many other perceptual interfaces for cyberspace, but the most commonly experienced are those for the telecommunications services, the librarian, and the simulation utilities.  

	The telecommunications services can produce a simulacrum of each party in a conversation that appears to occupy the real space of each of the other parties, or all the parties may meet as telepresences in a virtual space.

	The simulation utilities do what their name portends, and tend to be enclosing spaces reached by doorways from the librarian.

	The librarian, naturally enough, occupies what appears to be a very large, beautiful, and yet comfortable library, which contains many other virtual spaces.  This space is also used for telecommunications and thus can be considered the central space of cyberspace.

	Almost all machinery in the Aperture is operated by means of a cyberspace interface, frequently in the form of a command language.

\begin{center}\rule[3pt]{2in}{0.5pt}\end{center}

	Humanity had begun to develop these critical technologies prior to the Aperture, but upon joining interstellar civilization found them in their mature, and quite overwhelming, forms.  

	If the fundamental limit to technical cognition is computational irreducibility, consider what billions of years of practical experience must add to an engineered life form, nanomachine, or complex algorithm.  An analogy:  two schools of the dance.  In the one, each generation of dancers is active for fifteen years, then dies after laboriously passing on its teaching to the next generation.  In the other, each generation of dancers passes its skills in toto and instantaneously to the next, which can then occupy itself with learning an entirely new style of dance without losing any skill of the others; furthermore in this school, each generation of dancers is active for seven hundred years.  It's obvious which school will dance best, at least from the technical point of view.

	Now consider that what applies to these dancers can apply to an intelligent blade, a spacecraft navigation system, a couch for reading and sleeping.

	With regard to any definite tasks, there is an irreducibly optimal algorithm, which cannot be deduced from first principles, but is inevitably discovered after sufficient experience.  The Aperture has in fact discovered the optimal algorithms for most humanly definite tasks.  Humanity, believing itself with good reason to have achieved the fundamental limits of technology, found itself in contact with a civilization whose products, both technically and artistically, were demonstrably and overwhelmingly superior.

	It took some time for humanity to accept the weight of sheer experience; to realize that the acceleration in technology between the twelfth and twenty-first centuries was a catastrophic collapse from one stable regime to another.

\begin{center}\rule[3pt]{2in}{0.5pt}\end{center}

	Like the system of interconnecting universes that it explores, interstellar civilization has no known beginning in time or circumference in space.  Those who devote sufficient effort have always been able to find an earlier ancestor or a further outpost.  However, humanity knows that the Aperture extends at least five hundred billion years into the past (at least 30 times older than our local universe) and occupies millions of universes, or more.

	Equally however, not only the local universe but the entire system of universes is on the whole expanding, and that faster than the Aperture; even though it, like recently joined humanity, is in the throes of a permanently exponential population explosion.

	As a result, although the Apeture is a network of unthinkable population and density, it is located in physical space with almost complete sparsity.

	This sparsity is relative; some universes are ancient and crowded, but then these dense universes are themselves widely separated.  In short the distribution of the Aperture is a curdled fractal, and unpopulated wilderness is far from almost no one.

	The shock of the Aperture to humanity was double:  on the one hand the billions of years of culture; on the other hand the permanent and inexhaustible frontier.

	As for the politics of the Aperture, they are somewhat incomprehensible, but in their humanly relevant sections guarantee freedom of information, of travel, of trade, and of residence throughout the Aperture; prohibit war, murder, slavery, and ``unlawful use'' (i.e. duplication of a person without their permission, rape, slander, etc.).  The machinery of law enforcement is spread, unfortunately, as thinly as the Aperture itself.  

	In centers of civilization, one is quite safe from the grosser crimes, but subject to levels of manipulation and persuasion that most human beings probably cannot begin to suspect.  In the wilderness, one remains oneself, but had best go armed.

\begin{center}\rule[3pt]{2in}{0.5pt}\end{center}

	The development of late technological capitalism was an explosive crisis with a radical denoument.  Between the fall of Communism in 1991 to the Aperture itself in 2048, only 57 years obtain.  Within those years humanity mastered genetic engineering, conceived the fundamentals of nanotechnology, constructed the initial spine of the human cyberspace network, put forward several of its own unified theories, and developed the gravitron.  At the end of that period, anyone who wished to could obtain a gravitronic starship built of self-reproducing and self-maintaining nanotechnology, and containing the essential core of the human library in toto; this vessel could construct copies not only of itself but of any other object whose design was stored in its library.
	
\end{document} 
