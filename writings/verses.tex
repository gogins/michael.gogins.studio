\documentclass[english,11pt,letterpaper,onecolumn,parskip=full]{scrbook}
\usepackage{tex4ebook}
\usepackage{babel}
\usepackage{mathptmx}
% Extra leading.
\renewcommand{\baselinestretch}{1.125}
\usepackage{tocstyle}
\usetocstyle{allwithdot}
\settocstylefeature[-1]{leaders}{\hfill}
\settocstylefeature[-1]{pagenumberhook}{\nullfont}
\usepackage{ifthen}
\usepackage{keyval}
\usepackage{geometry}
\usepackage{url}
\usepackage{calc}
\usepackage{array}
\usepackage{graphicx}
\usepackage{color}
\usepackage{listings}
\usepackage{supertabular}
\usepackage[pdftex,
            pagebackref=true,
            colorlinks=true,
            linkcolor=blue,
            pdfpagelabels,
            pdfstartpage=3
           ]{hyperref}
\usepackage{fancyhdr}
\usepackage{poemscol}
\global\verselinenumbersfalse
\makeindex
\definecolor{LstColor}{cmyk}{0.1,0.1,0,0.025} 
\setcounter{tocdepth}{9}
\setlength{\headheight}{14pt} 
\setkomafont{pageheadfoot}{\usekomafont{title}}
\begin{document}
\pagestyle{plain}
\coverimage{verses-cover.jpg}
\frontmatter

\title{Selected Poems}
\author{Michael Gogins \\ \texttt{gogins@pipeline.com}}
\maketitle

\tableofcontents
\newpage
\chapter{Preface}
	Some of these poems were published in little magazines, mostly in the Western United States, from the 1970s through the 1990s.
\mainmatter
\pagestyle{headings}

\chapter{Prose Poems}

\section{I Sleep}

	I sleep with a naked sword between me and my love.  We lie not touching but dreaming of mutual death, our glittering wave that overwhelms the Western cities, the tidal wave a mile high and full of sunlight that shines between the floating houses and the dead things.

\attribution{August 15, 1988...June 1, 1989}


\newpage
\section{The Anti-Monogamist Crusade}

	In the anti-monogamist crusade I've been rancher, soldier, prisoner, urban refugee, and finally poet.  For I have learned the most fundamental trope.  It is dislocation, exile.

	Dislocation separates the meanings of one thing, say love, into the redoubled terms of metaphor:  one flesh, two bodies.  Dislocation analyses motives into the unity of opposites:  promises, power.  And so dislocation foreshadows the false denouements of our Empire.  Despite the barricades of the Monogamists, the capitol is decked in white today.  For today we celebrate our beloved Crown Prince's latest, in fact his seventieth, royal wedding.

	My own complete dislocation has been accomplished not through my being dispossessed of my ancestral holdings by the tanks of the Emperor, not through my forced exile from our mountains of black pine, not through the drugs and alcohol of the slums, and not through prolonged meditation or other ascesis, though I have suffered all of these, but through faith alone.  For only faith endures all change of body and state.

	The wheel of the seasons returns me again and again to my task, as if I am being hammered on the anvil of the Sun.  I am seeing all times becoming one time, all cities emptying into this one city wracked by civil war, and all peoples being pounded, despite themselves, by the clash of religions, into one people.

	When he first began to study Yoga, my older brother lived for several months in a mountain canyon with the half-Cherokee woman from the plains he later married.  Slide showing a half-cave in a red-rock cliff, propane stove and aluminum pots set along a slab of sandstone, plastic tarp for a tent, two smiling suntanned lovers.  Later, he joined me in the capital, in a rented house of mud brick in the middle of a block.  He built gliders of paper and balsa and kept a parrot, but his periodic months of Yogic celibacy infuriated her.  I would sit with them through the winter nights, drink tumblers of red wine in the white lamplight, and look at her olive, high-browed face, for it was really her I liked.  And the children in their pink and peach pajamas, cheap toys in hand, would come forward to examine me with solemn black eyes.

	Eventually, my older brother was recruited by a fiercely polygamous sect.  His wife divorced him in outrage, and returned with her children to the still-rebellious plains.  He went back with his two new wives, as one of the enemy, to our conquered mountains.  When winter came on, he was shot by our younger brother, who'd lost his own wife and lands to an Imperial.

	Now the cold wind off the tundra blows snow so high it covers the first doors and reaches the thresholds of the winter doors on the second stories.  In the loft where I used to work, I know that the windows are steaming up from the breath and perspiration of the operators, whose hands of tender flesh must screw together the titanium hands of robot after robot.  At the end of the street, the black brine grinds its floes of ice against the granite piers.  And through tubular bridges of light strung between the office buildings far above, traders and secretaries in pastel silks saunter to and fro, chewing nuts and spice.

	I came here with nothing.  Yet I am determined to rule, even if it is only myself, my inch square field of will.  Sometimes I go out in my old stubs of charred plywood, the yoke I wore as prisoner of war.  For I like to flaunt my rebellion and escape in the chaos of the enemy's very capitol.

	Fear does not rule me, or limit my limitless ambition, though we all rightly fear the armored divisions.  Nevertheless I have no wish to ape the repetitive conquests of our polygamous Prince with some futile victory of blood.  I am revolted by the sophistries shouted in the markets by professional debaters, and by the riots incited by armed Monogamist missionaries.  So I have chosen my own battlegrounds with care:  poetry and faith, the decisive contexts of action.

	This afternoon, after meditating upon the central figure, I sacrificed a pair of prayer wings.  I touched them with a match, and they blazed up into wings of fire, wings of smoke, wings of glowing ash, which crumbled with the breeze into aromatic dust.

	Now Yekaterina climbs the six flights to my roofless room.  She sees how cold and sick I am in my Mexican blanket, how I shiver and speak deliriously of computers to link each mind, of golden scarabs that we could forge to monitor and feed our brains.  She touches my forehead with her hot indifferent hand, and asks if my escape and my sacrifice were worth it.  I remember meeting her at the gallery where she works downtown.  That night they were showing a backwards-leaning throne made of oak trunks lashed together, far too big for anyone to sit in, and its seat was splintered as if by lightning.

	``Worth it?'' I ask.  Her tongue is so pink!  Above, a shell has torn away the roof.  As the night breeze blows ice clouds across a crescent moon, plaster dust drifts down.  I want to sneeze but am too weak.  I think of campfires in the mountains of my birth, and how my father and I used to sit our horses under black pines heavy with snow to look down into the distant glow of this city, coals banked against the dark ocean.  I think of the huts built everywhere here, their little fires flickering, even halfway up the great thick cables of the suspension bridges.  And I wonder how my sister fares, who lives on the other side of the line of rubble dividing the quarters.  But to visit her I'd have to pass those sandbag booths of dark-faced men who demand payment or a certain name, and I have no name but my own.  Indeed, names are too dangerous --- I have no name at all.  I have only these words, which swirl through my consciousness like the glowing sparks of prayer wings one burns as evening falls.

	Deliriously I speak to Yekaterina of restoring the ruined fortunes of my family through canny trading. of ships I will buy, though our nation no longer has ships except for ships of war.  And my family were ranchers and miners, anyway, who never smelled the sea.  In my fever I describe the suit I'll order, with glittering threads of power, bulletproof panels sewn in its lining, buttons of jewel.  ``Worth what?...'' I mutter, as the fiver rises in me again.  Yekaterina brings me another cup of water from the pump in the yard below.  I cough.  But when she leans over and I sense the heat beneath her soft grey sweater I start thinking of marriage, of dynasty even.

	The burning core which drives conquerors now burns in me, for I have tasted the very dregs of dispossession.  There is no glass in the round windows of this beachfront apartment where I squat with my rough friends, where in the cold noon retsina burned our throats, and where playing cards still lie scattered on the black carpet.  And the sea air in my nose reminds how in the summertime, before the fighting entered the city itself, my sister played her flute for coins on the boardwalk by the amusement pier.  Last I heard she was shining shoes and fetching lunches in the stock market, scurrying beneath its glowing screens.

	Dislocation has taught me that all empires, not only the Kingdom of God but even ours, are founded on the immolation of innocence.  It is why one side of a street is brick row houses with plaster saints and olive-skinned kids hosing down a Pontiac under peach trees in October sun, while the other side is all cyclone fence and railroad tracks, where the bitter wind from the north plains blows trash into the fence, and cardboard huts huddle against the burned-out factories.  But my fever confuses me:  Do I want to be the innocent immolated one, or one of the cursed immolating ones removing their curse by immolating the innocent one?

	Night falls, and battle satellites chase each other across the smoke-streaked blackness filled with stars.  To celebrate the seventieth marriage of our beloved Crown Prince, the lasers of the city fire needles of green light up from the tower tops, as if to defend his bright illusions against nightmares in the outer darkness.  But the black book I sleep on tells me that the only innocent one among us, knowing we cursed ones had no one else to immolate to remove our mutual curse, stepped forward and offered himself, himself and not another, to be bound between the oil-drenched wings of sacrifice.

	I enter more deeply into his act as I taste Yekaterina's second cup of cold water, its grains of wet plaster gritting in my teeth.  I suppose I should be glad she even thinks of me among our teeming surplus of unmarried men.  Now she's saying, ``A girl I know has been asking about us, if I would mind, and I said no....''  I think of how I have so often contemplated, and even oftener postponed, my marriage.  I remember mountain weddings of few words said once.

	Anti-aircraft guns are clapping from the edge of the seaport, firing either against the monogamous quarter or in celebration of this latest royal wife.  Yekaterina's face, her gallery girl's face, bent with momentary tenderness over mine, is lit by green flashes on the left side and by orange flashes on the right.  I refuse her wildly, I refuse them all, I feverishly refuse till I should not only be loved but also love, once and only once, in return.

	And I do not know if I am foolish to do so, but it certainly loosens my tongue.  I have learned the most fundamental trope.  It is dislocation, exile.  Dislocation separates the meanings of one thing, say love, into the redoubled terms of metaphor:  two bodies, one flesh.  Dislocation analyses motives into the unity of opposites:  power, promises.  And so dislocation illuminates the false denouements of our Empire.  Despite the barricades of the Monogamists, the capitol is decked in white today.  For today we celebrate our beloved Crown Prince's latest, in fact his seventieth, fallacious royal wedding.

	The whole city is white, white everywhere, except in the one most needful place.  For the Imperial groom and his seventieth teen-age bride, in the tinny box of television, in their heresy, are not wearing the white wings, the paper wings of monogamy.

\attribution{October 15, 1988...June 3, 1989}

\newpage
\section{\emph{Jupiter and Semele}, by Gustave Moreau}

He looks like a brute all right, slouching with a pout, with the red halo of a Satanic Christ, and dark smoldering eyes.  He's almost a beer ad or a faster, purer kind of car.  If only he weren't fourteen feet tall!
Semele stretches herself from his hip, perhaps she is trying to get away, and the clasp of his girdle regards her with a living eye of onyx that holds her in its grasp.  Or perhaps the white bow of Semele is stretched on his hip like his harp's mute box, that hides as much of his shadow as it can inside its square of geometry, the square of knowledge that resounds with shadow.

He muses:

Babe, you're such a waterfall of skin, Babe, your thighs are as white as my dream as I dream it, Babe, your skin is real but my dream unfortunately is not, Babe, so I have your whiteness stretched out over the night, and I am painting without thinking, but only more white, white over white, white in white.  I like to paint at night, and I paint white on white, and my frame is black and crusted, and my shadow is full of music, and I have our skeletons to shake at will inside my box of tin.

And when I stretch my own fabric thin enough for your bed, I'll become painful droplets that give mortal life.

\newpage
\section{In the Head of the Idol}
	My enemy, my son, I have forgotten your year.  I do not believe in reincarnation, but I am lost in the echoes of my father.  I confuse myself with him, and with his father, and sometimes you with me....
	
	At first I hesitated to name our true oppression.  It is shameful to want another's guilt for one's own, and without realizing it.  In the can I found out back there was a layer of your dead brothers and sisters under oil.  We also (naturally, in the course of this game, ``I'' becomes ``we'') --- we also have the fire of books and the smoke of false love, but nothing can hide the splendor of your irremovable lamp.  It makes us stick to the walls like its shadows because we are afraid to face it.
	
	If for one moment I ceased lying to myself I would resume the shape of Adam.  Where in the alleys would you find paradise then, Eve?  Now that the serpent is gone, we are reduced to corrupting each other.  I've hidden myself in this rented room, and I have a job in a garage dipping the new engine in solvent, but I'm afraid, my son, that you will find me and kill me for my key.  You wouldn't recognize your mother among all the others, but you'd know me.

	Since I was born I have forgotten all the angels of pre-existence and their swords except for one image, the shadow of a colossus falling over a battlefield from one end to the other.  Everybody was either dead or sleeping, and smoke arose from the sodden green trenches.

	When I pushed the shop doors open this morning I found the lathes had been working all night by themselves.  They had produced an enormous idol with steel fists for destroying other idols.  I must have programmed the lathes in my dreams.  Now I climb the rungs up his side into his huge crystal head, sit on the chair within, slip my feet into the boots, my hands into the silver gloves.  I don the soft helmet and the goggles of blue television, and pull the canopy down over me.

	Dust boils up around my first gigantic step.  I cannot stop trying to make myself into something more than a human being.  Soon I will give myself gills and wings, put voices and images in my brain for linking with machines.  I will join the ancient conspiracy of the alien nations, the nations that crept out of the black holes remaining from the fallen universes.  We will clement our rebellion by a treaty written in the chemical signatures of our mingled blood.

	Nevertheless, my son, the memory of your feet pursues me.  On the mountain you were pierced at my command and for that reason in whatever fire I dwell I will always taste your last kiss, and my blindness will be no defense.
\attribution{July 1, 1988}

\newpage
\section{The Resting Places of Sisyphus}
	Through its constant falling back and his rolling it forward yet again, a little further reach time, his boulder has become light to him, and Sisyphus has attained the top of his slope.  It is a valley which affords him rest.  The yellow sky is reflected in a stream and its ponds, geese ``vee'' along the jagged horizon.  There is a road, commerce, arches, taverns.

	For several days he rests and then --- again the slope, steeper and higher than before.  And so it goes, seasons revolving into years.  Hills, each higher or lower than the last in no discernible pattern, alternative with valleys, each peopled.  After languages and customs have blended into the blue of his eyes, Sisyphus conceives of watersheds, of an ocean, of a final valley in which he will scent at last the salt breeze that will end the commandment of his journey.  There he will leave his boulder behind, utterly indistinguishable on the rocky shore.  And there, at the mouth of a river, he will find a tile-roofed town that never heard of the tricks a king too clever for his own good once played upon the gods, a town with perhaps a few widows of seamen, and a window with a lamp.

	But the fear that these hills and valleys are only ripples on the waves of yet greater hills, repeating and expanding to infinity, arouses disquiet, becomes worse torture than that first slope Sisyphus so naively had thought unending.  For the intelligent well know that infinity is neither odd nor even, neither slope nor valley.  Does, then, this cordillera of brown and red peaks, this series of roads and tongues, neither end nor not end?

	And the geese, should he understand them as his comforters or as his tormentors?
\attribution{December 13, 1987...June 1, 1989}

\newpage
\section{The Language of Hunger}
	There is no substance to my life.  I have no myth, no story, no goal.  I live in a room in a great city and work in another room.  The light changes outside the windows, hour by hour and day by day, but it is all beyond me.  My friends are faces passing behind a glass screen.  I am aware of the wrongs of the age, but I do nothing to oppose them.  I am passive, waiting, depressed.

	I recall scraps:  white clouds over a lake, thighs opening, lamplight on a wall, conversations at night under the trees while the wind blows.  These scraps imply that I do have a life, but they do not connect.  They only remind me of my lost substance, so that rays of longing run out into the darkness to connect me with myself.  But the rays do not gather to any bright point, any star.  And so I remain in myself, not suicidal, not content, and not forgetting.

	I used to have adventures without purpose:  following a woman around a corner, reading a book pregnant with someone else's idea, boats.  Now all memories lie under a vast sea of ice, dead leaves trapped beneath autumn after autumn of frost.  I await some spring which will melt the ice and release the leaves to their colors, their flood, their end, the tender green their death will feed.

	I am suffering, but not for the wrongs or sins of others.  My pain does nothing to relieve another's pain, though I wish it did.  This longing has about it a tinge of blue, as though the gritty tops of the buildings felt a touch of warmth, as though the sunlight with which spring shines through the green trees along the tracks fell on naked backs, on those lying down, on children.

	I think of the book which will reveal all the other secret sufferers and interpret the tongues of the hidden angels.  I know that my task is to write this book, though of all who write I am the least like an angel.  There is no unselfishness in me, no generosity, only this mute hunger for the bread some fabulous sky might let fall, some port of peace in which empty boats might rock, the antipode of the darkness in the killing basement and under the lonely bed.

	This is a hunger which is blind and dumb, which does not even know the name of its food, which feeds on hollowness and becomes hollower, and yet which tastes the lies in the circling screech of the subway wheels and the untruth in the hum of the fluorescents in the office.  This hunger has searched the open mouths of lovers, the tumultuous light of their sheets, the scattered food on tables after dinner, and it has followed after the poor people hurrying to the fields in which they will labor.

	This hunger is deeper and mightier than I am, and when it forces open my mouth I taste that emptiness so profound its echo is greater than speaking.  I hear the directionless mutter of the homeless, of ghosts, and of characters in stories.  But these voices unite only in the hiss of the shell, the feeble thunder of silence; and the flashes of light which strike me are only sunlight reflecting from the windows as I leave the office.  Silently, I pass many people in the street.  And though I cannot yet speak, I know that the language I am learning will express everything in our walk, the junk in our purses, our briefcases, our shadowed eyes, and our secret wars.

	It is then that I know I am an army, and that I await the sharp tongue of light in the mouth of the sky, the cry of command which will break the silence, and the manes of the white horses.
\attribution{May 24, 1988...June 3, 1989}


\chapter{Verse}

\newpage
\section{Piano Microscope}

\begin{poem}
\begin{stanza}
My inflicted self's\verseline
Blow-blinded eye's no ear\verseline
Yet hears the sparrows on black wires\verseline
In rain as grey as rain\verseline
From fragile ink of sky
\end{stanza}

\begin{stanza}
Let my torn Bible hide you\verseline
In her Martian shadows\verseline
With clock wings of feathery steel\verseline
And feel the mask of quilt\verseline
To sleep with beating hands
\end{stanza}

\begin{stanza}
For the city of music\verseline
Is one white room\verseline
Where a child overinterprets guilt\verseline
And skips through bars of light\verseline
In sudden beats
\end{stanza}

\begin{stanza}
So rise, pilots\verseline
Through circumstance of blank\verseline
And with piano microscope\verseline
Search the stockings of sound up\verseline
To their soft detonation
\end{stanza}

\begin{stanza}
Where the deepest whisper turns\verseline
A brass knob without a legend\verseline
To open the bright mouth of duty\verseline
That shows no rank but \verseline
Obligation
\end{stanza}

\begin{stanza}
Till that tender monster my blue eye\verseline 
Roams in sudden streets of light\verseline
Cutting shadows down to make soft ice\verseline
And turning rays away\verseline
Back into ink
\end{stanza}

\begin{stanza}
For when we die\verseline
There are no \verseline
Audible prayers any more\verseline
Just darkness full of \verseline
Answer
\end{stanza}
\end{poem}

\newpage
\section{Gently}

\begin{poem}
\begin{stanza}
Your face is a burning window in this rainy night,\verseline
My hands are looking blindly for your name,\verseline
My heart has a mouth for it crammed with honey and shame,\verseline
My voice is a broken commandment.
\end{stanza}

\begin{stanza}
Love's the worthless thing everyone dies for,\verseline
The beggar who rules the world,\verseline
This burning dump whose flames dance forever,\verseline
Red slaves always dying chained to their dance.
\end{stanza}

\begin{stanza}
Love, the destitute emptiness,\verseline
Love, the riddle crying for its lost voice,\verseline
Love, the enemy, love, the fool,\verseline
Love, the rebel king, lowering himself to serve everything.
\end{stanza}

\begin{stanza}
The older I am the emptier I become,\verseline
Full of trembling air,\verseline
I, the mere echo of my broken vows,\verseline
Words spilled everywhere.
\end{stanza}

\begin{stanza}
But love's still the song at the end of the tongue,\verseline
Syllables dancing naked and perfumed.\verseline
And love is the war at the end of the world,\verseline
The warrior loving the warriors.
\end{stanza}

\begin{stanza}
The drowned book will swallow up its sea,\verseline
Dead ink reanimate dead mouths.\verseline
The mirror will shatter its diamond,\verseline
And the Sun will part the sky.
\end{stanza}

\begin{stanza}
The kiss that strips away the masks of the face,\verseline
The touch that puts all the useless things down at the end of the world,
\end{stanza}

\begin{stanza}
Gently\verseline
Will touch me.	
\end{stanza}
\attribution{January 14...June 1, 1989}
\end{poem}


\newpage
\section{Black Scum in a Silver Cup}

\begin{poem}
\begin{stanza}
Going to share my life with you.\verseline
Here's my life, black scum in a silver cup, blood that's not half so clean as the pretty blood of Jesus.\verseline
If you want to share my life, you have to drink this cup, just as Jesus drank it down, just as I tell myself I too some day must drink, I really must force myself to share your black life.\verseline
And I will, for Jesus is helping me drink, he is standing beside me and inside me, his hands guide my hand to the blinding cup.\verseline
Be like Jesus to me, drink my cup, help me be like Jesus to you, help me drink your cup, there is only one cup.\verseline
In this cup there is reflected only a whipped face, for the veil was torn when the cup was poured, and God revealed a human face poured out.\verseline
Going to share my life with you, in the human face of God poured out, the bitter wine of the twice-born.\verseline
Then we will be able to sing the song we are already whispering, and how crippled we are will become a sort of dance, a sort of rejoicing.\verseline
There is nothing that will make you more drunk that this cup of black scum.\verseline
You could marry a cloud covering the Sun, a woman with no brain, a man with no legs, a child with a crippled canary.\verseline
Maybe you're thinking you'll never get married, but drink this cup with me anyway.
\end{stanza}
\attribution{December 16, 1988...June 1, 1989}
\end{poem}


\newpage
\section{Evil Wing}

\begin{poem}
\begin{stanza}
The evil wing gives off a terrible negative heat, wills not to be known, pushes away the inquiring will with sick thoughts, a cloud of evil.\verseline
It would kill anyone to remain peaceful within me, larval, growing, dominant, asleep.\verseline
Nobody understands what it hums backwards in music, nobody understands what it paints in shadows, nobody understands its peaceful pain, nobody understands how it has always been rejected, nobody dreams the black sugar of its shame.\verseline
I will kill it by loving it, by bathing it with my tears, by feeding it my blood, by unfolding its bitter red wing in floods of clear light.\verseline
I will kill it by holding it to my chest and climbing into the arms of Christ, the bright lullaby of the nails, the white wings of the ever-folding cross.\verseline
I will kill it by giving it life, unconditional, stark as a spring on the Moon, foaming, dreaming of beer, of salmon swimming, of garden parties behind Lunar houses where nobody understands my Earthly tongue.\verseline
I will kill it before it kills me, and I will give birth, and I will pull myself out of my own side, and my moon will be astonished at the sudden birth of its Sun.
\end{stanza}
\attribution{December 17, 1988...June 1, 1989}
\end{poem}

\newpage
\section{Horn I Sweep}

\begin{poem}
\begin{stanza}
I am free in language as a tart. \verseline
I have a horn, I blow it to blow down your walls. \verseline
I blow, I blow down the walls of your sad closets, \verseline
Your wardrobe of guarded and desired dreams, \verseline
Your negligee of fog and childhood, your rocking horse desire. \verseline
I blow my horn free as a tart and I blow mean, \verseline
I shatter fogs with the hammer of memory and release black steam. \verseline
I blow in sleep with a circular motion, I am the motion, I am the red and infantile commotion. \verseline
I am bold and blue, I drink down music like air, I blow to describe you, to explain you, to shine on you and from inside you. \verseline
I have a hat I put on my horn to play a prank with time. \verseline
I dance back and forth with the enemy, the idiot, the crime. \verseline
I have the radiant closet of the dreaming bed, I find the skin in memory, I find the sin I sleep in, I awaken myself and I dance heavily and proudly on the drum of sleep with my sleeping giant. \verseline
I take up this horn and I sweep out the black tree of dreams, I sweep out stars with its sound, down fall shining cities, memories. \verseline
I get older and older, I wax musical in the shining moonlike bell of my blowing.  I am older with power, and I remember freely, without a hold on you, no price, just the memory of you without a name.  But your name, your precious name, three notes I blow from the touch of your belly. \verseline
I find where you live without knowing you live, and I tenderly play you a fence, a hedge of law, a tower of reason, a burning sword of shame to protect that lawless innocence you forgot, your dreaming, thoughtful, shining spring.
\end{stanza}
\attribution{October 20, 1988...June 3, 1989}
\end{poem}

\newpage
\section{Idiot Perfect}

\begin{poem}
\begin{stanza}
The future is idiot perfect, a woman with red horns. \verseline
She dances with her tail on the glass table of sleep. \verseline
The future, idiot perfect, scorns us... \verseline
All machinery making flour is out of control. \verseline
Drowning in cake, you dream the golden barn, \verseline
Its green doors, its spring breeze \verseline
Touching along your forearm. \verseline
There one speaks the tongue void as a cannon \verseline
Wailing.  You're lost, a ship in time --- \verseline
Black as a sea, your hollow boat \verseline
A sun-hidden farm.
\end{stanza}
\attribution{September 13, 1988...June 1, 1989}
\end{poem}

\newpage
\section{Armor of Memory}

\begin{poem}
\begin{stanza}
Trees are black fans, fly wing lace, whiskers, scratches on the purple armor of memory. \verseline
Cornices and roofs of cardboard bow against the overthundering ship of the white moon sailing. \verseline
Now is the time of yellow peeks glowing, the old wood of hidden balconies, unworkable doors. \verseline
Think of spelling, the ledges of touch dust, the sound of skin writing lonely letters up the smell of linoleum stairs. \verseline
Now your disheveling gold eyes are seas with two small black boats, I will drown in such bottomless boats. \verseline
The water of twilight pours in the windows, \verseline
Invisibly drowning your silent battle of books along the wall, your neglected plants, your clear distant stars. \verseline
The drowning piano marches through the water of dreams, the fate-engine spews out its bright salts as it sinks. \verseline
And my whole hull will split in you.  I have always longed to spill this treasure, \verseline
This golden treasure to light up your black depths.
\end{stanza}
\attribution{November 18, 1988...June 1, 1989}
\end{poem}

\newpage
\section{Judas Come Back}

\begin{poem}
\begin{stanza}
The twelfth hidden one \verseline
Is Judas, remembering.
\end{stanza}

\begin{stanza}
His train of nights is a strange bed, \verseline
The smell of the sea from open windows, bad lights.
\end{stanza}

\begin{stanza}
They flash on and off on his open shirt ---  \verseline
He touches the scar that circles his neck, wondering.
\end{stanza}

\begin{stanza}
Nobody ever lived except in a body, \verseline
Yet its wounds are always deeper than flesh.
\end{stanza}

\begin{stanza}
And he'll have no more children, though he wishes, \verseline
Even hides the book he writes his wars down in.
\end{stanza}

\begin{stanza}
He goes out.  There are those park stairs again, \verseline
The trees tossing their bronze tops above crickets.
\end{stanza}

\begin{stanza}
He thought he'd done everything, that he was everything, \verseline
But there is always more, always.  
\end{stanza}

\begin{stanza}
This is the mill stone \verseline
That no one sees.
\end{stanza}
\attribution{September 10, 15, 17, 1988, February 5, April 26, 1989}
\end{poem}

\newpage
\section{Conjuration}

\begin{poem}
\begin{stanza}
I dance on the dangerous black floor of your eyes\verseline
As you lie dreaming, looking up at your blank white ceiling.\verseline
Clouds of desire and of forgetfulness drift across\verseline
You like arms playing the piano more slowly than time.\verseline
Tango your rolled up stocking showing, tango my black Fedora,\verseline
Tango cloudy scuffed wax with steel tap shoes.
\end{stanza}

\begin{stanza}
In evil magic, against a turquoise wall, with brassy music,\verseline
We perform the dance of conjuration.  (Without a fortune\verseline
And disinherited, I applied myself to learn the keys\verseline
And the names of the objects, the roots, the desires.)
\end{stanza}

\begin{stanza}
That is why I dance, proudly, patched, unshaven, against a\verseline
Turquoise wall, and why you dance with me, for I have your name\verseline
In its crib of white lace, its crib of white lace.
\end{stanza}

\begin{stanza}
I understand so well how little we may have each other\verseline
That we may have each other so much more than the others.
\end{stanza}
\attribution{October 8, 1988...June 3, 1989}
\end{poem}

\newpage
\section{Farmington Bay}

\begin{poem}
\begin{stanza}
I am the least possible woman, I am the simplest man.\verseline
Since I am an atom, therefore I create a field.\verseline
I am an atomic clock, a clock one atom wide, there is only room for one feeling, a feeling the color of light.
\end{stanza}

\begin{stanza}
With brush of music to paint the sky of sound, where ducks fly drums across the green, and the sun goes down the mouth of the black violin.\verseline
In a wooden house you make a wooden house to have a wooden house in case it rains.  It rains memories of a face.  The face is a sweet place to trespass in the underside of things.\verseline
In a wooden house you make a wooden house to have a wooden house for when it burns.  It burns in china organs and it burns in paper reeds.  I lean my shotgun by the lamp and I know how you sound.\verseline
So many birds in your thighs, the rope of your hair to take the sky.
\end{stanza}
\end{poem}

\newpage
\section{Lictor}

\begin{poem}
\begin{stanza}
Wandering in this toothpick town with anvil feet and banjo,\verseline
I'm a cardboard cutout animated by relays and solenoids.\verseline
But my eyes are always shooting for the flesh beyond the fence.
\end{stanza}

\begin{stanza}
With every shot a puff of dust from the dusty old bed\verseline
Where a shell of lace the size and color of a leaf\verseline
Just born tries to hide your ancient face.
\end{stanza}

\begin{stanza}
It's the wrong size\verseline
For the world but I'm adjusting the world to fit you.
\end{stanza}

\begin{stanza}
With every step I split in two in mind as well as body,\verseline
I go to war with armies of forgotten mes and win and lose forever\verseline
But every flag and every torch reminds me of when
\end{stanza}

\begin{stanza}
I was whole, when I could touch you with my eyes,\verseline
And the rain's dream defined a roof of glass\verseline
For the sleeping body of the fog --- just a little smile please ---
\end{stanza}

\begin{stanza}
You and I rolling over like ocean liners burning at sea,\verseline
Flames running out over the black water, up and down the waves.
\end{stanza}

\begin{stanza}
All's fair in love and war:  therefore I make this language,\verseline
Itself a spy, itself a message, itself my general strategy.\verseline
With enough rhythm it will get a body together, and
\end{stanza}

\begin{stanza}
Coalesce right out of ash, colored fog with arms held out in front.\verseline
All over my linen wrappings dance black insect letters:\verseline
Love letters from forever.  Do not be afraid.
\end{stanza}

\begin{stanza}
Forever is the sound of one word laughing,\verseline
Forever is the sound of every lamp going out.
\end{stanza}

\begin{stanza}
So unroll my head to see if grey marmalade\verseline
Wrapped in tissue of bone can think of a stone\verseline
Clear enough to hold the sun of your memory.
\end{stanza}

\begin{stanza}
Orange tinge on cloud bellies,\verseline
On the ceiling of a room,\verseline
On white skin rolling towards me,
\end{stanza}

\begin{stanza}
Your flag \verseline
Of surrender, my vacant field
\end{stanza}

\begin{stanza}
Of what I will try repeatedly to name\verseline
In a way that fireworks boxing in purple clouds\verseline
Only vaguely indicate.
\end{stanza}

\begin{stanza}
Target within arrow,\verseline
Bird bringing cage tiny silver bell in beak,\verseline
My heart on a string, clotted carnelian.
\end{stanza}
\end{poem}

\newpage
\section{Long House}

\begin{poem}
\begin{stanza}
There was a block of houses, or a long apartment house, I think it was on Third South west of West Temple, in Salt Lake City.\verseline
I think Indians and Gypsies lived there.  I saw women sitting in white iron chairs with big white arms and white print dresses but their skin was dark their hair was dark.  I saw old brick crumbling I saw tarpaper roofs I saw chimneys against a dark blue sky I saw yellow windows I saw sprinklers arcing on the lawn I saw bare bulbs inside the yellow windows I saw yellow blinds half pulled down.  I wanted to live there.  I wanted to smell the sprinkler on the cool night lawn of summer and I wanted to go up steep wooden stairs and into a room with a naked bulb and a woman different from any woman I'd ever met.  Or no woman.  Crickets outside.  Railroads.\verseline
I wanted to be able to say I had lived four thousand years.  I wanted to be able to say wooden stairway.  I wanted to be able to say naked bulb.  I wanted to be able to say my mother's name.\verseline
Smell of sprinkled fresh cut lawns in summertime.\verseline
Sound of radio from an open car door.\verseline
Crickets singing in the cottonwoods.\verseline
Railroads.\verseline
Students from China men from Basque young men from Cedar City.\verseline
Sunrise Cafe.\verseline
Acid drifter don't run away with my sister.\verseline
Oklahoma Prison return address love letter blue paper God save us God bless.
\end{stanza}

\begin{stanza}
August night sky blue pit seething cottonwoods and elms.\verseline
Smell of sprinklers hot asphalt sound of television and sticky rubber.\verseline
Across the hot and dusty tracks,\verseline
On the drunk side of town
\end{stanza}

\begin{stanza}
I do not care if you will kiss me.\verseline
I do not care if you will suck my cock.\verseline
I do not care if you will remember me.\verseline
I will talk, I will strip in your bed, I will go away, I will do none of these things.
\end{stanza}

\begin{stanza}
Old cotton undershirt faded not quite white\verseline
Smell of linoleum cricket song\verseline
Skin washed but still a little sweaty\verseline
Lift your arms above your head
\end{stanza}

\begin{stanza}
Every history every morality\verseline
Cup of tea on kitchen table\verseline
Bulb light on the painted white\verseline
Eyes in eyes.
\end{stanza}

\begin{stanza}
Smell of sprinklers on dusty grass\verseline
Smell of cottonwoods exhaling into evening\verseline
Smell of gas burning across the junkyard\verseline
Smell of kerosene from the jets above
\end{stanza}

\begin{stanza}
Row house under diseased elms\verseline
Saturday night August sprinklers going\verseline
Windows without curtains lamps without shades\verseline
White iron chair on the porch without a roof\verseline
Brick without color brick skin brick soul
\end{stanza}

\begin{stanza}
Bed without a blanket\verseline
Dresser without a mirror\verseline
Lift your arms above your shoulders\verseline
Show me
\end{stanza}

\begin{stanza}
I do not drink but I am drunken\verseline
I speak without language\verseline
I am so other\verseline
I am in you
\end{stanza}

\begin{stanza}
All the memorized positions are useless\verseline
The springs are rusty and they sing\verseline
I cannot make a word mean one thing\verseline
I am saying I love you
\end{stanza}

\begin{stanza}
Force what together\verseline
The raised rings of wood on the dresser top
\end{stanza}

\begin{stanza}
In this attic the sound of the radiator\verseline
Lift your arms above your shoulders\verseline
Cotton lifts above your breasts\verseline
Bright snow outside
\end{stanza}

\begin{stanza}
There is no need for unity\verseline
Relax you too will die\verseline
The circle will be within you\verseline
Ripples in the darkness silver as Sun
\end{stanza}

\begin{stanza}
Someone says ``I am''\verseline
And you are.%
\end{stanza}
\attribution{}
\end{poem}

\newpage
\section{Love's Laboratory}

\begin{poem}
\begin{stanza}
In the blackest room of all I stand,\verseline
I am the darkened mirror.
\end{stanza}

\begin{stanza}
But as God approaches me\verseline
My silver fog clears suddenly.
\end{stanza}

\begin{stanza}
For she is beckoning\verseline
With her golden candlestick.
\end{stanza}

\begin{stanza}
And the candle is a figurine of me,\verseline
The wick of my head's aflame,
\end{stanza}

\begin{stanza}
The wax of my flesh \verseline
Runs down her hand.
\end{stanza}

\begin{stanza}
With her other hand she opens the door.\verseline
Night's black cardboard falls gently over.
\end{stanza}

\begin{stanza}
It is day.\verseline
Crickets sound in the dry grass.
\end{stanza}

\begin{stanza}
Above the railroad ravine rise all\verseline
The tar-black turrets and chimneys faintly smoking.
\end{stanza}

\begin{center}\rule[3pt]{2in}{0.5pt}\end{center}

\begin{stanza}
So I found my Indian apartment\verseline
Where there are no more Indians,\verseline
On the dry hill over the old library\verseline
Where bums pass out on the tables muttering
\end{stanza}

\begin{stanza}
With the rot of dreams and broken names still crammed in their mouths.\verseline
And I catch them bowing under that weight\verseline
With the scorpion splendid in my plastic tieclasp\verseline
And the parakeet's bright cage on the windowledge in the sun high above me.
\end{stanza}

\begin{center}\rule[3pt]{2in}{0.5pt}\end{center}

\begin{stanza}
And I drive a silent cab in Hollywood.\verseline
Even the Scientologists never speak to me,\verseline
Though I know all the hotels and I know all the hills.
\end{stanza}

\begin{stanza}
For L.A. is still a walking town,\verseline
There are stairs full of dog piss between the dark junipers \verseline
Down the steep sides to Chinatown.
\end{stanza}

\begin{stanza}
And the building on the hill by the library\verseline
Has a glass scallop awning and wavy glass doors\verseline
That swing on grimy hallways that swim in yellow glare
\end{stanza}

\begin{stanza}
To the prow of the corner room where I sleep and sail \verseline
The sea of lights below me in the Valley of Angels\verseline
On a breeze of bougainvillea and exhaust and raw gasoline.
\end{stanza}

\begin{stanza}
And the red and blue and green eyelashes stroke down \verseline
The soft black cheeks of the streets.
\end{stanza}

\begin{center}\rule[3pt]{2in}{0.5pt}\end{center}

\begin{stanza}
These days I make love alone \verseline
Between sleepless white walls\verseline
With a name that does not love me.\verseline
Name, where did you come from, name, how do you live?
\end{stanza}

\begin{stanza}
Name with black roots searching my heart,\verseline
Name with the breath flower stolen and given,\verseline
Given and stolen, Oh God\verseline
Help me live and not die.
\end{stanza}

\begin{stanza}
For I cannot stand against memory \verseline
And I have no magic against the girl \verseline
Who invades my dreams, her face, \verseline
Her sweet narrow hands.
\end{stanza}

\begin{stanza}
She turns away from me,\verseline
She is milk and honey, marble and bronze\verseline
Looking down sunlit\verseline
Steps.
\end{stanza}

\begin{stanza}
Now I am only a broken mirror and the bodiless hand that always opens this door.\verseline
From the black cabinet \verseline
It seems I must always take my doll of flame,\verseline
My golden picture of eyes.
\end{stanza}

\begin{center}\rule[3pt]{2in}{0.5pt}\end{center}

\begin{stanza}
But when she comes at last to me, it is and is not she.\verseline
My solitary cell has become a laboratory\verseline
Where someone larger than I ever knew \verseline
Is working deep within me.
\end{stanza}

\begin{stanza}
For love has a laboratory abattoir\verseline
Where she takes your hair\verseline
Apart and the air is bright\verseline
With knives and lights.
\end{stanza}

\begin{stanza}
Her metal finger \verseline
In your silken brain\verseline
Stirs the dreams up \verseline
And the pain.
\end{stanza}

\begin{stanza}
Then on its column of steel her table tilts\verseline
Till the dead body spills \verseline
Onto black and white tiles\verseline
For love to dance with in high heels 
\end{stanza}

\begin{stanza}
And nothing else\verseline
Till love has done her dance and gone.
\end{stanza}

\begin{center}\rule[3pt]{2in}{0.5pt}\end{center}

\begin{stanza}
And even in death 
\end{stanza}

\begin{stanza}
I am being experimented upon by love.\verseline
The electrodes glitter, their sharp points needle my skin,\verseline
The electrical wheels begin to spin, faster and faster.\verseline
Love jumps up and down, chattering to her familiars,\verseline
The lightning she loves is gathering.\verseline
In her black and white nest of shadows and apparatus\verseline
Love will at last reveal her ghastly ancientness.
\end{stanza}

\begin{stanza}
So why should love fling away her white coat,\verseline
Why should love disdain the blood that covers her,\verseline
Why should love sing as she operates,\verseline
Why should love be so beautiful?\verseline
Why should love sew my parts back mixed\verseline
With entrails and limbs of creatures I hated,\verseline
Why should she shock me to life I can't live?
\end{stanza}

\begin{center}\rule[3pt]{2in}{0.5pt}\end{center}

\begin{stanza}
Love's laboratory has a golden key, fashioned of tears refined incessently, the tear within the tears, the tear of repentance, the tear of glory.
\end{stanza}

\begin{stanza}
In love's laboratory there are windows of snow, servants asleep in sunlit bottles, a teakettle nosecone toppling from the stove.
\end{stanza}

\begin{stanza}
In love's laboratory the instruments are shrieks on a black scale.
\end{stanza}

\begin{stanza}
In the alembic of emptiness, as a torch of black death, I acknowledge my sin, I bewail my deep malice.
\end{stanza}

\begin{stanza}
I sleep in the red oven, let the fat run off me and burn in the flames, become a black rock with a carved smile.
\end{stanza}

\begin{stanza}
And in love's laboratory I am crushed, in the crowded pages of her dictionary, by her thousand tons of history, in the diamond jaws of mystery, till the weeping cloud of me becomes a stone
\end{stanza}

\begin{stanza}
And I reflect\verseline
The sun alone.
\end{stanza}
\attribution{}
\end{poem}

\section{When I looked into the pit...}

\begin{poem}
\begin{stanza}
When I looked into the pit I saw my name weeping in captivity, but your name was in the pit also, silent, dark.\verseline
Your name would not permit me to escape until I felt the sorrow of my captor.  Your name was an infinity of sorrow, dead people in ditches, abortions, easy excuses.  Your name was an infinity of pity, even for evil which delights in evil.\verseline
Your name was dark to me, but there is no comfort anywhere else, so I took refuge in it even though it destroyed me.
\end{stanza}

\begin{stanza}
You rejoice with bankers, and exult with tank commanders, and drink beer with torturers.\verseline
You are silent with the silent ones, you weep inside the dead child's stopped heart.\verseline
You took the burden of human happiness upon yourself.  You experiment with it like a child.
\end{stanza}

\begin{stanza}
One word and winter would become spring, one word and time would freeze or run backwards.\verseline
Thieves steal from you to give to you, for you are begging for love in their thievish hearts.\verseline
Murderers trample on your breaking ribs to glorify you in the dead churches of their own hearts.
\end{stanza}

\begin{stanza}
Whole cities long to forget you, your name is not spoken there, it is put to uses of murder, burn them stone them shoot them, your name the rag of evil, your name the humility of death, your name that refuses to leave anyone, the towers of ignorance that stand because they stand on your name.
\end{stanza}
\attribution{June 12, 1990}
\end{poem}

\newpage
\section{Judge of Sleep}

\begin{poem}
\begin{stanza}
God really did appear on the side of men and women, the side of themselves they call nonbeing because they never dare look into it.\verseline
They're always ending the world in there.  I'm telling it to you straight as I can.  You're always ending the world in there.  I'm always ending in the world in there, and Jesus is always wrestling with Jacob and me in there.\verseline
You judge, you judge of sleep.  You executioner, you tender, tender executioner with your masklss mask.\verseline
My death insists on a second birth, and I will drag myself out of my own side with a chain of spit and blood.\verseline
I'm not self-willed, the self that's dragging me out of myself is the dark self in my dark side.\verseline
Even if I go crazy, I'll go crazy wrestling with love.\verseline
I won't go crazy at all, for God is on my side, hauling, hauling like a baby hauls.  It's Christmas, and the baby Jesus is hauling me through a fragrant world of straw.\verseline
Jesus and I, we remember you, we remember your world that comes before and after sleep, our tiny fists are curled in remembrance of the world that comes before and after.
\end{stanza}
\attribution{December 16, 1988...June 1, 1989}
\end{poem}

	
\end{document} 
