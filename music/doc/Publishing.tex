\documentclass[english,11pt,letterpaper,onecolumn]{scrartcl}

%\usepackage[utf8]{inputenc}
\usepackage{babel}
\usepackage{mathptmx}
% Extra leading.
\renewcommand{\baselinestretch}{1.125}
\usepackage{tocloft}
\usepackage{fancyhdr}
%\usepackage{scrlayer-scrpage}
\usepackage{ifthen}
\usepackage{keyval}
\usepackage{geometry}
\usepackage{url}
\usepackage{calc}
\usepackage{array}
\usepackage{graphicx}
\usepackage{color}
\usepackage{listings}
\usepackage{supertabular}
%\usepackage{scrpage2}
\usepackage[pdftex,
            pagebackref=true,
            colorlinks=true,
            linkcolor=blue,
            pdfpagelabels,
            pdfstartpage=3
           ]{hyperref}
\usepackage{poemscol}
\global\verselinenumbersfalse
\makeindex
\definecolor{LstColor}{cmyk}{0.1,0.1,0,0.025} 
\setcounter{tocdepth}{9}
\newcommand\floor[1]{\lfloor#1\rfloor}
\newcommand\ceil[1]{\lceil#1\rceil}
\begin{document}

\title{Publishing My Works}
\author{Michael Gogins \\ \texttt{michael.gogins@gmail.com}}
\maketitle
%\pagestyle{scrheadings}

%\lohead{Publishing}

I have produced a number of works that deserve to be published. These include albums of fixed media electroacoustic music as well as works performed in festivals and conferences, or prepared for such performance. There also are some writings and some videos, and may be other things in the future, especially photographs. I need to move ahead with publishing these works. For the time being, that will mean uploading my works to CDBaby or YouTube, with links to Tumblr.


The major issue is that I have published two albums with CDBaby and plan to publish more, but there is some conflict between CDBaby which requires to be the exclusive digital distributor, and my other channels for publication such as SoundCloud, Tumblr, and YouTube. These are the steps:

\begin{enumerate}
\item I have verified my accounts with the United States Copyright Office, ASCAP, YouTube, and AdSense. I still need to register with AdRev.
\item Of course, identify which works I wish to publish and ensure they are finished! I have written a Python script \texttt{create\_complete\_playlist.py} that searches my hard disk for soundfiles and compiles both a Windows playlist and a tab-delimited table of pieces.
\item Make sure the finished works are, where possible, tagged with name, title, copyright notice, and other data. If possible, I will create a script for this. My earlier approach of doing this at the time of rendering is no longer feasible because I now have too many ways of rendering works.
\item Register those works with the United States Copyright Office at: \url{https://eco.copyright.gov/eService_enu/}. I should copyright  the ``phonorecord'' as registration type SR as well as, at the same time, registering for composition and arrangement copyrights. It is not necessary to deposit a score, the phonorecord suffices.
\item Register the same works with ASCAP at: \url{https://www.ascap.com/member-access/}. 
\item For works that I have not yet published:
\begin{enumerate}
	\item Upload works to CDBaby. CDBaby will then promote digital distribution, upload to ``Michael Gogins - Topic'' channel on YouTube, and upload to Amazon.
	\item Link to works on CDBaby from Tumblr, Facebook, etc.
\end{enumerate}
\item For works that I have published somewhere but not with CDBaby:
\begin{enumerate}
	\item Register the works with a content identification service according to ASCAP's instructions: \url{https://www.ascap.com/help/royalties-and-payment/make-money-youtube}. 
	\item Convert the works to WMV files for upload to my ``Michael Gogins'' channel, monetized, on YouTube: \url{https://support.google.com/youtube/answer/1696878?hl=en}. I will create a script or template or something for making this prettier and easier. Included information should be simply a sox sonogram, labeled with the title of the work and a dated copyright notice, perhaps even a few words of program notes and a link to my blog.
\end{enumerate}
\end{enumerate}

A mistake I made in the past was deleting my master list. I will just keep adding it to and do a minimal update of status, e.g. published to where, worth publishing, tagged. And the default title is the filename! I should indeed make sure I can find the \emph{Publish} code and soundfiles, but I should \emph{not} worry too much or spend too much time tracking the other stuff.

I will continue looking for other ways to publish music that will (a) protect my rights, (b) get me a bigger and better audience, and (c) make me money.

I have been urging Heidi to get her demo, which is quite good, online somehow. She needs to get the permissions from the performers before she can follow my path. I will identify just what needs to be done. If that is too much, perhaps she can make a new demo where she plays, or where she gets the paperwork done for her players. I will remind her yet again....

\subsection*{2016 December 5}

I have finished my spreadsheet and am almost done listening to all the pieces, but it is clear that this is only a first step in evaluating these works. I will need to group them into albums and then listen to many of them again. For example, there are way too many similar ``IFS'' or ``sound fractal'' pieces. There should be an album of these containing only the best ones, even if it is short or long.

Two preliminary thoughts. The good news: more of these are listenable or even attractive than I thought! The bad news: too much stuff is on the borderline of good and suffers from being too unvarying, or from some fatal and probably avoidable flaw somewhere along the way, or both. I obviously must, must, \emph{must} police the clinkers much more sedulously.

\subsection*{2016 December 6}

I have sifted through the works and ended up with 4.8 hours worth of ``publishable'' stuff. Probably only some of this is really worth publishing. I am throwing stuff out that is not good enough.

\subsubsection*{Sound Fractals}

\begin{supertabular}{|c|l|r|l|}
\hline 
Track & Title & Time & Notes \\ 
\hline 
01 & soundifs.py.wav & 13:14 & Suite of iterated function systems \\&&&translated to one soundfile.\\
02 & Apophysis-091206-3.wav & 2:02 & Apophysis-generated fractal flame \\&&&translated to a soundfile.\\
03 & ifs-n.py.wav & 5:02 &Iterated function system translated to a \\&&&soundfile using granular synthesis in Python.\\
04 & ifs-i.py.wav & 5:02 &Iterated function system translated to a \\&&&soundfile using granular synthesis in Python.\\
05 & ifs-2008-01-23-a.py.stereo.wav & 5:02 & Iterated function system translated to a \\&&&soundfile using granular synthesis in Python.\\
\hline 
Total   &  & 30:22 & \\
\hline 
\end{supertabular} 

\subsubsection*{The Garden of the Hesperides }

\begin{supertabular}{|c|l|r|l|}
\hline 
Track & Title & Time & Notes \\ 
\hline 
00 & Hesperide.wav & 7:20 & A daughter of Atlas. \\&&&Recurrent iterated function system with chord spaces, \\&&&rendered using HTML5, Csound, and LuaJIT.\\
00 & Ladon-1-d.wav & 8:22 & The dragon guarding the daughters of Atlas and the apples of immortality. \\&&&Recurrent iterated function system with chord spaces, \\&&&rendered using HTML5, Csound, and LuaJIT.\\
00 & Telamon.wav & 8:20 & Herakles the Enduring, thief of the apples of immortality. \\&&&Recurrent iterated function system with chord spaces, \\&&&rendered using HTML5, Csound, and LuaJIT.\\
\hline 
Total   &  &  & \\
\hline 
\end{supertabular} 

\subsubsection*{Slow and Other Changes}

\begin{supertabular}{|c|l|r|l|}
\hline 
Track & Title & Time & Notes \\ 
\hline 
00 & drone.py.wav & 9:49& \\
00 & Drone-VIII-d.wav & 8:04 & \\
00 & Mountain\_Drone.norm.wav & 10:10 & \\
00 & Sevier.wav & 9:00 & \\

00 & mkg-2008-06-15-d.wav & 0:00 & \\
00 & mkg-2008-06-20-a-1.wav & 0:00 & \\
00 & mkg-2008-09-16-c.wav & 0:00 & \\
00 & mkg-2008-12-25-b.wav & 0:00 & \\
00 & mkg-2009-01-10-b.wav & 0:00 & \\
00 & mkg-2009-01-10-c.wav & 0:00 & \\
00 & mkg-2007-01-20-b.py.wav & 0:00 & \\
00 & mkg-2009-02-03-d.wav & 0:00 & \\
00 & mkg-2009-09-14-o-1.py.wav & 0:00 & \\
00 & mkg-2009-09-14-r.py.wav & 0:00 & \\
00 & Yellow\_Leaves.4.lua.norm.wav & 0:00 & \\
00 & mkg-2010-03-16-e-1.py.wav & 0:00 & \\
00 & mkg-2008-06-15-d-3.wav & 0:00 & \\
00 & Aeolus\_Study.3.native\_reverb.lua.norm.wav & 0:00 & \\
00 & Untouching-1.norm.wav & 0:00 & \\
00 & Blue\_Leaves\.norm.wav & 0:00 & \\
00 & Blue\_LeavesBlue\_Leaves\_1.wav & 4:06 & \\
00 & Semblance 09 chaotic dynamical system.wav & 0:00 & \\



\hline 
\end{supertabular} 

\subsection*{2016 December 7}

\subsubsection*{Tags and Such}

Data and metadata required for each of my pieces. As much as possible, this data will be created by a script based on the audio soundfiles and a text file of other inputs.

\begin{itemize}
	\item Master soundfile of high-resolution audio, tagged as follows:
	\begin{enumerate}
		\item ID3v2 header and the following frames.
		\item APIC embedded picture, should be MIME type \texttt{image/png}.
		\item COMM comments (program notes, may contain embedded newlines).
		\item TALB album title.
		\item TCOM composer (Michael Gogins).
		\item TOWN owner (Michael Gogins).
		\item TCOP copyright message (Coyright (C) 9999 by Michael Gogins).
		\item TPUB publisher (Irreducible Productions).
		\item TDAT date of recording (DDMM).
		\item TIT2 title (often the same as the original master filename).
		\item TOFN original filename.
		\item TRCK track number (if in an album).
		\item TYER year of recording (YYYY).
	\end{enumerate}
	\item MP3 file with the same tags.
	\item Source code file or files.
	\item MIDI file (where possible) with the same tags.
	\item PDF of score from MIDI file (if MIDI file exists). The score should have a cover page with the same image as the master soundfile and with metadata deriving from the master soundfile.
	\item WMV file with inputs:
	\begin{enumerate}
		\item PDF
		\item Master soundfile encoded.

	\end{enumerate}
\end{itemize}




\end{document} 
