\documentclass[english,11pt,letterpaper,onecolumn]{scrartcl}

%\usepackage[utf8]{inputenc}
\usepackage{babel}
\usepackage{mathptmx}
% Extra leading.
\renewcommand{\baselinestretch}{1.125}
\usepackage{tocloft}
\usepackage{fancyhdr}
%\usepackage{scrlayer-scrpage}
\usepackage{ifthen}
\usepackage{keyval}
\usepackage{geometry}
\usepackage{url}
\usepackage{calc}
\usepackage{array}
\usepackage{graphicx}
\usepackage{color}
\usepackage{listings}
\usepackage{supertabular}
%\usepackage{scrpage2}
\usepackage[pdftex,
            pagebackref=true,
            colorlinks=true,
            linkcolor=blue,
            pdfpagelabels,
            pdfstartpage=3
           ]{hyperref}
\usepackage{poemscol}
\global\verselinenumbersfalse
\makeindex
\definecolor{LstColor}{cmyk}{0.1,0.1,0,0.025} 
\setcounter{tocdepth}{9}
\newcommand\floor[1]{\lfloor#1\rfloor}
\newcommand\ceil[1]{\lceil#1\rceil}
\begin{document}

\title{Publishing My Works}
\author{Michael Gogins \\ \texttt{michael.gogins@gmail.com}}
\maketitle
%\pagestyle{scrheadings}

%\lohead{Publishing}

I have produced a number of works that deserve to be published. They mostly are works of fixed media electroacoustic music performed in festivals and conferences, or prepared for such performance. There also are some writings and some videos, and may be other things in the future, especially photographs. I need to move ahead with publishing these works. For the time being, that will mean uploading my works to YouTube, CDBaby, SoundCloud, my own Web site, and other venues for online distribution. These are the steps:

\begin{enumerate}
\item Of course, identify which works I wish to publish and ensure they are finished! I have written a Python script \texttt{create\_complete\_playlist.py} that searches my hard disk for soundfiles and compiles both a Windows playlist and a tab-delimited table of pieces.
\item Make sure the finished works are, where possible, tagged with name, title, copyright notice, and other data. I will create a script for this.
\item Register those works with the United States Copyright Office at: \url{https://eco.copyright.gov/eService_enu/}. I have made sure I can log in. I should copyright not only the sound recording but also the score, if I can create it. This only makes sense in some cases, but certainly not e.g. for the ``sound fractals'' unless I claim that the score is the program.
\item Register the same works with ASCAP at: \url{https://www.ascap.com/member-access/}. I have made sure I can log in.
\item Create a YouTube channel for myself, when I monetize my uploads I get all the ad revenue directly from YouTube. I have made sure I have a channel and I can log into it. I may need to make sure I have payment arranged.
\item Register the same works with a content identification service according to ASCAP's instructions: \url{https://www.ascap.com/help/royalties-and-payment/make-money-youtube}. CDBaby already does this for me for my CDBaby works. I have made sure I can log in to CDBaby at \url{https://members.cdbaby.com/Login.aspx}. CDBaby is still making me a hundredth of a cent every day or so. And they are paying me! I have verified that the direct deposit is happening. I also opted into their YouTube and other media monetization, even though they take a sizeable cut, because it simplifies my life. Also, CDBaby automatically uploads albums to YouTube and Amazon.
\noindent For works that I already have published, CDBaby might not be appropriate, but I can still put those works on YouTube and try to get my ASCAP cut by using AdRev or whatever.
\item YouTube's instructions for converting soundfiles to WMV files for upload are here: \url{https://support.google.com/youtube/answer/1696878?hl=en}. But CDBaby has done this for me automatically. I will create a script or template or something for making this prettier and easier. Included information should be:
\begin{itemize}
\item Artwork. Might include a sox sonogram.
\item My name.
\item The appropriate URL.
\item Album title, if there is one.
\item Title of the piece.
\item Duration of the piece.
\item Comments, if any.
\end{itemize}
\noindent How much of this can be done in tags? Current standard tags are \url{http://id3.org/id3v2.3.0}.
\end{enumerate}

\noindent In sum: I will split doing this myself directly on my YouTube channel, to doing it through CDBaby, which will take care of all digital distribution including YouTube and Amazon. As far as I can tell it's OK to publish with CDBaby things that already are on my own Web site, but perhaps I should clarify that. The steps are:

\begin{enumerate}
\item Identify on computer and in spreadsheet.
\item Tag. 
\item Apply for copyright. 
\item Register with ASCAP.
\item Add new title to CDBaby or upload WMV to my YouTube channel.
\item Post the link to CDBaby or my YouTube channel on my blog.
\end{enumerate}

A mistake I made in the past was deleting my master list. I will just keep adding it to and do a minimal update of status, e.g. published to where, worth publishing, tagged. And the default title is the filename! I should indeed make sure I can find the \emph{Publish} code and soundfiles, but I should \emph{not} worry too much or spend too much time tracking the other stuff. And I should make sure I am up to date with all the outfits that I need to register with.

I will continue looking for other ways to publish music that will (a) protect my rights, (b) get me a bigger and better audience, and (c) make me money.

I have been urging Heidi to get her demo, which is quite good, online somehow. She needs to get the permissions from the performers before she can follow my path. I will identify just what needs to be done. If that is too much, perhaps she can make a new demo where she plays, or where she gets the paperwork done for her players. I will remind her yet again....

\subsection*{2016 December 5}

I have finished my spreadsheet and am almost done listening to all the pieces, but it is clear that this is only a first step in evaluating these works. I will need to group them into albums and then listen to many of them again. For example, there are way too many similar ``IFS'' or ``sound fractal'' pieces. There should be an album of these containing only the best ones, even if it is short or long.

Two preliminary thoughts. The good news: more of these are listenable or even attractive than I thought! The bad news: too much stuff is on the borderline of good and suffers from being too unvarying, or from some fatal and probably avoidable flaw somewhere along the way, or both. I obviously must, must, \emph{must} police the clinkers much more sedulously.

\subsection*{2016 December 6}

I have sifted through the works and ended up with 4.8 hours worth of ``publishable'' stuff. Probably only some of this is really worth publishing, but it is enough to get started compiling the works into albums. I suppose the best approach is to create albums, allocate the works, and then throw out stuff that's not good enough. I will postpone thinking about what to do about things I have already published.

\subsubsection*{Sound Fractals}

\begin{tabular}{|c|l|r|l|}
	\hline 
	Track & Title & Time & Notes \\ 
	\hline 
	00 & ifs-2008-01-23-a.py.stereo.wav & 0:00 & \\
	00 & ifs-b.py.wav & 0:00 & \\
	00 & ifs-i.py.wav & 0:00 & \\
	00 & ifs-j.py.stereo.wav & 0:00 & \\
	00 & ifs-n.py.wav & 0:00 & \\
	00 & Apophysis-090301-1.wav & 0:00 & \\
	00 & Apophysis-090301-100.grains.wav & 0:00 & \\
	00 & Apophysis-091201-101.wav & 0:00 & \\
	00 & Apophysis-091206-3.wav & 0:00 & \\
	00 & soundifs.py.wav & 0:00 & \\
	\hline 
\end{tabular} 

\subsubsection*{The Garden of the Hesperides}

\begin{tabular}{|c|l|r|l|}
	\hline 
	Track & Title & Time & Notes \\ 
	\hline 
	00 & Hesperide.wav & 7:20 & \\
	00 & Ladon-1-d.wav & 8:22 & \\
	00 & Telamon.wav & 8:20 & \\
	
	00 & Blue\_LeavesBlue\_Leaves\_1.wav & 4:06 & \\
	00 & Semblance 09 chaotic dynamical system.wav & 0:00 & \\
	
	
	
	
	\hline 
\end{tabular} 

\subsubsection*{Slow and Other Changes}

\begin{tabular}{|c|l|r|l|}
	\hline 
	Track & Title & Time & Notes \\ 
	\hline 
	00 & drone.py.wav & 9:49& \\
	00 & Drone-VIII-d.wav & 8:04 & \\
	00 & Mountain\_Drone.norm.wav & 10:10 & \\
	00 & Sevier.wav & 9:00 & \\
	
	00 & mkg-2008-06-15-d.wav & 0:00 & \\
	00 & mkg-2008-06-20-a-1.wav & 0:00 & \\
	00 & mkg-2008-09-16-c.wav & 0:00 & \\
	00 & mkg-2008-12-25-b.wav & 0:00 & \\
	00 & mkg-2009-01-10-b.wav & 0:00 & \\
	00 & mkg-2009-01-10-c.wav & 0:00 & \\
	00 & mkg-2007-01-20-b.py.wav & 0:00 & \\
	00 & mkg-2009-02-03-d.wav & 0:00 & \\
	00 & mkg-2009-09-14-o-1.py.wav & 0:00 & \\
	00 & mkg-2009-09-14-r.py.wav & 0:00 & \\
	00 & Yellow\_Leaves.4.lua.norm.wav & 0:00 & \\
	00 & mkg-2010-03-16-e-1.py.wav & 0:00 & \\
	00 & mkg-2008-06-15-d-3.wav & 0:00 & \\
	00 & Aeolus\_Study.3.native\_reverb.lua.norm.wav & 0:00 & \\
	00 & Untouching-1.norm.wav & 0:00 & \\
	00 & Blue\_Leaves\.norm.wav & 0:00 & \\
	
	
	
	\hline 
\end{tabular} 

\end{document} 
