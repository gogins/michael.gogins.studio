% This tallk should last between 10 and 15 minutes,
% and consist of 10 to 15 slides.
% I should cut back and forth between slides and actual pieces.
% I should leave  a little more than 5 minutes for questions.

\documentclass{beamer}

\mode<presentation>
{
  \usetheme{Warsaw}
  \definecolor{links}{HTML}{2A1B81}
  \hypersetup{colorlinks,linkcolor=,urlcolor=links}
  %\usetheme{Frankfurt}
  \usecolortheme{seagull}
  % or ...
  \setbeamercovered{transparent}
  % or whatever (possibly just delete it)
}

\usepackage[english]{babel}
\usepackage[utf8]{inputenc}
\usepackage{times}
\usepackage[T1]{fontenc}
\usepackage{fancyvrb}
\usepackage{listings}
\usepackage{graphicx}
\usepackage{attachfile}
\usepackage{ifthen}

\newboolean{localPieces} %Declaration, defaults to false
\setboolean{localPieces}{false} %Assignment

\title[cloud-5] % (optional, use only with long paper titles)
{cloud-5: \\\
A System for Composing and Publishing Cloud Music}

\author[Gogins] % (optional, use only with lots of authors)
{Michael Gogins \\ \url{http://michaelgogins.tumblr.com} }
% - Give the names in the same order as the appear in the paper.
% - Use the \inst{?} command only if the authors have different
%   affiliation.

\institute[Irreducible Productions] % (optional, but mostly needed)
{
  Irreducible Productions\\
  New York
}
% - Use the \inst command only if there are several affiliations.
% - Keep it simple, no one is interested in your street address.

\date[18 September 2024] % (optional, should be abbreviation of conference name)
{18 September 2024}
% - Either use conference name or its abbreviation.
% - Not really informative to the audience, more for people (including
%   yourself) who are reading the slides online

\subject{Computer Music}
\expandafter\def\expandafter\insertshorttitle\expandafter{%
    \insertshorttitle\hfill%
    \insertframenumber\,/\,\inserttotalframenumber}
% This is only inserted into the PDF information catalog. Can be left
% out. 
\begin{document}
\lstset{basicstyle=\ttfamily\tiny,commentstyle=\ttfamily\tiny,tabsize=2,breaklines,fontadjust=true,keepspaces=false,fancyvrb=true,showstringspaces=false,moredelim=[is][\textbf]{\\emph\{}{\}}}

\begin{frame}
  \titlepage
\end{frame}

\begin{frame}{Introduction}
\begin{itemize}
\item cloud-5 is a new system for composing, performing, and publishing computer music. 
\item In this talk, I focus on motivations for creating cloud-5.
\item I show how cloud-5 accommodates composing, performing, and publishing in one environment. 
\item Details on the implementation may be found in the paper presented in this conference, and especially in the \href{https://github.com/gogins/cloud-5}{cloud-5 GitHub repository}.
\end{itemize}
\end{frame}

 % \tableofcontents
  % You might wish to add the option [pausesections]

\section{Motivations}

\begin{frame}{Motivations}
\begin{itemize}
\item My primary motivation is to make the kind of music that I would like to hear, but nobody else seems to be making, by means of algorithmic composition and synthesis.
\item As I wander down this thorny path, I find myself wanting to perform some pieces live.
\item I also find myself wanting to make some pieces that play indefinitely, that are "always on."
\item And to generate visuals \emph{from} my pieces, or to transform visuals \emph{into} pieces.
\item And to drastically simplify the publication of my pieces.
\item And to drastically simplify the infrastructure for doing all of this.
\end{itemize}
\end{frame}

\begin{frame}{Publication}
\begin{itemize}
\item The record business is dying...
\item ...but at the same time, it has become easier and easier to publish music on the World Wide Web.
\item Such music can be streamed from files -- or it can be Web pages that actually generate music in real time.
\item Doing pieces as Web pages makes it possible to use all the resources of contemporary Web browsers.
\item Within its security sandbox, \href{https://html5test.co/}{a browser is basically an operating system plus a game engine plus a high-resolution media player}.
\end{itemize}
\end{frame}

\begin{frame}{New possibilities}
\begin{itemize}
\item I call music that is played by Web pages on the World Wide Web \emph{cloud music} because it exists only in the cloud, the omnipresent computing infrastructure of the World Wide Web.
\item I feel it is very important to grasp that cloud music is essentially \emph{a new medium for music} and as such,\emph{ it offers new possibilities of musical expression}.
\begin{itemize}
\item A piece can play forever.
\item A piece can generate visuals, or be generated by visuals.
\item A piece can interact with its listeners.
\item Listeners can use a piece to create new pieces.
\end{itemize}
\end{itemize}
\end{frame}

\begin{frame}{New possibilities}
\begin{itemize}
\item cloud-5 is certainly not the first system that publishes music as Web pages on the World Wide Web:
\begin{itemize}
\item \href{https://gibber.cc/playground/}{Gibber}
\item \href{https://strudel.cc/}{Strudel}
\item \href{https://ide.csound.com/}{Web-IDE}
\item ...and others.
\end{itemize}
\item These systems support multiple users, and provide a playground for shared learning and experimentation..
\item By contrast, cloud-5 is designed to host \emph{permanent} pieces without compromise. Such pieces can be shared or not, at the discretion of the composer.
\item cloud-5 supports algorithmic composition and synthesis at a high level of power, yet a low overhead of development and maintenance.
\end{itemize}
\end{frame}

\section{Design}

\begin{frame}{Design}
\begin{itemize}
\item A musical composition is a single Web page.
\item The score, the orchestra for performing the score, and all required libraries are embedded in that single Web page.
\item The libraries, any musical assets, and all other pre-requisites are \emph{static resources} on the server filesystem.
\item Writing a piece is done using a text editor such as Visual Studio Code.
\item A build system is not used.
\item The version-less nature of Internet protocols and Web browsers means that cloud-5 pieces should last forever.
\end{itemize}
\end{frame}

\section{Implementation}
\begin{frame}{Implementation}
\begin{itemize}
\item cloud-5 is a combination of existing components::
\begin{itemize}
\item The sound processing language Csound, compiled from C to WebAssembly.
\item The live coding system Strudel, written in JavaScript.
\item The algorithmic composition system CsoundAC, compiled from C++ to WebAssembly.
\item GLSL shaders.
\end{itemize}
\item These components all provide JavaScript APIs, and JavaScript in a composition is used to connect and control the components.
\end{itemize}
\end{frame}

\begin{frame}{Composing}
\begin{itemize}
\item A piece is a single Web page constructed from custom HTML elements that are defined in \texttt{cloud-5.js}.
\item A high-level menu system provides the shell for a piece.
\item The design is modular. Components can be omitted or added. A naming convention makes it clear what the composer can add to the shell in order to make an actual piece.
\item An extension for Visual Studio code makes it possible to edit Csound, Strudel, HTML, and JavaScript code in a piece, and immediately serve that piece in a local Web server to play in the composer's Web browser.
\end{itemize}
\end{frame}

\begin{frame}{Publishing}
\begin{itemize}
\item Copy your cloud-5 directory to any Web server. It contains all the libraries and assets that you need.
\item That Web server can be a local server for writing pieces, or a server hosted on the Internet for publishing pieces.
\item The cloud-5 directory can be the server's root directory, or it can be a subdirectory.
\item Musical compositions, of course, are Web pages in the cloud-5 directory.
\item Modify the index in the Web server to list and link to pieces to be published.
\item Period.
\end{itemize}
\end{frame}


\begin{frame}[allowframebreaks]
  \frametitle<presentation>{References}
    
  \begin{thebibliography}{10}
    
  \beamertemplatebookbibitems
  % Start with overview books.

  \bibitem{GBlog} \href{http://michaelgogins.tumblr.com/}{Michael Gogins, blog}.
  
  \bibitem{GGithub} \href{https://github.com/gogins/gogins.github.io}{Michael Gogins. ``Computer Music Resources.''}

  \bibitem{CQT2008} \href{http://www.sciencemag.org/content/320/5874/346.abstract}{Clifton Callender, Ian Quinn, and Dmitri Tymoczko. ``Generalized voice-leading spaces.'' \emph{Science}, 320:346–
348, 2008.}

  \bibitem{G1991} {Michael Gogins. ``How I Became Obsessed with Finding a Mandelbrot Set for Sounds,'' \textbf{\textit{News of Music}} \textbf{13}:129-139.}

  \bibitem{FS2005} \href{http://www.mtosmt.org/issues/mto.05.11.3/mto.05.11.3.fiore_satyendra.pdf}{T.M. Fiore and R. Satyendra. ``Generalized Contextual
Groups.'' \emph{Music Theory Online}, 11(3), 2005}.

  \bibitem{G2006}
    \href{https://www.dropbox.com/s/ztej71n2fbn4tq4/Lindenmayer_Systems_Based_on_Riemannian_Transformations.pdf}{Michael Gogins. ``Score generation in voice-leading
and chord spaces.'' In Georg Essl and Ichiro Fujinaga,
editors, \emph{Proceedings of the 2006 International Computer Music Conference}, San Francisco, California,
2006. International Computer Music Association.}

  \bibitem{T2006} \href{http://www.sciencemag.org/content/313/5783/72.abstract?ijkey=wzKBea3ktKdu2&keytype=ref&siteid=sci}{Dmitri Tymoczko. ``The Geometry of Musical Chords.'' \emph{Science}, 313:72–74, 2006.}

  \end{thebibliography}

\end{frame}

\end{document}


