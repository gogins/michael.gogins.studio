% This tallk should last between 10 and 15 minutes,
% and consist of 10 to 15 slides.
% I should cut back and forth between slides and actual pieces.
% I should leave  a little more than 5 minutes for questions.

\documentclass{beamer}

\mode<presentation>
{
  \usetheme{Warsaw}
  \definecolor{links}{HTML}{2A1B81}
  \hypersetup{colorlinks,linkcolor=,urlcolor=links}
  %\usetheme{Frankfurt}
  \usecolortheme{seagull}
  % or ...
  \setbeamercovered{transparent}
  % or whatever (possibly just delete it)
}

\usepackage[english]{babel}
\usepackage[utf8]{inputenc}
\usepackage{times}
\usepackage[T1]{fontenc}
\usepackage{fancyvrb}
\usepackage{listings}
\usepackage{graphicx}
\usepackage{attachfile}
\usepackage{ifthen}

\newboolean{localPieces} %Declaration, defaults to false
\setboolean{localPieces}{false} %Assignment

\title[cloud-5] % (optional, use only with long paper titles)
{cloud-5: \\\
A System for Composing and Publishing Cloud Music}

\author[Gogins] % (optional, use only with lots of authors)
{Michael Gogins \\ \url{http://michaelgogins.tumblr.com} }
% - Give the names in the same order as the appear in the paper.
% - Use the \inst{?} command only if the authors have different
%   affiliation.

\institute[Irreducible Productions] % (optional, but mostly needed)
{
  Irreducible Productions\\
  New York
}
% - Use the \inst command only if there are several affiliations.
% - Keep it simple, no one is interested in your street address.

\date[18 September 2024] % (optional, should be abbreviation of conference name)
{18 September 2024}
% - Either use conference name or its abbreviation.
% - Not really informative to the audience, more for people (including
%   yourself) who are reading the slides online

\subject{Computer Music}
\expandafter\def\expandafter\insertshorttitle\expandafter{%
    \insertshorttitle\hfill%
    \insertframenumber\,/\,\inserttotalframenumber}
% This is only inserted into the PDF information catalog. Can be left
% out. 
\begin{document}
\lstset{basicstyle=\ttfamily\tiny,commentstyle=\ttfamily\tiny,tabsize=2,breaklines,fontadjust=true,keepspaces=false,fancyvrb=true,showstringspaces=false,moredelim=[is][\textbf]{\\emph\{}{\}}}

\begin{frame}
  \titlepage
\end{frame}

\begin{frame}{Introduction}
\begin{itemize}
\item cloud-5 is a new system for composing, performing, and publishing computer music. 
\item In this talk, I focus on motivations for creating cloud-5.
\item Details on the implementation may be found in the paper presented in this conference, and especially in the \href{https://github.com/gogins/cloud-5}{cloud-5 GitHub repository}.
\item I show how cloud-5 accommodates composing, performing, and publishing in one environment. 
\end{itemize}
\end{frame}

 % \tableofcontents
  % You might wish to add the option [pausesections]

\section{Motivations}

\begin{frame}{Motivations}
\begin{itemize}
\item My \emph{primary} motivation is to make the kind of music that I would like to hear, but nobody else seems to be making, using algorithmic composition and synthesis.
\item As I wander down this thorny path, I find myself wanting to perform some pieces live.
\item I also find myself wanting to make some pieces that play indefinitely, that are "always on."
\item And to generate visuals \emph{from} my pieces, or to transform visuals \emph{into} pieces.
\item And to drastically simplify the publication of my pieces.
\item And to drastically simplify the infrastructure for doing all of this.
\end{itemize}
\end{frame}

\begin{frame}{Fundamental Dilemma}
\begin{itemize}
\item My \emph{secondary} motivation is ongoing changes in how music is made and disseminated.
\item If musicians use social media to publish music, it will probably be captured by social media (e.g.YouTube and SoundCloud) in order to gather personal   data from them and their audience to use for targeted advertising.
\item This is annoying, and cuts mechanical royalties to practically nothing.
\item Such social media also tend to stylistic conformity and ghettoization.
\end{itemize}
\end{frame}

\begin{frame}{Fundamental Dilemma}
\begin{itemize}
\item But if musicians want to avoid capture by advertisers, things get more difficult:
\begin{itemize}
\item Musicians can use social media that rely on paid subscriptions instead of ads to make money.
\item There are a few social media that are free of both ads and subscriptions.
\item Musicians can pay to set up their own Web sites, which has gotten easier and cheaper to do.
\end{itemize}
\item Yet if musicians create their own Web sites, that opens up wonderful new possibilities.
\end{itemize}
\end{frame}

\begin{frame}{Online Music}
\begin{itemize}
\item Online music can be streamed from files -- or it can be Web pages that actually generate music in real time.
\item Writing pieces as Web pages makes it possible to use all the resources of contemporary Web browsers.
\item Within its security sandbox, \href{https://html5test.co/}{a browser is basically an operating system plus a fast general-purpose programming language plus a game engine plus a high-resolution media player}.
\end{itemize}
\end{frame}

\begin{frame}{Cloud Music}
\begin{itemize}
\item I call music that is played by Web pages on the World Wide Web \emph{cloud music} because it exists only in the cloud, the omnipresent computing infrastructure of the World Wide Web.
\item I feel it is very important to grasp that cloud music is essentially \emph{a new medium for music} and as such,\emph{ it offers new possibilities of musical expression}.
\begin{itemize}
\item A piece can simply be an \href{http://localhost:8000/cloud5-example-score-generator.html}{algorithmically generated piece of fixed duration}, "tape music."
\item A piece can \href{http://localhost:8000/cloud_music_no_2.html}{generate visuals}.
\item \href{http://localhost:8000/cloud_music_no_1.html}{Visuals can generate music}.
\item A piece can \href{http://localhost:8000/cloud5-example-visual-music.html}{interact} with its listener.
\item Listeners can use a piece to \href{http://localhost:8000/cancycle.html}{create new pieces}.
\item \href{http://localhost:8000/cancycle.html}{A piece can play indefinitely (or not).}
\end{itemize}
\end{itemize}
\end{frame}

\begin{frame}{New possibilities}
\begin{itemize}
\item cloud-5 is certainly not the first system that publishes music as Web pages on the World Wide Web:
\begin{itemize}
\item \href{https://gibber.cc/playground/}{Gibber}
\item \href{https://strudel.cc/}{Strudel}
\item \href{https://ide.csound.com/}{Web-IDE}
\item ...and others.
\end{itemize}
\item These systems support multiple users, and provide a playground for shared learning and experimentation..
\item By contrast, cloud-5 is designed to host \emph{permanent} pieces without compromising capabilities.
\item cloud-5 supports algorithmic composition and synthesis at a high level of power, yet it has a low overhead for development and maintenance.
\end{itemize}
\end{frame}

\section{Design}

\begin{frame}{Design}
\begin{itemize}
\item A cloud-5 composition is a single Web page.
\item The score, the orchestra for performing the score, and all required libraries are embedded in that one Web page.
\item The libraries, any musical assets, and all other pre-requisites are \emph{static resources} on the server filesystem.
\item Writing a piece is done using a text editor such as Visual Studio Code.
\item No build system is used.
\item The version-less nature of Internet protocols and Web browsers means that cloud-5 pieces will last indefinitely.
\end{itemize}
\end{frame}

\section{Implementation}
\begin{frame}{Implementation}
\begin{itemize}
\item cloud-5 is a combination of existing components:
\begin{itemize}
\item The sound processing language \href{https://csound.com/}{Csound}, compiled from C to WebAssembly.
\item The live coding system \href{https://strudel.cc}{Strudel}, written in JavaScript.
\item The algorithmic composition system \href{https://github.com/gogins/csound-ac/blob/master/README.md}{CsoundAC}, compiled from C++ to WebAssembly.
\item \href{https://www.shadertoy.com/}{GLSL shaders}.
\end{itemize}
\item Each of these components provides a JavaScript API, and JavaScript in a composition is used to connect and control the components.
\end{itemize}
\end{frame}

\begin{frame}{Composing}
\begin{itemize}
\item A cloud-5 piece is a single Web page assembled from custom HTML elements that are defined in \texttt{cloud-5.js}.
\item The custom elements encapsulate the components (Csound, Strudel, etc.).
\item A high-level menu system provides the shell for a piece.
\item The design is modular. Components can be omitted or added. A naming convention makes it clear what composers can add to the shell in order to make actual pieces.
\item The computer music \href{https://github.com/gogins/csound-ac/tree/master/vscode-playpen}{playpen} extension for Visual Studio Code makes it possible to edit Csound, Strudel, HTML, and JavaScript code in a piece, and immediately play that revision from a local Web server .
\end{itemize}
\end{frame}

\begin{frame}{Publishing}
\begin{itemize}
\item Copy your cloud-5 directory to any Web server. It contains all the libraries and assets that you need.
\item The cloud-5 directory can be the server's root directory, or it can be a subdirectory.
\item The Web server can be a local server for writing pieces or performing live pieces, or a server hosted on the Internet for publishing pieces.
\item Musical compositions, of course, are Web pages in the cloud-5 directory.
\item Modify the index page in your Web server to list and link to pieces to be published.
\item Period.
\end{itemize}
\end{frame}

\end{document}


